\documentclass[a4paper,14pt]{extarticle}
\usepackage[T2A]{fontenc}
\usepackage[utf8]{inputenc}

\usepackage[english,russian]{babel} %используем русский и английский языки с переносами
\usepackage{amssymb,amsfonts,amsmath,mathtext,cite,enumerate,float} %подключаем нужные пакеты расширений

\usepackage[pdftex]{graphicx, color}
\usepackage{subfigure}
\usepackage{color}
\usepackage{listings}

\usepackage{algorithm}
\usepackage{algpseudocode}

\usepackage{pdflscape}
\usepackage{longtable}
\usepackage{float}
\floatname{algorithm}{Листинг}

\DeclareGraphicsExtensions{.png,.pdf,.jpg,.mps,.bmp}
\graphicspath{{pictures/chapter1/}, {pictures/chapter2/}, {pictures/chapter3/}, {pictures/chapter4/}}
\usepackage{bmpsize}

\usepackage[nooneline]{caption} \captionsetup[table]{justification=raggedleft} \captionsetup[figure]{justification=centering,labelsep=endash}

\usepackage[left=2cm,right=2cm,top=2cm,bottom=2cm,bindingoffset=0cm]{geometry} % Меняем поля страницы

\usepackage{setspace}
\onehalfspacing % Полуторный интервал

%
% Начало документа
%
\begin{document}
\renewcommand{\figurename}{Рисунок}
\begin{titlepage}
\newpage

\begin{center}
Государственное образовательное учреждение высшего профессионального образования \\
\vspace{1cm}
\Large<<Московский государственный технический университет имени Н.Э. Баумана>> \\*
(МГТУ им. Н.Э. Баумана) \\*
\hrulefill
\end{center}

\flushright{ФАКУЛЬТЕТ ИНФОРМАТИКИ И СИСТЕМ УПРАВЛЕНИЯ}
\flushright{КАФЕДРА ТЕОРЕТИЧЕСКОЙ ИНФОРМАТИКИ И КОМПЬЮТЕРНЫХ ТЕХНОЛОГИЙ}

\vspace{1em}

\begin{center}
\Large Пояснительная записка \\ к дипломному проекту на тему:
\end{center}

\vspace{2.0em}

\begin{center}
	\Large
    \textsc{ Автоматический анализ тональности сообщений сети Twitter на основе
        комбинированных методов обучения и словарей }
\end{center}

\vspace{4em}

\begin{flushleft}
Студент--дипломник \hrulefill \hspace{1pt} Русначенко Н. Л. \\
\vspace{1.5em}
Научный руководитель \hrulefill \hspace{1pt} Лукашевич Н. В.\\
\vspace{1.5em}
\end{flushleft}

\vspace{\fill}

\begin{center}
Москва 2016
\end{center}

\end{titlepage}

\renewcommand{\baselinestretch}{1.5}

%
% Abstract
%


%
% Содержание
%
\renewcommand{\contentsname}{\centering Содержание}
\tableofcontents

%
% Список литературы
%
\clearpage
\newpage
\bibliographystyle{styles/utf8gost705u}  %% стилевой файл для оформления по ГОСТу
\addcontentsline{toc}{section}{\large Список Литратуры}
\begin{flushleft}
\bibliography{biblio}     %% имя библиографической базы (bib-файла)
\end{flushleft}

\end{document}
