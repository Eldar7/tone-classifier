\documentclass[a4paper,14pt]{extarticle}
\usepackage[T2A]{fontenc}
\usepackage[utf8]{inputenc}

\usepackage[english,russian]{babel} %используем русский и английский языки с переносами
\usepackage{amssymb,amsfonts,amsmath,mathtext,cite,enumerate,float} %подключаем нужные пакеты расширений

\usepackage[pdftex]{graphicx, color}
\usepackage{subfigure}
\usepackage{color}
\usepackage{listings}

\usepackage{algorithm}
\usepackage{algpseudocode}

\usepackage{pdflscape}
\usepackage{longtable}
\usepackage{multirow}
\usepackage[table,xcdraw]{xcolor}
\usepackage{float}
\usepackage{booktabs}
\usepackage{cases}
\floatname{algorithm}{Листинг}

\setlength{\parindent}{1.25cm}      % Абзацный отступ: 1.25 cm
\usepackage{indentfirst}            % 1-й абзац имеет отступ

\DeclareGraphicsExtensions{.png,.pdf,.jpg,.mps,.bmp}
\graphicspath{{pictures/chapter1/}, {pictures/chapter2/}, {pictures/chapter3/}, {pictures/chapter4/}}
\usepackage{bmpsize}
\usepackage[section]{placeins}
\usepackage[nooneline]{caption} \captionsetup[table]{justification=raggedleft} \captionsetup[figure]{justification=centering,labelsep=endash}

\usepackage[left=2cm,right=2cm,top=2cm,bottom=2cm,bindingoffset=0cm]{geometry} % Меняем поля страницы
\usepackage{textcomp,eurosym}
\usepackage{setspace}
\onehalfspacing % Полуторный интервал

\renewcommand{\lstlistingname}{Листинг}

\usepackage{changepage}

\usepackage{tikz} %для рисования графиков
\usepackage{pgfplots}
\usepackage[hidelinks]{hyperref}

% перенос ячеек в таблице
\newcommand{\specialcell}[2][c]{%
\begin{tabular}[#1]{@{}c@{}}#2\end{tabular}}

%
% Начало документа
%
\begin{document}
\renewcommand{\figurename}{Рисунок}
\begin{titlepage}
\newpage

\begin{center}
Государственное образовательное учреждение высшего профессионального образования \\
\vspace{1cm}
\Large<<Московский государственный технический университет имени Н.Э. Баумана>> \\*
(МГТУ им. Н.Э. Баумана) \\*
\hrulefill
\end{center}

\flushright{ФАКУЛЬТЕТ ИНФОРМАТИКИ И СИСТЕМ УПРАВЛЕНИЯ}
\flushright{КАФЕДРА ТЕОРЕТИЧЕСКОЙ ИНФОРМАТИКИ И КОМПЬЮТЕРНЫХ ТЕХНОЛОГИЙ}

\vspace{1em}

\begin{center}
\Large Пояснительная записка \\ к дипломному проекту на тему:
\end{center}

\vspace{2.0em}

\begin{center}
	\Large
    \textsc{ Автоматический анализ тональности сообщений сети Twitter на основе
        комбинированных методов обучения и словарей }
\end{center}

\vspace{4em}

\begin{flushleft}
Студент--дипломник \hrulefill \hspace{1pt} Русначенко Н. Л. \\
\vspace{1.5em}
Научный руководитель \hrulefill \hspace{1pt} Лукашевич Н. В.\\
\vspace{1.5em}
\end{flushleft}

\vspace{\fill}

\begin{center}
Москва 2016
\end{center}

\end{titlepage}

\renewcommand{\baselinestretch}{1.5}

%
% Аннотация
%
\renewcommand{\abstractname}{{Аннотация}}
\renewcommand{\abstractname}{\Huge{Аннотация\\[1.5cm]}}
\begin{abstract}
В данной работе рассматривается применение методов машинного обучения к решению
задачи тональной классификации русскоязычных сообщений сети Twitter в сфере
банков и телекоммуникаций. В качестве подхода к классификации сообщений
рассматривается использование метода <<опорных векторов>> (SVM). Для повышения качества
классификации было объявлено множество вспомогательных признаков, в особенности
признаки на основе лексиконов оценочных слов. Сравниваются результаты в зависимости
от типа обучающей коллекции (сбалансированная/не сбалансированная), от их объемов,
преимущество применения признаков на основе лексиконов к каждому из типов.
Была предпринята попытка участия в соревнованиях по тональной классификации
сообщений (SentiRuEval-2016). Результаты соревнований продемонстрировали устойчивое
третье место по обеим задачам. После соревнований были предприняты попытки улучшения
качества путем более тонкой настройки классификатора, а также извлечением большей
информации из лексиконов. Финальные настройки позволили добиться качества,
сравнимого с победителем соревнования.
\end{abstract}
\clearpage

\clearpage

%
% Содержание
%
\renewcommand{\contentsname}{\centering Содержание}
\tableofcontents

%
% Введение
%
\section{Введение}
% Актуальность, Проблема.
Огромное количество людей по всему миру пользуются микроблоговой сетью
{\it Twitter}.
Среди сообщений сети встречаются такие, авторы которых выражают мнениe и оценку
качества в различных сферах услуг.
Рост скорости появления информации ведет к разработке автоматических систем
тонального анализа.

% Постановка задачи.
Формат большинства сообщений сети представляет собой короткотекстовые посты.
Поэтому под задачей тональной классификации понимается оценка всего
сообщения по отношению к компаниям, в области которых это сообщение написано.
Оценка сообщения может быть положительной, нейтральной, либо негативной.
В русскоязычной сети на протяжении уже нескольких лет, наибольший интерес прикован к
{\it отзывам о банках} и {\it отзывам о телекоммуникационных компаниях}~\cite{tonalityAnalysis}.

% Описание.
В этой работе будет рассмотрен подход, основанный на использовании
словарей тональной окраски термов для устранения проблемы недостатка
данных для обучения модели на основе SVM классификатора.
Будут рассмотрены признаки, в том числе и на основе лексиконов, которые позволяют
существенно повысить качество классификации.
Дополнительно проведем тюннинг классификатора, чтобы понять как влияет величина
отступа для разделения классов на качество работы модели.

% Про результаты.
Подход протестирован на прошедших соревнованиях {\it SentiRuEval-2016}
в рамках конференции {\it Dialogue-2016},
продемонстрировав 3-e место среди всех участников~\cite{dialog2016}.
Рассмотренные в данной статье улучшения позволили приблизиться к результату
победителя.


%
% Обзор предметной области
%
\newcommand\twitter{{\it Twitter }}

\newpage
\section{Обзор предметной области}
    % Объяснить, какие подходы к классификации рассматриваются в этой области.

    \subsection{Подходы к тональной классификации на основе методов машинного обучения}
        \subsubsection{Метод <<Наивного Байеса>>}
        % На основе статьи
        \subsubsection{Метод <<Максимальной энтропии>>}
        % Из статьи
        \subsubsection{Метод <<Опорных векторов>>}
        % 9.1.2

        % SVM with multiple classes
    \subsection{Признаки, используемые для классификации сообщений}
        % Лемматизация термов сообщения на основе метрик
        % bag of words
        % tf-idf
        % tf-idf + вспомогательный словарь.
    \subsection{Способы построения тональных лексиконов}

    \subsection{Оценка качества классификационной модели}
    % Посмотреть обзоры из обзора соревнований.
        \subsubsection{Соревнования в области тональной классификации коротких сообщений}
        % сюда добавить SentiRuEval-2015 года.



%
% Построение классификатора
%
\newpage
    \section{Описание подхода к решению задачи тонального анализа сообщений}
    \label{sec:buildingApproachDescription}
    % Описать классификатор, который планируется использовать
    Основным и единственным параметром для классификаторов на основе методов машинного
    обучения являются вектора, описывающие исходные сообщения.
    Рассмотрим процесс преобразования сообщений в вектор, а также добавим
    дополнительные признаки которые бы способствовали повышению качества
    классификации.

    \subsection{Обработка сообщений}
    \label{sec:buildingMsgProcessing}
    % Векторизация, ее параметры
    Процесс обработки сообщений коллекции сообщений состоит из выполнения
    следующих этапов:
    \begin{enumerate}
        \item Лемматизация слов сообщений с целью получения списка термов;

        \item Все термы сообщения можно разделить на два класса: основные и
            дополнительные. Ко второму классу относятся следующие термы:
            \begin{itemize}
                \item Символы <<Ретвита>> --- термы со значением <<RT>>;
                \item Ссылки на ресурсы сети Интернет --- {\it URL\hspace{1pt}}-адреса;
                \item Имена пользователей сети \twitter --- термы с префиксом <<@>>;
                \item Хэштеги (от англ. {\it Hashtags}) --- термы с префиксом <<\#>>.
            \end{itemize}
            Все остальные термы относятся к классу основных, и на текущем этапе
            термы этого класса остаются в сообщении. Среди множества дополнительных
            термов, в сообщении остаются только хэштеги и {\it URL\hspace{1pt}}-адреса;

        \item Применение предварительно составленного {\it списка стоп слов} ---
            это термы, которые должны быть исключены из сообщения.
            % Описать подробнее (После)
        \item Замена некоторых биграмм и униграмм на тональные префиксы.
            % Описать подробнее (После)
    \end{enumerate}

    Рассмотрим последний пункт обработки сообщения подробнее.
    Для выполнения этого этапа, предполагается что предварительно составлен
    список пар $L_{tone} = {\langle t, s\rangle}$, где $t$ -- терм, а $s$ --
    тональная оценка (<<+>> или <<-->>). На этом этапе, для каждого терма сообщения $t_i \in L_{tone}$
    выполняется замена на соответствующую оценку $s$, которая становится префиксом
    следующего терма $t_{i+1}$. Пример преобразования:
    \begin{center}
        \it
        Сейчас \underline{хорошо} работать \underline{не} то что раньше.

        Сейчас +работать --то что раньше.
    \end{center}

    Среди описанных выше этапов обработки сообщений, обязательным и неизменным
    является только лемматизация слов сообщений. Использование и настройки
    остальных этапов могут быть изменены в зависимости от предпочтений
    пользователя.

    % Написать про tf-idf и про дополнительный словарь
    Для получения весовых коэффициентов термов, предполагается использовать меру {\it tf-idf}.
    %Дополнительно рассмотрим {\it искусственное} расширение исходной
    %обучающей коллекции посредством информации о числе вхождений большинства
    %популярных термов в корпус объемом $10^6$ сообщений. Такое расширение
    %позволит оценить частоты термов на основе коллекции потенциально
    %большего объема чем объем исходной обучающей коллекции.

    \subsection{Вспомогательные признаки классификации}
    \label{sec:buildingAdditionalFeatures}
    % В этот раздел вносим признаки, которые добавлялись к основной векторизации
    Помимо термов, составляющих вектор сообщения, в векторизацию предполагается внести
    все признаки, перечисленные в п. \ref{sec:additionalFeatures}:
    \begin{itemize}
        \item Преобразование эмотиконов сообщения в числовой коэффициент.
        Предварительно составляются два множества эмотиконов: негативные и
        положительные. Для каждого множества определяется $C$ -- суммарное число
        вхождений его элементов в рассматриваемое сообщение.
        Результирующий числовой коэффициент вычисляется по формуле: $C_+ - C_-$;

        \item Подсчет количества термов, записанных в верхнем регистре;

        \item Подсчет числа знаков препинания: <<?>>, <<...>>, <<!>>;

        \item Относительно каждого из предварительно составленных лексиконов, вычисляется
            сумма численных коэффициентов лексикона для термов входящих в рассматриваемое
            сообщение. Если терм отсутствует в лексиконе, то в качестве коэффициента рассматривается
            нулевое значение.

            Дополнительно производится нормализация полученного значения в
            диапазоне $\left[ -1, 1 \right]$ на основе следующего преобразования:
            \begin{numcases}{}
                s = 1 - e^{|x|}, x > 0  {\label{eq:norm1}}  \\
                s = - (1 - e^{|x|}), x < 0 {\label{eq:norm2}}
            \end{numcases}
    \end{itemize}
    \subsection{Коллекции данных для обучения}
    % Здесь рассказываем про коллекции, которые использовались несбалансированные для обучения коллекции
    Для обучения классификатора предполагается использовать соответствующие коллекции
    данных соревнований {\it SentiRuEval}. Поскольку в предоставляемых данных
    число тональных сообщений существенно уступает объему класса нейтральных сообщений,
    то дополнительно планируется создать сбалансированную обучающую коллекцию.
    В работе \cite{diploma2015}, применительно к классификаторам типа {\it Наивного Байеса}
    и {\it SVM}, отмечается существенный прирост качества при использовании
    коллекций сбалансированного типа.

    % Про балансировку коллекций в том числе.
    Предоставляемые коллекции для обучения достаточно велики (см. таблицу
    \ref{table:trainTableStat}) по объему для проведения ручной балансировки. Для решения
    подобной задачи можно воспользоваться готовыми корпусами, в котором сообщения
    автоматически распределены на две тональные группы: положительные и
    негативные. В целях исследования улучшения качества, в качестве источника
    дополнительных тональных сообщений можно использовать корпус коротких
    сообщений сети \twitter, предоставляемый Ю. Рубцовой \cite{rubtsovaCollection}.
    Характеристики такого корпуса представлены в таблице \ref{table:rubtsovaCorpusSpecs}.

    \begin{table}[H]
    \centering
    \caption{Параметры корпуса коротких сообщений сети \twitter, Ю.Рубцова}
    \label{table:rubtsovaCorpusSpecs}
    \begin{tabular}{|c|c|}
    \hline
    Коллекция & Число сообщений \\ \hline
    {\it positive} & 114\hspace{3pt}991 \\ \hline
    {\it negative} & 111\hspace{3pt}923 \\ \hline
    \end{tabular}
    \end{table}

    % (Как производить балансировку)
    Среди всех сообщений каждого класса корпуса коротких сообщений, необходимо
    отобрать небольшой процент тех, которые являются наиболее эмоциональными.
    В общем случае, можно сказать, что требуется функция, которая бы на основе
    слов сообщения, а также эмоциональных коэффициентов каждого из слов,
    позволила бы оценить рассматриваемое предложение.

    Вычисление эмоциональных коэффициентов можно производить
    с использованием лексиконов, на основе подхода, рассмотренного в п. \ref{sec:soEvaluation}.
    Определение оценки сообщения можно произвести одним из следующих способов:
    \begin{itemize}
        \item Вычисление {\it суммы эмоциональных коэффициентов} всех входящих в сообщение слов;
        \item Вычисление {\it максимального коэффициента} среди всех слов сообщения.
    \end{itemize}

    Если полученные значения рассматривать в формате абсолютных величин, то
    введение нижнего порогового значения позволяет отобрать наиболее подходящие
    сообщения. Таким образом можно получить сообщения, которые являются
    наиболее предпочтительными для балансировки обучающих коллекций.


%
% Программная реализация
%
\definecolor{gray}{rgb}{0.4,0.4,0.4}
\definecolor{cyan}{rgb}{0.0,0.6,0.6}
\definecolor{darkblue}{rgb}{0.0,0.0,0.6}

\lstdefinestyle{bash}
{
    language=bash,
    extendedchars=true,
    keywordstyle=\bfseries,
    basicstyle=\footnotesize,
    frame=single,
    showstringspaces=false,
    commentstyle=\color{red},
    identifierstyle=\color{black},
    keywordstyle=\color{blue},
    texcl=true,
    morekeywords={make, git, sudo, pip, npm, ./extract.py},
}

\lstdefinestyle{xml}
{
    extendedchars=true,
    keywordstyle=\bfseries,
    basicstyle=\footnotesize,
    frame=single,
    tabsize=2,
    texcl=true,
    morestring=[b]",
    morestring=[s]{>}{<},
    morecomment=[s]{<?}{?>},
    showstringspaces=false,
    stringstyle=\color{black},
    identifierstyle=\color{darkblue},
    keywordstyle=\color{cyan},
    morekeywords={name},
    commentstyle=\color{gray},
}
\newcommand\xml{{\it XML }}

\newpage
\section{Реализация предложенного подхода}
    % Какие функции выполняет приложение?
    % (Позволяет согласно описанному подходу прозводить семантический анализ
    % сообщений, а также измерять качество работы классификатора на основе описанных
    % оценочных метрик).
    Приложение предоставляет пользователю следующий набор возможностей:
    \begin{enumerate}
        \item Определение тональности сообщений тестовой коллекции по отношению к
        рассматриваемым компаниями. Для каждого сообщения тестовой коллекции,
        классификатор проставляет одну оценку из множества значений $\{1, 0, -1\}$;
        \item Составление собственной тестовой коллекции сообщений, по тематике
        схожих с отзывами о банках или телекоммуникационных компаниях
        (см. п. \ref{sec:tonalityCompetition}).
        \item Вычисление оценки работы классификатора на сообщениях тестовой
        коллекции с помощью сревнения полученных результатов с эталонными значениями.
        %В качестве оценок используются метрики качества $F_{macro}(neg, pos)$ и
        %$F_{micro}(neg, pos)$;
        \item Извлекать актуальные данные из сети \twitter с помощью
        {\it Streaming API}, а также набор инструментов для построения лексиконов
        на их основе.
    \end{enumerate}

    \subsection{Формат представления коллекций}
    % Здесь рассматриваются:
    %  -Формат данных, которые подаются на вход для проведения семантической
    % классификации. (XML)
    %  -Формат выходных данных (оценка качества работы классификатора
    %   (опциально), XML файл с просталвленными оценками)

    Изначально, для описания коллекций используется документ формата \xml,
    который представляет собой экспорт таблицы с сообщениями из СУБД {\it MySQL}.
    Именно в таком формате, организаторы соревнований {\it SentiRuEval}
    распространяют все типы коллекций по каждой тематике. Структура документа
    поделена на две части:
    \begin{enumerate}
        \item Описание таблицы, испольуемой для хранения текстовых сообещний с
            тональными оценками (см. листинг \ref{lst:collectionTableExample});
        \item Сообщения с заполненными в соотствии с форматом таблицы
            парамтерами (см. листинг \ref{lst:collectionMessageExample});
    \end{enumerate}

    \newpage
    \lstset{style=xml}
    \newpage
    \lstinputlisting[caption="Формат описание таблицы для хранения сообщений",
        label={lst:collectionTableExample}]{parts/code/collectionTableExample.xml}

    \lstset{style=xml}
    \lstinputlisting[caption="Формат представления сообщений",
    label={lst:collectionMessageExample}]{parts/code/collectionMessageExample.xml}

    Все параметры сообщения можно разделить на два класса: {\it обязательные} и
    {\it дополнительные}. Каждое сообщение содержит следующий набор
    обязательных параметров (см. листинг \ref{lst:collectionMessageExample}):
    \begin{enumerate}
        \item {\tt id} --- идентификатор сообщения, уникальный в пределах рассматриваемой коллекции;
        \item {\tt twitid} --- уникальный идентификатор сообщения сети \twitter;
        \item {\tt date} --- дата появления сообщения в сети;
        \item {\tt text} --- текст сообщения.
    \end{enumerate}

    Набор дополнительных параметров зависит от предметной области, в которой
    рассматривается сообщение. В общем случае, это организации, по отношению к
    которым рассматривается сообщение.

    \subsection{Формат выходных данных}
    %
    % Выходные данные
    %
    Одним из результатов работы классификатора является \xml
    документ, формат которого совпадает с представлением описания коллекций \ref{...},
    содержащий идентификаторы тестовых сообщений коллекций с проставленными оценками.
    %Пример такого документа представлен в листинге \ref{lst:classifierResults}.
    %\lstset{style=xml}
    %\lstinputlisting[caption="Формат описания результатов классификатора.", label={lst:classifierResults}]{parts/code/classifierResults.xml}

    %В общем случае, результирующий документ
    %может исключать текст соответствующего сообщения и содержать только
    %идентифицирующую информацию. В данном случае текст сообщения присутствует
    %ввиду удобства просмотра результата.

    Оценка работы классификатора представлет собой текстовое сообщение,
    содержащее результаты показаний  $F_{macro}(neg, pos)$ и $F_{micro}(neg, pos)$,
    а также значения пераметров точности и полноты каждого тонального класса, на
    основе которых были вычислены результаты. Пример вывода результата
    качества работы классифкатора представлен в листинге \ref{lst:classifierScores}.

    \lstset{style=bash}
    \lstinputlisting[caption="Формат описания результатов классификатора.", label={lst:classifierScores}]{parts/code/classifierScores.xml}

    \subsection{Разработка тонального класификатора}
        % Архитектура проекта
        % Описать зависимости, а также что требуется реализовать.
        % Для решения задачи сентиментального анализа требуется:
        %   - классификатор (LibSVM)
        %   - модули обработки сообщений (Mystem + тональные префиксы + списки стоп слов)
        %
        % Выбор языка для реализации приложения
        %   - Python
        % хранения коллекций сообщений сети Twitter, а также вспомогательных данных:
        %   - База данных
        \subsubsection{Обработка сообщений сети Twitter}
        % Описать про модуль обработки.

        \subsubsection{Использование LibSVM для классификации методом <<Опорных векторов>>}
        % Описать про использование SVM классификатора

        \subsubsection{Вспомогательные признаки классификации}
        % Модуль добавления признаков в векторизацию сообщений.

    \subsection{Разработка вспомогательных инструментов}
    % Выбор языка для реализации вспомогательных инструментов приложения
    %   - Python
        \subsubsection{Прием текстовых сообщений сети Twitter}

        \subsubsection{Автоматическая тональная разметка принятых сообщений}

        \subsubsection{Создание лексиконов методом определения тональности словосочетаний}

    \subsection{Руководство пользователя}
        \subsubsection{Настройка компонентов приложения}
        % Здесь необходимо описать, какие предварительные действия требуется выполнить,
        % чтобы пользователь смог, пользоваться приложением.
        % Добавление лексиконов в хранилище
        Сборка и эксплуатация приложения проверялась под операционной системой
        {\tt Ubuntu 14.04 x64}. Изначально необходимо выполнить установку зависимостей.
        Список комманд представлен в листинге \ref{lst:dependencies}.
        \lstset{style=bash}
        \lstinputlisting[caption="Установка зависимостей проекта", label={lst:dependencies}]{parts/code/init.sh}

            % Инициализация обучающих коллекций
            По-умолчанию, приложение включает в себя обучающую/тестовую/эталонную
            коллекции соревнований {\tt SentiRuEval} за 2015 и 2016 года. Для
            инициализации коллекций в хранилище, пользователю необходимо выполнить
            последовательность комманд, представленных в листинге \ref{lst:collectionsInit}.
            \lstset{style=bash}
            \lstinputlisting[caption="Инициализация обучающих коллекций", label={lst:collectionsInit}]{parts/code/collectionsInit.sh}

            % Создание сбалансированных обучающих коллекций
            Применение балансировки к обучающей повышает качество оценки сообщений.
            \cite{diploma2015} В листинге \ref{lst:collectionsBalancedInit} приводится
            последовательность комманд, позволяющая создать сбалансированные коллекции
            на основе обучающих коллекций {\tt SentiRuEval}, дополненных сообщениями
            из "Корпуса коротких текстов" \cite{rubtsovaCollection}.
            \lstset{style=bash}
            \lstinputlisting[caption="Создание сбалансированных коллекций.", label={lst:collectionsBalancedInit}]{parts/code/collectionsBalancedInit.sh}

            % Инициализация лексиконов
            В приложении содержит по-умолчанию лексиконы, представленные в текстовом
            формате. Для применения лексиконов в классификаторе, необходимо преобразовать
            текстовые данные в хранилище. Для этого пользователю
            достаточно выполнить следующие действия, представленные в листинге
            \ref{lst:lexiconsInit}.
            \lstset{style=bash}
            \lstinputlisting[caption="Инициализация лексиконов по-умолчанию", label={lst:lexiconsInit}]{parts/code/lexiconsInit.sh}

        \subsubsection{Работа с приложением}

            % Векторизация сообщений обучающей и тестовых коллекций
            Для работы с приложением, пользователю необходимо перейти в каталог проекта {run},
            Каталог содержит настройки векторизации сообщений:
            % Добавить ссылку на описание формата
            \begin{itemize}
                \item {\tt msg.conf} --- описывает настройки семантической обработки текста
                    сообщений в формате {\tt JSON}.
                \item {\tt features.conf} --- предоставляет настройку списока
                    дополнительных признаков векторизации в формате {\tt JSON}.
            \end{itemize}

            Настройки параметров данных для классификации описаны в файле {\tt Makefile}.
            Каждая из настроек описывается именем, которое включает в себя
            характеристику следующих параметров (формат определения значений необходимых
            параметров представлен в листинге \ref{lst:makefile}):

            \begin{enumerate}
                \item {\tt TASK\_TYPE} -- тип решаемой задачи ({bank} или {ttk});
                \item {\tt TEST\_TABLE} -- имя тестовой коллекции в хранилище;
                \item {\tt TRAIN\_TABLE} -- имя обучающей коллекции в хранилище;
                \item {\tt MODEL\_NAME} -- имя используемой модели векторизации сообщений;
                \item {\tt ETALON\_RESULT} -- ссылка на эталонную коллекцию.
            \end{enumerate}

            \lstset{style=bash}
            \lstinputlisting[caption="Пример настройки параметров классификации в Makefile.",
                label={lst:makefile}]{parts/code/makefile.sh}

            Чтобы векторизовать содержимое обучающей и тестовых коллекций, а также
            произвести оценку качества работы классификатора,необходимо выполнить
            комманду:
            \begin{center}
                {\tt make {\{название\_настройки\}}}
            \end{center}

            По завершении процесса, вся необходимая информация по оценке качества
            работы будет отображена на экране в формате, представленном в листинге
            \ref{lst:classifierScores}.
            Для очистки каталога с результатами необходимо использовать следующую
            комманду:
            \begin{center}
                {\tt make clean\_results}
            \end{center}

%        \subsubsection{Создание лексиконов на основе сообщений сети
%            {\tt Twitter} }
%        Помимо существующих в приложении лексиконов, пользователю предоставляется
%        возможность создавать их c нуля. Пользователю необходимо выполнить следующую
%        последовательность действий (см. листинг \ref{lst:lexiconCreation}):
%        \begin{enumerate}
%            \item Подключиться к приему потока сообщений сети {\tt Twitter}. Пользователю
%                необходимо будет предварительно предоставить ключи авторизации;
%            \item На основе принятого потока, отфильтровать тональные сообщения
%            и создать два класса сообщений: положительные и негативные;
%            \item Сгенерировать лексикон на основе полученных классов.
%        \end{enumerate}
%
%        % Листинг выполнения подключения.
%        \lstset{style=bash}
%        \lstinputlisting[caption="Построение пользовательского лексикона.", label={lst:lexiconCreationPart1}]{parts/code/lexiconCreationPart1.sh}
%
%        % Листинг фильтрации собщений.
%
%        % Листинг генерации лексиконов.


%
% Тестирование
%
\begin{table}[htp!]
\caption{Эталонные коллеции предоставленные организаторами.
        {\bf $N_+, N_0, N_-$} -- число сообщений положительного, нейтрального и
        негативного классов соответственно;
        {\bf $\sum$} -- общее число сообщений в коллекции.
        {\bf жирным шрифтом} отмечены классы наибольшего объема в каждой
        коллекции
    }
\label{table:testCollections}
\centering
\begin{tabular}{lcccc}
\hline
\multicolumn{1}{c|}{Название} & \multicolumn{1}{c|}{$N_+$} & \multicolumn{1}{c|}{$N_0$}       & \multicolumn{1}{c|}{$N_-$}      & $\sum$ \\ \hline
$BANK_{15}$                   & 348                        & {\bf 3548}                       & 668                             & 4564   \\
$TCC_{15}$                    & 399                        & {\bf 2601}                       & 916                             & 3916   \\
$BANK_{16}$                   & 312                        & {\bf 2240}                       & 772                             & 3324   \\
$TCC_{16}$                    & 226                        & 1016                             & {\bf 1054}                      & 2296   \\ \hline
\end{tabular}
\end{table}


%
% Технико-экономическое обоснование
%
\newpage
\section{Технико-экономическое обоснование}
    %Разработка программного обеспечения~---~достаточно трудоемкий и длительный
    %процесс, требующий выполнения большого числа разнообразных операций.
    %Организация и планирование процесса разработки программного продукта или
    %программного комплекса при традиционном методе планирования предусматривает
    %выполнение следующих работ:
    %\begin{itemize}
    %    \item формирование состава выполняемых работ и группировка их по стадиям разработки;
    %    \item расчет трудоемкости выполнения работ;
    %    \item установление профессионального состава и расчет количества исполнителей;
    %    \item определение продолжительности выполнения отдельных этапов разработки;
    %    \item построение календарного графика выполнения разработки;
    %    \item контроль выполнения календарного графика.
    %\end{itemize}

    %Далее приведен перечень и состав работ при разработке программного средства
    %для автоматического установления связей между сообщениями твиттера и новостными
    %статьями.
    %Отметим, что процесс разработки программного продукта характеризуется совместной
    %работой разработчиков постановки задач и разработчиков программного обеспечения.

    %Укрупненный состав работ по стадиям разработки программного продукта:
    %\begin{enumerate}
    %    \item Техническое задание;
    %    \item Эскизный проект;
    %        \begin{itemize}
    %            \item Предварительная разработка структуры входных и выходных данных,
    %            \item Разработка общего описания алгоритмов реализации решения задач,
    %            \item Разработка пояснительной записки,
    %            \item Консультации разработчиков постановки задач,
    %            \item Согласование и утверждение эскизного проекта;
    %        \end{itemize}
    %    \item Технический проект;
    %    \item Рабочий проект;
    %    \item Внедрение.
    %\end{enumerate}

    %Трудоемкость разработки программной продукции зависит от ряда факторов,
    %основными из которых являются следующие: степень новизны разрабатываемого
    %программного комплекса, сложность алгоритма его функционирования, объем
    %используемой информации, вид ее представления и способ обработки, а также
    %уровень используемого алгоритмического языка программирования.
    %Чем выше уровень языка, тем трудоемкость меньше.

    По степени новизны разрабатываемый проект относится к \textit{группе новизны A}
    – разработка программных комплексов, требующих использования принципиально
    новых методов их создания, проведения НИР и т.п.

    По степени сложности алгоритма функционирования проект относится к
    \textit{2 группе сложности} - программная продукция, реализующая
    учетно-статистические алгоритмы.

    По виду представления исходной информации и способа ее контроля программный
    продукт относится к \textit{группе 12} - исходная информация представлена в
    форме документов, имеющих различный формат и структуру и \textit{группе 22}
    - требуется печать документов одинаковой формы и содержания, вывод массивов
    данных на машинные носители.

    \subsection{Трудоемкость разработки программной продукции}
    \label{subsec:trud}
        Трудоемкость разработки программной продукции~($\tau_{PP}$) может быть
        определена как сумма величин трудоемкости выполнения отдельных стадий
        разработки программного продукта из выражения:
        $$\tau_{PP} = \tau_{TZ} + \tau_{EP} + \tau_{TP} + \tau_{RP} + \tau_{V},$$
        где $\tau_{TZ}$~---~трудоемкость разработки технического задания на создание программного продукта;
        $\tau_{EP}$~---~трудоемкость разработки эскизного проекта программного продукта;
        $\tau_{TP}$~---~трудоемкость разработки технического проекта программного продукта;
        $\tau_{RP}$~---~трудоемкость разработки рабочего проекта программного продукта;
        $\tau_{V}$~---~трудоемкость внедрения разработанного программного продукта.

        \subsubsection{Трудоемкость разработки технического задания}
            Расчёт трудоёмкости разработки технического задания~($\tau_{PP}$)~[чел.-дни] производится по формуле:
            $$\tau_{TZ} = T^Z_{RZ} + T^Z_{RP},$$
            где $T^Z_{RZ}$~---~затраты времени разработчика постановки задачи на разработку ТЗ,~[чел.-дни];
            $T^Z_{RP}$~---~затраты времени разработчика программного обеспечения на разработку ТЗ,~[чел.-дни].
            Их значения рассчитываются по формулам:
            \begin{equation}
                T^Z_{RZ} = t_Z \cdot  K^Z_{RZ}   \nonumber
            \end{equation}
            \begin{equation}
                T^Z_{RP} = t_Z \cdot  K^Z_{RP}   \nonumber
            \end{equation}
            где $t_Z$~--~норма времени на разработку ТЗ на программный продукт
            (зависит от функционального назначения и степени новизны разрабатываемого
            программного продукта),~[чел.-дни].
            В нашем случае по таблице получаем значение~(группа новизны – А, функциональное назначение – технико-экономическое планирование):
            $$t_Z = 79.$$
            $K^Z_{RZ}$~---~коэффициент, учитывающий удельный вес трудоемкости работ,
            выполняемых разработчиком постановки задачи на стадии ТЗ.
            В нашем случае~(совместная разработка с разработчиком ПО):
            $$K^Z_{RZ} = 0.65.$$
            $K^Z_{RP}$~---~коэффициент, учитывающий удельный вес трудоемкости работ,
            выполняемых разработчиком программного обеспечения на стадии ТЗ.
            В нашем случае~(совместная разработка с разработчиком постановки задач):
            $$K^Z_{RP} = 0.35.$$
            Тогда:
            $$\tau_{TZ} = 79 \cdot ~(0.35 + 0.65) = 79.$$

        \subsubsection{Трудоемкость разработки эскизного проекта}
            Расчёт трудоёмкости разработки эскизного проекта~($\tau_{EP}$)~[чел.-дни] производится по формуле:
            $$\tau_{EP} = T^E_{RZ} + T^E_{RP},$$
            где $T^E_{RZ}$~---~затраты времени разработчика постановки задачи на разработку эскизного проекта~(ЭП),~[чел.-дни];
            $T^E_{RP}$~---~затраты времени разработчика программного обеспечения на разработку ЭП,~[чел.-дни].
            Их значения рассчитываются по формулам:
            $$T^E_{RZ} = t_E \cdot  K^E_{RZ},$$
            $$T^E_{RP} = t_E \cdot  K^E_{RP},$$
            где $t_E$~--~норма времени на разработку ЭП на программный продукт~(зависит от функционального назначения и степени новизны разрабатываемого программного продукта),~[чел.-дни].
            В нашем случае по таблице получаем значение~(группа новизны – А, функциональное назначение – технико-экономическое планирование):
            $$t_E = 175.$$
            $K^E_{RZ}$~---~коэффициент, учитывающий удельный вес трудоемкости работ, выполняемых разработчиком постановки задачи на стадии ЭП.
            В нашем случае~(совместная разработка с разработчиком ПО):
            $$K^E_{RZ} = 0.7.$$
            $K^E_{RP}$~---~коэффициент, учитывающий удельный вес трудоемкости работ, выполняемых разработчиком программного обеспечения на стадии ТЗ.
            В нашем случае~(совместная разработка с разработчиком постановки задач):
            $$K^E_{RP} = 0.3.$$
            Тогда:
            $$\tau_{EP} = 175 \cdot ~(0.3 + 0.7) = 175.$$

        \subsubsection{Трудоемкость разработки технического проекта}
            Трудоёмкость разработки технического проекта~($\tau_{TP}$)~[чел.-дни]
            зависит от функционального назначения программного продукта, количества
            разновидностей форм входной и выходной информации и определяется по формуле:
            $$\tau_{TP} = (t^T_{RZ} + t^T_{RP})\cdot K_V\cdot K_R,$$
            где $t^T_{RZ}$~---~норма времени, затрачиваемого на разработку технического проекта~(ТП) разработчиком постановки задач,~[чел.-дни];
            $t^T_{RP}$~---~норма времени, затрачиваемого на разработку ТП разработчиком ПО,~[чел.-дни].
            По таблице принимаем~(функциональное назначение~---~технико-экономическое планирование,
            количество разновидностей форм входной информации~---~1~(результаты качества работы классификатора),
            количество разновидностей форм выходной информации~---~2~(тональная оценка сообщений, оценка работы классификатора)):
            $$t^T_{RZ} = 38, \hspace{1cm} t^T_{RP} = 9$$
            $K_R$~---~коэффициент учета режима обработки информации. По таблице принимаем~(группа новизны~---~А, режим обработки информации~---~реальный масштаб времени):
            $$K_R = 1.45$$
            $K_V$~---~коэффициент учета вида используемой информации, определяется по формуле:
            $$K_V = \dfrac {K_P\cdot n_P + K_{NS}\cdot n_{NS} + K_B\cdot n_B} {n_P + n_{NS} + n_B },$$
            где $K_P$~---~коэффициент учета вида используемой информации для переменной информации;
            $K_{NS}$~---~коэффициент учета вида используемой информации для нормативно-справочной информации;
            $K_B$~---~коэффициент учета вида используемой информации для баз данных;
            $n_P$~---~количество наборов данных переменной информации;
            $n_{NS}$~---~количество наборов данных нормативно-справочной информации;
            $n_B$~---~количество баз данных.
            Коэффициенты находим по таблице~(группа новизны - А):
            $$K_P=1.70, \hspace{1cm} K_{NS}=1.45, \hspace{1cm} K_B=4.37.$$
            Количество наборов данных, используемых в рамках задачи:
            $$n_P=3, \hspace{1cm} n_{NS}=0, \hspace{1cm} n_B=1.$$
            Находим значение $K_V$:
            $$K_V = \dfrac{1.70\cdot 3+1.45\cdot 0+4.37\cdot 1}{3+0+1}=2.3675.$$
            Тогда:
            $$\tau_{TP} = (38+9)\cdot 2.3675\cdot 1.67 = 185.2$$

        \subsubsection{Трудоемкость разработки рабочего проекта}
            Трудоёмкость разработки рабочего проекта~($\tau_{RP}$)~[чел.-дни]
            зависит от функционального назначения программного продукта, количества
            разновидностей форм входной и выходной информации, сложности алгоритма
            функционирования, сложности контроля информации, степени использования
            готовых программных модулей, уровня алгоритмического языка
            программирования и определяется по формуле:
            $$\tau_{RP} = (t^R_{RZ} + t^R_{RP})\cdot K_K\cdot K_R\cdot K_Y \cdot K_Z\cdot K_{IA},$$
            где $t^R_{RZ}$~---~норма времени, затраченного на разработку рабочего проекта на алгоритмическом языке высокого уровня разработчиком постановки задач,~[чел.-дни].
            $t^R_{RP}$~---~норма времени, затраченного на разработку рабочего проекта на алгоритмическом языке высокого уровня разработчиком ПО,~[чел.-дни].
            По таблице принимаем~(функциональное назначение~---~технико-экономическое планирование,
            количество разновидностей форм входной информации~---~1,
            количество разновидностей форм выходной информации~---~2:
            $$t^R_{RZ} = 11,\hspace{1cm} t^R_{RP} = 68.$$
            $K_K$~---~коэффициент учета сложности контроля информации.
            По таблице принимаем~(степень сложности контроля входной информации~---~11, степень сложности контроля выходной информации~---~22):
            $$K_K = 1.07.$$
            $K_R$~---~коэффициент учета режима обработки информации.
            По таблице принимаем~(группа новизны~---~А, режим обработки информации~---~реальный масштаб времени):
            $$K_R = 1.75.$$
            $K_Y$~---~коэффициент учета уровня используемого алгоритмического языка программирования. По таблице принимаем значение~(интерпретаторы, языковые описатели):
            $$K_Y = 0.8.$$
            $K_Z$~---~коэффициент учета степени использования готовых программных модулей. По таблице принимаем~(использование готовых программных модулей составляет около 50%%):
            $$K_Z = 0.6.$$
            $K_{IA}$~---~коэффициент учета вида используемой информации и сложности алгоритма программного продукта, его значение определяется по формуле:
            $$K_IA = \dfrac {K'_P\cdot n_P + K'_{NS}\cdot n_{NS} + K'_B\cdot n_B} {n_P + n_{NS} + n_B },$$
            где $K'_P$~---~коэффициент учета сложности алгоритма ПП и вида используемой информации для переменной информации;
            $K'_{NS}$~---~коэффициент учета сложности алгоритма ПП и вида используемой информации для нормативно-справочной информации;
            $K'_B$~---~коэффициент учета сложности алгоритма ПП и вида используемой информации для баз данных.
            $n_P$~---~количество наборов данных переменной информации;
            $n_{NS}$~---~количество наборов данных нормативно-справочной информации;
            $n_B$~---~количество баз данных.
            Коэффициенты находим по таблице~(группа новизны - А):
            $$K'_P=2.02, \hspace{1cm} K'_{NS}=1.21, \hspace{1cm} K'_B=1.05.$$
            Количество наборов данных, используемых в рамках задачи:
            $$n_P=3, \hspace{1cm} n_{NS}=0,\hspace{1cm} n_B=1.$$
            Находим значение $K_{IA}$:
            $$K_{IA} = \dfrac{2.02\cdot 3+1.21\cdot 0+1.05\cdot 1}{3+0+1}=1.7775.$$
            Тогда:
            $$\tau_{RP} = (11+68)\cdot 1.00\cdot 1.75\cdot 0.8\cdot 0.6\cdot 1.7775 = 126.2$$


        \subsubsection{Трудоемкость выполнения стадии <<Внедрение>>}
            Расчёт трудоёмкости разработки технического проекта~($\tau_{V}$)~[чел.-дни] производится по формуле:
            $$\tau_{V} = (t^V_{RZ} + t^V_{RP})\cdot K_K\cdot K_R\cdot K_Z,$$
            где $t^V_{RZ}$~---~норма времени, затрачиваемого разработчиком постановки задач на выполнение процедур внедрения программного продукта,~[чел.-дни];
            $t^V_{RP}$~---~норма времени, затрачиваемого разработчиком программного обеспечения на выполнение процедур внедрения программного продукта,~[чел.-дни].
            По таблице принимаем~(функциональное назначение~---~технико-экономическое планирование,
            количество разновидностей форм входной информации~---~1,
            количество разновидностей форм выходной информации~---~2):
            $$t^V_{RZ} = 13,\hspace{1cm} t^V_{RP} = 15.$$
            Коэффициент $K_K$ и $K_Z$ были найдены выше:
            $$K_K=1.07, \hspace{1cm} K_Z=0.6.$$
            $K_R$~---~коэффициент учета режима обработки информации. По таблице принимаем~(группа новизны~---~А, режим обработки информации~---~реальный масштаб времени):
            $$K_R = 1.60.$$
            Тогда:
            $$\tau_{V} = (17+19)\cdot 1.07\cdot 1.60\cdot 0.6= 36.9$$
        Общая трудоёмкость разработки ПП:
        $$\tau_{PP} = 79+175+185.24+126.21+36.98= 602.43$$

            %
            %
            %

    \subsection{Расчет количества исполнителей}
    \label{subsec:slaves}
        Средняя численность исполнителей при реализации проекта разработки и внедрения ПО определяется соотношением:
        $$N=\dfrac {Q_p} {F},$$
        где $t$~---~затраты труда на выполнение проекта (разработка и внедрение ПО); $F$~---~фонд рабочего времени.
        Разработка велась 5 месяцев с 1 января 2016 по 31 мая 2016.
        Из таблицы получаем, что фонд рабочего времени $$F=664.$$
        %\begin{table}[h!]
        %    \caption{Количество рабочих дней по месяцам}
        %    \centering
        %    \label{tabular:work_day}
        %    \begin{tabular}{|c|c|c|}
        %        \hline
        %        \bf{Номер месяца} & \bf{Интервал дней}& \bf{Количество рабочих дней} \\ \hline
        %        1 & 01.01.2016~-~31.01.2016 & 15 \\ \hline
        %        3 & 01.02.2016~-~29.02.2016 & 20 \\ \hline
        %        4 & 01.03.2016~-~31.03.2016 & 21 \\ \hline
        %        5 & 01.04.2016~-~30.04.2016 & 21 \\ \hline
        %        6 & 01.05.2016~-~31.05.2016 & 19 \\ \hline
        %        \multicolumn{2}{|c|}{Итого} & 96 \\ \hline
        %    \end{tabular}
        %\end{table}
        Получаем число исполнителей проекта:
        $$N=\dfrac{4816}{664}=7$$
        Для реализации проекта потребуются 3 старших инженеров и 4 простых инженеров.
        Исходя из того, что в месяце в среднем 22 рабочих дня, то для выполнения
        всего проекта потребуется около 4,7 месяца. На выполнение всего проекта,
        требуется около 22 недель.
    \subsection{Ленточный график выполнения работ}
        На основе рассчитанных в главах \ref{subsec:trud}, \ref{subsec:slaves} трудоёмкости и фонда рабочего времени найдём количество рабочих дней, требуемых для выполнения каждого этапа разработка.
        Результаты приведены в таблице~\ref{tabular:tau_PP}.
        Планирование и контроль хода выполнения разработки проводится по
        ленточному графику выполнения работ.
        \begin{table}[ht!]
            \begin{adjustwidth}{-1cm}{}
            \caption{Ленточный график выполнения работ }
            \centering
            \label{tabular:lenta}
            \begin{tabular}{|c|c|c|c|c|c|c|c|c|c|c|c|c|c|c|c|c|c|c|c|c|c|c|c|c|}
                \hline
                & & \multicolumn{23}{|c|}{Календарные дни} \\ \cline{3-25}
                \parbox[t]{3mm}{\multirow{4}{*}[2em]{\rotatebox[origin=c]{90}{Номер стадии}}} &
                \parbox[t]{3.6mm}{\multirow{4}{*}[5.8em]{\rotatebox[origin=c]{90}{Продолжительность [раб.-дни]}}} &
                \rotatebox[origin=c]{90}{~01.01.2016~-~03.01.2016~} &
                \rotatebox[origin=c]{90}{~04.01.2016~-~10.01.2016~} &
                \rotatebox[origin=c]{90}{~11.01.2016~-~17.01.2016~} &
                \rotatebox[origin=c]{90}{~18.01.2016~-~24.01.2016~} &
                \rotatebox[origin=c]{90}{~25.01.2016~-~31.01.2016~} &
                \rotatebox[origin=c]{90}{~01.02.2016~-~07.02.2016~} &
                \rotatebox[origin=c]{90}{~08.02.2016~-~14.02.2016~} &
                \rotatebox[origin=c]{90}{~15.02.2016~-~21.02.2016~} &
                \rotatebox[origin=c]{90}{~22.02.2016~-~28.02.2016~} &
                \rotatebox[origin=c]{90}{~29.02.2016~-~06.03.2016~} &
                \rotatebox[origin=c]{90}{~07.03.2016~-~13.03.2016~} &
                \rotatebox[origin=c]{90}{~14.03.2016~-~20.03.2016~} &
                \rotatebox[origin=c]{90}{~21.03.2016~-~27.03.2016~} &
                \rotatebox[origin=c]{90}{~28.03.2016~-~03.04.2016~} &
                \rotatebox[origin=c]{90}{~04.04.2016~-~10.04.2016~} &
                \rotatebox[origin=c]{90}{~11.04.2016~-~17.04.2016~} &
                \rotatebox[origin=c]{90}{~18.04.2016~-~24.04.2016~} &
                \rotatebox[origin=c]{90}{~25.04.2016~-~01.05.2016~} &
                \rotatebox[origin=c]{90}{~02.05.2016~-~08.05.2016~} &
                \rotatebox[origin=c]{90}{~08.05.2016~-~15.05.2016~} &
                \rotatebox[origin=c]{90}{~16.05.2016~-~22.05.2016~} &
                \rotatebox[origin=c]{90}{~23.05.2016~-~29.05.2016~} &
                \rotatebox[origin=c]{90}{~30.05.2016~-~31.05.2016~}
                \\ \cline{3-25}
                & & \multicolumn{23}{|c|}{Количество рабочих дней} \\ \cline{3-25}
                  &    & 0 & 0 & 5 & 5 & 5 & 5 & 5 & 6 & 5 & 5 & 3 & 5 & 5 & 5 & 5 & 5 & 5 & 5 & 4 & 4 & 5 & 5 & 2 \\ \hline
                1 & 26 &   &   & 5 & 5 & 5 & 5 & 5 & 1 &   &   &   &   &   &   &   &   &   &   &   &   &   &   &   \\ \hline
                2 & 25 &   &   &   &   &   &   &   & 4 & 5 & 4 & 5 & 5 & 2 &   &   &   &   &   &   &   &   &   &   \\ \hline
                3 & 27 &   &   &   &   &   &   &   &   &   &   &   &   & 3 & 5 & 5 & 5 & 4 &   &   &   &   &   &   \\ \hline
                4 & 18 &   &   &   &   &   &   &   &   &   &   &   &   &   &   &   &   & 1 & 4 & 4 & 4 & 4 & 1 &   \\ \hline
                5 & 9  &   &   &   &   &   &   &   &   &   &   &   &   &   &   &   &   &   &   &   &   &   & 4 & 5 \\ \hline
            \end{tabular}
        %\end{sidewaystable}
        \end{adjustwidth}
        \end{table}


    \subsection{Определение себестоимости программной продукции}
        В работе над проектом используется специальное оборудование~---~персональные
        электронно-вычислительные машины (ПЭВМ) в количестве 1 шт.
        Стоимость одной ПЭВМ составляет 45000 рублей.
        Месячная норма амортизации K = 2,7\%.
        Тогда за 4 месяцев работы расходы на амортизацию составят $P = 45000 \cdot  1 \cdot  0.027 \cdot  4 = 4860$~рублей.
        \begin{table}[ht!]
            %\small
            \centering
            \caption{Итоговая смета затрат}
            \label{tabular:costs}
            \begin{tabular}{|c|c|c|}
                \hline
                \bf{\specialcell{Номер \\ стадии}} & \bf{Название стадии}  &  \bf{\specialcell{Количество\\ рабочих дней}} \\ \hline
                1 & Затраты на оплату труда             & 2 189 265 \\ \hline
                2 & Дополнительная заработная плата     & 218 926 \\ \hline
                3 & Отчисления в ФСС                    & 842 867 \\ \hline
                4 & Амортизация оборудования            & 5 512 \\ \hline
                5 & Накладные расходы                   & 4 597 456 \\ \hline
                \multicolumn{2}{|c|}{Итого}   & 7 854 027 \\ \hline
            \end{tabular}
        \end{table}
        Вывод (см. таблицу \ref{tabular:costs}): затраты на разработку программы
        составляют : $7 854 027$ рублей.

    \subsection{Определение стоимости программной продукции}
        Для определения стоимости работ необходимо на основании плановых сроков
        выполнения работ и численности исполнителей рассчитать общую сумму затрат на
        разработку программного продукта.
        Если ПП рассматривается и создается как продукция производственно-технического назначения,
        допускающая многократное тиражирование и отчуждение от непосредственных разработчиков, то ее цена~$P$ определяется по формуле:
        $$P = K\cdot C+Pr,$$
        где $C$~---~затраты на разработку ПП (сметная себестоимость);
        $K$~---~коэффициент учёта затрат на изготовление опытного образца ПП как продукции производственно-технического назначения~($K=1.1$);
        $Pr$~---~нормативная прибыль, рассчитываемая по формуле:
        $$Pr= \frac {C \cdot  \rho_N} {100},$$
        где $\rho_N$~---~норматив рентабельности, $\rho_N=30\%$;
        Получаем стоимость программного продукта:
        $$ P=1.1\cdot 7 854 027 + 2 356 208\cdot 0.3=10 \hspace{4pt}995  \hspace{4pt}638$$

    \subsection{Расчет экономической эффективности}
        Основными показателями экономической эффективности является чистый дисконтированный доход~(NPV) и срок окупаемости вложенных средств.
        Чистый дисконтированный доход определяется по формуле:
        $$NPV=\sum_{t=0}^T (R_t-Z_t) \cdot  \dfrac{1}{(1+E)^t},$$

        $T$~---~горизонт расчета по месяцам;

        $t$~---~период расчета;

        $R_t$~---~результат, достигнутый на $t$ шаге (стоимость);

        $Z_t$~---~текущие затраты (на шаге $t$);

        $E$~---~приемлемая для инвестора норма прибыли на вложенный капитал.

        На момент начала 2015 года, ставка рефинансирования 8.25\% годовых~(ЦБ РФ),
        что эквивалентно 0.68\% в месяц.
        В виду особенности разрабатываемого продукта он может быть продан лишь
        однократно.
        Отсюда получаем: $E=0.0068.$

        В таблице~\ref{tabular:npv} находится расчёт чистого дисконтированного
        дохода.
        График его изменения приведён на рисунке~\ref{pic:npv}.

        \begin{table}[ht!]
            %\small
            \begin{adjustwidth}{-1cm}{}
            \caption{Расчёт чистого дисконтированного дохода }
            \centering

            \label{tabular:npv}
            \begin{tabular}{|c|c|c|c|c|}
                \hline
                \bf{\specialcell{Месяц}} &
                \bf{\specialcell{Текущие затраты,\\ руб.}} &
                %\bf{\specialcell{Кол-во \\ работников}} &
                \bf{\specialcell{Затраты с начала \\ года, руб.}} &
                \bf{\specialcell{Текущий доход, \\ руб.}} &
                \bf{\specialcell{ЧДД, руб.}} \\ \hline

                Январь  & 939 352   & 939 352   & 0         & -939 352 \\ \hline
                Февраль & 1 783 352 & 2 712 705 & 0         & -2 700 853 \\ \hline
                Март    & 1 783 352 & 4 486 057 & 0         & -4 450 492 \\ \hline
                Апрель  & 1 783 352 & 6 259 410 & 0         & -6 188 439 \\ \hline
                Мая     & 1 353 798 & 7 613 208 & 10 995 638&  3 197 585 \\ \hline

            \end{tabular}
            \end{adjustwidth}
        \end{table}

        \begin{figure}[h!]
            \centering
            \begin{tikzpicture}[scale=1]
                \begin{axis}[ylabel=ЧДД (руб.), xlabel=Количество месяцев с начала проекта, ] %\tiny
                    \addplot coordinates {
                        (1, -939352)
                        (2, -2700853)
                        (3, -4450492)
                        (4, -6188439)
                        (5,  3197585)
                    };
                \end{axis}
            \end{tikzpicture}
            \caption{График изменения чистого дисконтированного дохода}
            \label{pic:npv}
        \end{figure}

        Согласно проведенным расчетам, проект является рентабельным.
        Разрабатываемый проект позволит превысить показатели качества существующих
        систем и сможет их заменить. Итоговый ЧДД составил: 3 197 585 рублей.

    \subsection{Результаты}
        В рамках организационно-экономической части был спланирован календарный
        график проведения работ по созданию подсистемы поддержки проведения
        диагностики промышленных, а также были проведены расчеты по трудозатратам.
        Были исследованы и рассчитаны следующие статьи затрат: материальные затраты;
        заработная плата исполнителей; отчисления на социальное страхование;
        накладные расходы.

        В результате расчетов было получено общее время выполнения проекта,
        которое составило 105 рабочих дней, получены данные по суммарным затратам
        на создание системы для автоматического сопоставления тонального классификатора
        сети \twitter, которые составили 7 613 208 рублей.

        Согласно проведенным расчетам, проект является рентабельным.
        Цена данного программного проекта составила 10 995 638 рублей, итоговый
        ЧДД составил 3 197 585 рублей.


%
% Заключение
%
\newpage
\section*{Заключение}
В статье был описан подход к решению задачи тональной классификации сообщений
сети {\it Twitter} с использованием лексиконов.
Они нашли свое применение в качестве дополнительных признаков в векторе
сообщений, а также для отбора наиболее тональных сообщений с целью увеличения
объема обучающих коллекций.

Качество работы классификатора было протестировано в системах анализа тональности
русского языка {\it SentiRuEval-2015} и {\it SentiRuEval-2016}.
Добавляя в сообщения признаки на их основе, а также применяя лексиконы
для расширения коллекций наиболее тональными сообщениями, удалось добиться
стабильного роста качества классификации.

На последней демонстрируется довольно высокий результат (3-е место) относительно
остальных участников тестирования.
После проведения настройки классификатора, рассматриваемый в статье подход
можно считать одним из наиболее успешных на сегодняшний день для проведения
тональной классификации сообщений различных областей русскоязычной сети
{\it Twitter}.


%
% Приложения
%
\newpage
\addcontentsline{toc}{part}{Приложение А. Извлечение сообщений из социальной сети Twitter}
\section*{Приложение А. Извлечение сообщений из социальной сети Twitter}

\lstdefinestyle{python}
{
    language=python,
    keywordstyle=\bfseries,
    basicstyle=\footnotesize,
    frame=single,
    showstringspaces=false,
    morekeywords={elif, join},
}

\lstinputlisting[]{parts/appendix/twitterConsumer.py}

\newpage
\addcontentsline{toc}{part}{Приложение Б. Наиболее эмоциональные термы корпуса коротких текстов}
\section*{Приложение Б. Наиболее эмоциональные термы корпуса коротких текстов}
\lstset{
numbers=none,
literate={а}{{\selectfont\char224}}1
{б}{{\selectfont\char225}}1
{в}{{\selectfont\char226}}1
{г}{{\selectfont\char227}}1
{д}{{\selectfont\char228}}1
{е}{{\selectfont\char229}}1
{ё}{{\"e}}1
{ж}{{\selectfont\char230}}1
{з}{{\selectfont\char231}}1
{и}{{\selectfont\char232}}1
{й}{{\selectfont\char233}}1
{к}{{\selectfont\char234}}1
{л}{{\selectfont\char235}}1
{м}{{\selectfont\char236}}1
{н}{{\selectfont\char237}}1
{о}{{\selectfont\char238}}1
{п}{{\selectfont\char239}}1
{р}{{\selectfont\char240}}1
{с}{{\selectfont\char241}}1
{т}{{\selectfont\char242}}1
{у}{{\selectfont\char243}}1
{ф}{{\selectfont\char244}}1
{х}{{\selectfont\char245}}1
{ц}{{\selectfont\char246}}1
{ч}{{\selectfont\char247}}1
{ш}{{\selectfont\char248}}1
{щ}{{\selectfont\char249}}1
{ъ}{{\selectfont\char250}}1
{ы}{{\selectfont\char251}}1
{ь}{{\selectfont\char252}}1
{э}{{\selectfont\char253}}1
{ю}{{\selectfont\char254}}1
{я}{{\selectfont\char255}}1
{А}{{\selectfont\char192}}1
{Б}{{\selectfont\char193}}1
{В}{{\selectfont\char194}}1
{Г}{{\selectfont\char195}}1
{Д}{{\selectfont\char196}}1
{Е}{{\selectfont\char197}}1
{Ё}{{\"E}}1
{Ж}{{\selectfont\char198}}1
{З}{{\selectfont\char199}}1
{И}{{\selectfont\char200}}1
{Й}{{\selectfont\char201}}1
{К}{{\selectfont\char202}}1
{Л}{{\selectfont\char203}}1
{М}{{\selectfont\char204}}1
{Н}{{\selectfont\char205}}1
{О}{{\selectfont\char206}}1
{П}{{\selectfont\char207}}1
{Р}{{\selectfont\char208}}1
{С}{{\selectfont\char209}}1
{Т}{{\selectfont\char210}}1
{У}{{\selectfont\char211}}1
{Ф}{{\selectfont\char212}}1
{Х}{{\selectfont\char213}}1
{Ц}{{\selectfont\char214}}1
{Ч}{{\selectfont\char215}}1
{Ш}{{\selectfont\char216}}1
{Щ}{{\selectfont\char217}}1
{Ъ}{{\selectfont\char218}}1
{Ы}{{\selectfont\char219}}1
{Ь}{{\selectfont\char220}}1
{Э}{{\selectfont\char221}}1
{Ю}{{\selectfont\char222}}1
{Я}{{\selectfont\char223}}1
}

\begin{lstlisting}[caption="Эмоционально положительные термы"]
"D", "DDD", "DDDD", "XD", "DD", "царевич", "xD", "DDDDD", "DDDDDD", "уругвай",
"#улыбнуло", "хрещатик", "баярлалаа", "#ЛУЧИРАДОСТИОТРАДОСТИ", "конгениальность",
"#рубль", "аге", "XDD", "позаимствовать", "ржач", "листаться", "бесподобный",
"Microsoft", "позитив", "реквестировать", "улыбнуть", "#музыка", "бугага",
"ехууу";
\end{lstlisting}

\begin{lstlisting}[caption="Эмоционально негативные термы"]
"теракт", "погибший", "ебанат", "цымбаларь", "#сми", "пострадавший", "гренобль",
"цег", "#Вконтакте", "траур", "критический", "однобокость", "михаэль", "поч",
"хнык", "блинн", "таскание", "пичаливать", "грусный", "пичаливать", "печалька",
"почемууу", "покоиться", "пращать", "смертница", "навидеть", "разочарованный",
"мазерать", "о", "скорбеть";
\end{lstlisting}

\newpage
\addcontentsline{toc}{part}{Приложение B. Список используемых тональных префиксов}
\section*{Приложение B. Список используемых тональных префиксов}
\lstset{
numbers=none,
keywordstyle=\bfseries,
basicstyle=\footnotesize,
frame=single,
literate={а}{{\selectfont\char224}}1
{б}{{\selectfont\char225}}1
{в}{{\selectfont\char226}}1
{г}{{\selectfont\char227}}1
{д}{{\selectfont\char228}}1
{е}{{\selectfont\char229}}1
{ё}{{\"e}}1
{ж}{{\selectfont\char230}}1
{з}{{\selectfont\char231}}1
{и}{{\selectfont\char232}}1
{й}{{\selectfont\char233}}1
{к}{{\selectfont\char234}}1
{л}{{\selectfont\char235}}1
{м}{{\selectfont\char236}}1
{н}{{\selectfont\char237}}1
{о}{{\selectfont\char238}}1
{п}{{\selectfont\char239}}1
{р}{{\selectfont\char240}}1
{с}{{\selectfont\char241}}1
{т}{{\selectfont\char242}}1
{у}{{\selectfont\char243}}1
{ф}{{\selectfont\char244}}1
{х}{{\selectfont\char245}}1
{ц}{{\selectfont\char246}}1
{ч}{{\selectfont\char247}}1
{ш}{{\selectfont\char248}}1
{щ}{{\selectfont\char249}}1
{ъ}{{\selectfont\char250}}1
{ы}{{\selectfont\char251}}1
{ь}{{\selectfont\char252}}1
{э}{{\selectfont\char253}}1
{ю}{{\selectfont\char254}}1
{я}{{\selectfont\char255}}1
{А}{{\selectfont\char192}}1
{Б}{{\selectfont\char193}}1
{В}{{\selectfont\char194}}1
{Г}{{\selectfont\char195}}1
{Д}{{\selectfont\char196}}1
{Е}{{\selectfont\char197}}1
{Ё}{{\"E}}1
{Ж}{{\selectfont\char198}}1
{З}{{\selectfont\char199}}1
{И}{{\selectfont\char200}}1
{Й}{{\selectfont\char201}}1
{К}{{\selectfont\char202}}1
{Л}{{\selectfont\char203}}1
{М}{{\selectfont\char204}}1
{Н}{{\selectfont\char205}}1
{О}{{\selectfont\char206}}1
{П}{{\selectfont\char207}}1
{Р}{{\selectfont\char208}}1
{С}{{\selectfont\char209}}1
{Т}{{\selectfont\char210}}1
{У}{{\selectfont\char211}}1
{Ф}{{\selectfont\char212}}1
{Х}{{\selectfont\char213}}1
{Ц}{{\selectfont\char214}}1
{Ч}{{\selectfont\char215}}1
{Ш}{{\selectfont\char216}}1
{Щ}{{\selectfont\char217}}1
{Ъ}{{\selectfont\char218}}1
{Ы}{{\selectfont\char219}}1
{Ь}{{\selectfont\char220}}1
{Э}{{\selectfont\char221}}1
{Ю}{{\selectfont\char222}}1
{Я}{{\selectfont\char223}}1
}



\begin{lstlisting}[caption="Словосочетания используемые для составления тональных префиксов", label={lst:tonePrefixes}]
"имитировать": "-", "даже если": "-", "снижение": "-", "не назвать": "-",
"уменьшение": "-", "много": "+", "весьма": "+", "просто": "+", "сильно": "+",
"спад": "-", "все время": "+", "явный": "+", "снизить": "-", "совершенно": "+"
, "снизиться": "-", "ликвидировать": "-", "значительный": "+","ослабление": "-",
"разрушать": "-", "сильнейший": "+", "нельзя назвать": "-", "разительный": "+",
"колоссально": "+", "ничего": "-", "острый": "+", "повышать": "+",
"ослаблять": "-", "пресекать": "-", "масштабный": "+","повысить": "+",
"невообразимый": "+", "настолько": "+", "якобы": "-", "вырасти": "+",
"редкостный": "+", "сильный": "+", "чрезвычайный": "+", "имитация": "-",
"намного": "+", "заметный рост": "+", "жуть как": "+", "увеличить": "+",
"необычайно": "+", "безусловный": "+", "противодействовать": "-",
"большой": "+", "крайне": "+", "перестать": "-","увеличение": "+",
"масштабность": "+", "разрушение": "-", "безумно": "+", "абсолютный": "+",
"особенно": "+", "жутко": "+", "преодолеть": "-", "запросто": "+",
"отсутствие": "-", "рост": "+", "порядочный": "+", "усилить": "+","вполне": "+",
"не": "-", "недостаточно": "-", "избыток": "+", "лишиться": "-","отменить": "-",
"абсолютно": "+", "дикий": "+", "совсем": "+", "невероятно": "+", "очень": "+",
"лишить": "-", "утратить": "-", "нет никакой": "+", "ослабить": "-",
"запредельный": "+", "утрачивать": "-", "полный": "+", "нарастание": "+",
"по-настоящему": "+", "немыслимый": "+", "гораааздо": "+", "не просто": "+",
"ликвидация": "-", "снять": "-", "преодоление": "-", "избавиться": "-",
"повышение": "+", "падение": "-", "полностью": "+", "вырастать": "+",
"предельно": "+", "нереальный": "+", "противодействие": "-", "разрушить": "-",
"совершенный": "+", "рекордно": "+", "серьезный": "+", "снижаться": "-",
"снимать": "-", "нет": "-", "достаточно": "+", "уменьшать": "-", "отмена": "-",
"поистине": "+", "никакой": "-", "наибольший": "+", "неимоверно": "+",
"уменьшить": "-", "стопроцентный": "+", "гораздо": "+", "нереально": "+",
"отсутствовать": "-", "невиданный": "+", "запрет": "-", "великий вероятность":
"+", "увеличивать": "+", "конец": "-", "лишаться": "-", "защита от": "-",
"избавление": "-", "без": "-", "потеря": "-", "жгучий": "+", "колоссальный":
"+", "терять": "-", "утрата": "-", "самый": "+", "лишать": "-", "потерять": "-",
"совсем-совсем": "+", "дефицит": "-", "чрезвычайно": "+", "нейтрализация": "-"
, "усиливать": "+", "колоссальнейший": "+", "избавляться": "-", "усиление": "+",
\end{lstlisting}


%
% Список литературы
%
\clearpage
\newpage
\bibliographystyle{styles/utf8gost705u}  %% стилевой файл для оформления по ГОСТу
\addcontentsline{toc}{section}{\large Список Литературы}
\begin{flushleft}
\bibliography{biblio}     %% имя библиографической базы (bib-файла)
\end{flushleft}

\end{document}
