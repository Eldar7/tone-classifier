\documentclass[a4paper,11pt]{extarticle}
\usepackage[T2A]{fontenc}
\usepackage[utf8]{inputenc}

\usepackage[english,russian]{babel} %используем русский и английский языки с переносами
\usepackage{amssymb,amsfonts,amsmath,mathtext,cite,enumerate,float} %подключаем нужные пакеты расширений

\usepackage[pdftex]{graphicx, color}
\usepackage{color}
\usepackage{listings}
% подсчет числа изображений, таблиц, ссылок
\usepackage[figure, table]{totalcount}
\usepackage{totcount}
\newtotcounter{citnum} %From the package documentation
\def\oldbibitem{} \let\oldbibitem=\bibitem
\def\bibitem{\stepcounter{citnum}\oldbibitem}
\usepackage{color, colortbl}

\usepackage{algorithm}
\usepackage{algpseudocode}

\usepackage{pdflscape}
\usepackage{longtable}
\usepackage{multirow}
\usepackage[table,xcdraw]{xcolor}
\usepackage{float}
\usepackage{booktabs}
\usepackage{cases}
\floatname{algorithm}{Листинг}
\linespread{1.3}

\setlength{\parindent}{1.25cm}      % Абзацный отступ: 1.25 cm
\usepackage{indentfirst}            % 1-й абзац имеет отступ

\DeclareGraphicsExtensions{.png,.pdf,.jpg,.mps,.bmp}
\graphicspath{{pictures/chapter1/}, {pictures/chapter2/}, {pictures/chapter3/}, {pictures/chapter4/}}
\usepackage{bmpsize}
\usepackage[section]{placeins}
\usepackage[nooneline]{caption}
\captionsetup[table]{justification=justified, position=bottom}
\captionsetup[figure]{justification=justified, labelsep=endash}
\captionsetup[subfigure]{justification=centering}

\usepackage[left=2.5cm,right=2.5cm,top=3.6cm,bottom=3.6cm,bindingoffset=0cm]{geometry} % Меняем поля страницы
\usepackage{textcomp,eurosym}
\usepackage{setspace}
\onehalfspacing % Полуторный интервал

\renewcommand{\lstlistingname}{Листинг}

\usepackage{changepage}

\usepackage{tikz} %для рисования графиков
\usepackage{pgfplots}
\usepackage[hidelinks]{hyperref}
\usepackage{graphicx}
\usepackage{subcaption}

\usepackage{lastpage}

% Перенос ячеек в таблице
\newcommand{\specialcell}[2][c]{%
\begin{tabular}[#1]{@{}c@{}}#2\end{tabular}}

\begin{document}
    % Объявление команд
    \newcommand\twitter{{\it Twitter }}
    \begin{center}
        \bf
        МЕТОДЫ ИНТЕГРАЦИИ ЛЕКСИКОНОВ В МАШИННОЕ ОБУЧЕНИЕ ДЛЯ СИСТЕМ АНАЛИЗА
        ТОНАЛЬНОСТИ
    \end{center}
    \begin{center}
        Русначенко Н. Л. (kolyarus@yandex.ru),
        МГТУ им. Н.Э. Баумана, Москва, Россия

        Лукашевич Н.В. (louk\_nat@mail.ru)
    \end{center}
    \begin{center}
        \bf
        METHODS OF LEXICON INTEGRATION WITH MACHINE LEARNING FOR SENTIMENT
        ANALYSIS SYSTEM
    \end{center}
    \begin{center}
        Rusnachenko N. L. (kolyarus@yandex.ru), BMSTU, Moscow, Russia

        Loukachevitch N.V. (louk\_nat@mail.ru)
    \end{center}


    % Abstract.
    \renewcommand{\abstractname}{\Huge{Аннотация\\[1.5cm]}}
\begin{abstract}
В данной работе рассматривается применение методов машинного обучения к решению
задачи тональной классификации русскоязычных сообщений сети Twitter в сфере
банков и телекоммуникаций. В качестве подхода к классификации сообщений
рассматривается использование метода <<опорных векторов>> (SVM). Для повышения качества
классификации было объявлено множество вспомогательных признаков, в особенности
признаки на основе лексиконов оценочных слов. Сравниваются результаты в зависимости
от типа обучающей коллекции (сбалансированная/не сбалансированная), от их объемов,
преимущество применения признаков на основе лексиконов к каждому из типов.
Была предпринята попытка участия в соревнованиях по тональной классификации
сообщений (SentiRuEval-2016). Результаты соревнований продемонстрировали устойчивое
третье место по обеим задачам. После соревнований были предприняты попытки улучшения
качества путем более тонкой настройки классификатора, а также извлечением большей
информации из лексиконов. Финальные настройки позволили добиться качества,
сравнимого с победителем соревнования.
\end{abstract}
\clearpage


    % Введение.
    \section{Введение}
% Актуальность, Проблема.
Огромное количество людей по всему миру пользуются микроблоговой сетью
{\it Twitter}.
Среди сообщений сети встречаются такие, авторы которых выражают мнениe и оценку
качества в различных сферах услуг.
Рост скорости появления информации ведет к разработке автоматических систем
тонального анализа.

% Постановка задачи.
Формат большинства сообщений сети представляет собой короткотекстовые посты.
Поэтому под задачей тональной классификации понимается оценка всего
сообщения по отношению к компаниям, в области которых это сообщение написано.
Оценка сообщения может быть положительной, нейтральной, либо негативной.
В русскоязычной сети на протяжении уже нескольких лет, наибольший интерес прикован к
{\it отзывам о банках} и {\it отзывам о телекоммуникационных компаниях}~\cite{tonalityAnalysis}.

% Описание.
В этой работе будет рассмотрен подход, основанный на использовании
словарей тональной окраски термов для устранения проблемы недостатка
данных для обучения модели на основе SVM классификатора.
Будут рассмотрены признаки, в том числе и на основе лексиконов, которые позволяют
существенно повысить качество классификации.
Дополнительно проведем тюннинг классификатора, чтобы понять как влияет величина
отступа для разделения классов на качество работы модели.

% Про результаты.
Подход протестирован на прошедших соревнованиях {\it SentiRuEval-2016}
в рамках конференции {\it Dialogue-2016},
продемонстрировав 3-e место среди всех участников~\cite{dialog2016}.
Рассмотренные в данной статье улучшения позволили приблизиться к результату
победителя.


    % Рассматриваемые подходы.
    \section{Подходы к решению задачи тональной классификации с использованием лексиконов}
    % Добавить обзор статей
    \subsection{Автоматическое порождение тональных лексиконов}
    В работе \cite{severyn} описан способ построения лексикона
    на основе метода <<удаленного контроля>>. В качестве исходных сообщений,
    авторы подхода использовали корпус сообщений сети {\it Twitter}, содержащий
    для каждого сообщения {\it метки мнений} ({\it positive} и {\it negative}).
    Такие метки легли в основу обучения контроля полярности классификатора.

    % Описание модели (постановка задачи, раскрытие термина)
    Задача контроля полярности ставится следующим образом. Пусть имеется
    размеченные данные $ \{{\bf x}_i, {\bf y}_i \}_{i=1}^{n}$, на основе
    которых необходимо построить функцию принятия решения
    $f({\bf x}) \to {\bf y}$, которая бы на основе входных параметров
    определяла бы результирующую метку сообщения.
    В частности, авторы использовали линейную модель {\it SVM} классификатора, с
    функцией предсказания следующего вида:
    \begin{equation}
        f = sign(w^T{\bf x} + b)
    \end{equation}

    Где $w$ -- весовые коэффициенты, полученные на основе обучающей коллекции;
    $b$ -- поправочный коэффициент.

    % Алгоритм построения модели
    Авторы статьи предлагают следующий подход автоматического построения
    лексикона и его использования для создания классификационной модели:
    \begin{enumerate}
        \item Составление неразмеченного корпуса сообщений $C$ сети {\it Twitter}.
        \item Для каждого сообщения $t_i \in C$ использовать подсказки
            (хэштеги, эмотиконы) для получения метки ({\it positive} и {\it negative})
            $y_i \in \{-1; +1\}$. Использование эмотиконов вида <<:-)>>, <<:-(>>
            в качестве индикатора выражения автора сообщения в целом.
        \item Извлечение биграмм и униграмм особенности сообщения $t_i$ в
            вектор ${\bf x}_i \in R^{|L|}$, где $L$ -- лексикон, состоящий из
            термов формата биграмм и униграмм;
        \item Построить классификационную модель ${\bf w}$ на основе корпуса
            $C = \{{\bf x}_i, {\bf y}_i \}_{i=1}^{N}$ следующим образом:
        \begin{equation}
            {\bf w} = \sum\limits_{i=1}^{N}\alpha_i y_i {\bf x}_i
        \end{equation}
        Здесь ${\bf x}_i$ выступают в качестве опорных векторов; $y_i$ -- их метки;
        $\alpha_i$ -- параметр краевой задачи, который вносит вклад в
        $w$ в случае когда положителен.
        \item Каждый компонент $w_j$ обученной модели $w$, соответствует компоненту $l_i$
            лексикона $L$, что устанавливает связи с ассоциативной оценкой.
    \end{enumerate}

    % Алгоритм построения лексикона
    Используемый лексикон составлен на основе \twitter корпуса {\it Emoticon140}.
    Метки для корпуса расставлялись на основе эмотиконов, содержащихся в
    тексте сообщений.
    Так, сообщения содержащие эмотиконы типа <<:)>> считались положительными,
    а <<:(>> -- отрицательными.
    Объем корпуса составляет {\it $1.6$ млн. сообщений} с одинаковым распределением
    положительных и негативных сообщений.

    Для составления лексикона используется подход на основе вычисления
    точечной меры взаимоинформации (см. п. \ref{sec:soEvaluation}).
    Дополнительно авторами были составлены собственные лексиконы: MPQA, BingLiu, NRC.
    На этапе предварительного тестирования и настройки модели, отмечается прирост
    качества при увеличении числа используемых лексиконов.

    % Результаты
    Подход демонстрирует хорошие результаты качества работы классификационной
    модели. На соревнованиях {\it Semeval-2014} такой подход занял второе
    место.
    Применительно к коллекциям $SMS$ и $Twitter$, оценка качества работы на
    основе $F-$меры колеблется в диапазоне 66.8 -- 71\%.

    \subsection{Подход к решению задачи тональной классификации, предложенный: Saif M. Mohammad, Svetlana Kiritchenko,
        Xiaodan Zhu}

    Статья \cite{modernApproach} предлагает способы построения {\it SVM}
    классификаторов для решения следующих задач классификации:
    \begin{itemize}
        \item Определения тональной оценки для всего сообщения в целом;
        \item Выявления тональности термов сообщения.
    \end{itemize}

    % Описание подхода (только для случая оценки сообщения в целом)
        % Идея с автоматической генерацией лексикона
    Ключевой идеей повышения качества классификации являются лексиконы,
    которые созданы автоматически на основе коллекции сети {\it Twitter}.
    Для этого, авторы разделили все сообщения корпуса на тональные классы с
    помощью такой метаинформации в сообщениях, как хэштеги.
    Для этого были составлены два множества хэштегов: {\it positive} и
    {\it negative}.

    Объем коллекции, на основе которой составлен лексикон, составляет {\it $775$
    тыс. сообщений}. Для распределения сообщений на классы, авторы использовали
    следующую логику:
    \begin{enumerate}
        \item Если в сообщении встречался хотя бы один хэштег из множества {\it positive}, то
            сообщение считается положительным;
        \item Если в сообщении встречается хотя бы один хэштег из {\it negative} множества, то
            сообщение считается отрицательным.
    \end{enumerate}

    Для построения лексиконов, авторы использовали метод {\it <<тональности словосочетаний>>}
    (см. п. \ref{sec:soEvaluation}, формула \ref{eq:so}) \cite{lexiconSO}.

    В качестве классификатора, авторы использовали {\it SVM} с линейным ядром, и
    параметром штрафной функции $C = 5\cdot 10^{-3}$.
    Векторизация сообщения включала в себя набор дополнительных признаков:
    % Используемые признаки
    \begin{itemize}
        \item {\bf Учет регистра:} количество слов, записанных в верхнем регистре;
        \item {\bf Учет хэштегов:} число входящих в сообщение хэштегов (слов с префиксом <<\#>>);
        \item {\bf Символьные $n$-граммы:} присутствие или отсутствие последовательности
            подряд идущих одинаковых символов длиной в 3, 4, и 5 символов;
        \item {\bf На основе лексиконов:}
            Множество всех лексиконов включает в себя три лексикона, созданных
            в ручную ({\it NRC Emotion Lexicon, MPQA, Bing Liu Lexicon)}, а также
            два автоматически сгенерированных ({\it Hashtag Sentiment Lexicon,
            Sentiment140}). В качестве термов выступали биграммы, униграммы,
            Пусть $w$ -- рассматриваемый токен, $p$ -- полярность. Тогда,
            авторами были составлены следующие признаки:% функции
            \begin{itemize}
                \item Число токенов, для которых выполнено: $score(w, p) > 0$;
                \item Суммарное значение $\sum_{w \in tweet} score(w,p)$;
                \item Вычисление максимума $\max_{w \in tweet} score(w,p)$;
                \item Учет оценки последнего токена, при условии: $score(w,p) > 0$.
            \end{itemize}
        \item {\bf Пунктуация:}
            \begin{itemize}
                \item Подсчет числа последовательностей символов <<!>> и <<?>>,
                    а также подсчет случаев когда они оба символа встречаются в одной последовательности;
                \item Учет содержания знаков <<?>> или <<!>> в последнем терме.
            \end{itemize}
        \item {\bf Эмотиконы:}
            \begin{itemize}
                \item Присутствие или отсутствие положительных и негативных
                    эмотиконов в любой позиции сообщения;
                \item Признак, указывающий на наличие эмотикона (положительного
                    или негативного) в конце сообщения.
            \end{itemize}
        \item {\bf Удлиненные слова:} число слов, в которых символ повторяется
            более двух раз, например <<sooooo>>;
        \item {\bf Суффикс отрицания:} добавляет к слову суффикс {\it <<\_NEG>>},
            в случае, если перед ним имеется конструкция: {\it отрицание +
            знак пунктуации}.
            Под отрицанием понимаются слова вида: {\it<<no>>}, {\it <<shouldn't>>}.
            В качестве знаков пунктуации рассматриваются символы:
            <<,>>, <<.>>, <<:>>, <<;>>, <<!>>, <<?>>.
            Пример преобразования:
            \begin{center}
                \it
                \underline{shouldn't,} perfect \\
                perfect\_NEG
            \end{center}
    \end{itemize}

    % Результаты
    Среди всех команд, принимавших участие в соревновании {\it SemEval-2013 'Detecting
    Sentiment in Twitter'}, описанный подход занял первое место как в задаче
    определения тональности отдельного терма, так и сообщения в целом.
    По задаче оценки термов, авторам подхода удалось добиться оценки
    $F_{score}$ = $69.02\%$. На задаче тонального анализа на уровне сообщения,
    лучшая оценка несколько выше: $88.93\%$.


    % Описание подхода.
    \section{Описание подхода}
    \label{sec:buildingApproachDescription}
    В области классификации сообщений методами машинного обучения, использование
    {\it SVM} классификатора (в сравнении с {\it Naive Bayes}) обусловлено результатами
    тестирования в \cite{svmAdvantages}, которые показывают преимущество SVM на униграммной
    модели обработки сообщений.\footnote{
        Использование униграммной модели упрощает процесс обработки сообщения с
        точки зрения добавления метаинформации, в том числе и на основе лексиконов.
        В текущем подходе все термы, содержащиеся во всех лексиконах, являются
        униграммами.
    }
    Для построения обучающей модели и предсказания
    тональности на ее основе, используется библиотека LibSVM \cite{svmClassifier}.

    \subsection{Обработка сообщений}
    \label{sec:buildingMsgProcessing}
    % Векторизация, ее параметры
    Процесс обработки сообщений коллекции сообщений состоит из следующих этапов:
    \begin{enumerate}
        \item Лемматизация слов сообщений с целью получения списка термов\footnote{
            Использование пакета Yandex Mystem:
            \url{http://tech.yandex.ru/mystem/}
        };

        \item Из сообщения удаляются следующие термы:
            символы <<Ретвита>> (термы со значением <<RT>>),
            имена пользователей сети {\it Twitter} (термы с префиксом <<@>>).
            Таким образом, помимо слов естественных языков в сообщении остаются
            \#хэштеги и {\it URL\hspace{1pt}}-адреса;
        \item Замена некоторых биграмм и униграмм на тональные префиксы.
            Для выполнения этого этапа, используется предварительно составленый
            список пар\footnote{
                [Ссылка на github.]
            }
            $D_{tone} = {\langle t, s\rangle}$, где $t$ -- терм, а $s$ --
            тональная оценка (<<+>> или <<-->>). На этом этапе, для каждого терма $t_i$
            сообщения $m$, такого что $t_i \in D_{tone}$ выполняется замена на соответствующую
            оценку $s$, которая становится префиксом следующего терма $t_{i+1}$.
            Пример преобразования:
            \begin{center}
                \it
                Сейчас \underline{хорошо} работать, \underline{плохо} было раньше.

                Сейчас +работать, -было раньше.
            \end{center}
        \item Для получения весовых коэффициентов термов предполагается
            использовать меру {\it tf-idf}.
    \end{enumerate}

    \subsection{Вспомогательные признаки классификации}
    \label{sec:buildingAdditionalFeatures}
    % В этот раздел вносим признаки, которые добавлялись к основной векторизации
    Помимо термов, составляющих вектор сообщения, предполагается внести
    следующие признаки:
    \begin{itemize}
        \item На основе эмотиконов: предварительно составляются два множества
        эмотиконов (положительные и отрицательные).
        Для каждого множества определяется $C$ -- суммарное число вхождений его
        элементов в рассматриваемое сообщение.
        Результирующий числовой коэффициент вычисляется по формуле: $C_+ - C_-$;

        \item Подсчет количества термов, записанных в ВЕРХНЕМ РЕГИСТРЕ;

        \item Подсчет числа знаков препинания: <<?>>, <<...>>, <<!>>;

        \item Пусть $L$ -- множество составленных лексиконов. Тогда относительно
            каждого лексикона $l_j \in L$ для сообщения $m$, вычисляется:
            \begin{gather}
                \sum\limits_{i=1}^N l_j(t_i), \text{ где } t_i \in m
            \end{gather}
            Если терм $t_i$ отсутствует в лексиконе, то в качестве коэффициента
            рассматривается $l_j(t_i) = 0$.
            Дополнительно выполняется нормализация полученного значения в
            диапазоне $\left[ -1, 1 \right]$ на основе преобразования:
            \begin{numcases}{}
                s = 1 - e^{-|x|}, x > 0  {\label{eq:norm1}}  \\
                s = - (1 - e^{-|x|}), x < 0 {\label{eq:norm2}}
            \end{numcases}
    \end{itemize}

    % Построение обучающих коллекций.
    \subsection{Коллекции данных для обучения}
    \label{sec:train}
    % Здесь рассказываем про коллекции, которые использовались несбалансированные для обучения коллекции
    Для обучения классификатора предполагается использовать соответствующие
    коллекции данных соревнований {\it SentiRuEval} (см. таблицу
    \ref{table:trainCollections}).
    Коллекции $I_{bank}^{16}, \hspace{0.1cm} I_{tcc}^{16}$ являются объединением
    размеченных экспертами данных за 2015 и 2016 года.

    \begin{table}[htp!]
\caption{Обучающие коллеции предоставленные организаторами.
        {\bf $N_+, N_0, N_-$} -- число сообщений положительного, нейтрального и
        негативного классов соответственно;
        {\bf $\sum$} -- общее число сообщений в коллекции;
        нейтральный класс является {\bf наиболее частотным}.
    }
\label{table:trainCollections}
\centering
\begin{tabular}{lcccc}
\hline
\multicolumn{1}{c|}{Название} & \multicolumn{1}{c|}{$N_+$} & \multicolumn{1}{c|}{$N_0$} & \multicolumn{1}{c|}{$N_-$} & $\sum$            \\ \hline
    $I_{bank}^{15}$           & 356                        & \textbf{3\hspace{2pt}482}  & 1\hspace{2pt}077           & 4\hspace{2pt}915  \\
    $I_{tcc}^{15}$            & 956                        & \textbf{2\hspace{2pt}269}  & 1\hspace{2pt}634           & 4\hspace{2pt}859  \\
    $I_{bank}^{16}$           & 1\hspace{2pt}354           & \textbf{4\hspace{2pt}870}  & 2\hspace{2pt}550           & 8\hspace{2pt}783  \\
    $I_{tcc}^{16}$            & 704                        & \textbf{6\hspace{2pt}756}  & 1\hspace{2pt}741           & 9\hspace{2pt}102  \\ \hline
\end{tabular}
\end{table}


    Поскольку в предоставляемых
    данных число тональных сообщений существенно уступает объему класса
    нейтральных сообщений, то дополнительно планируется создать {\it сбалансированную
    обучающую коллекцию}.
    В работе \cite{svmAdvantages}, применительно к классификаторам {\it
    Наивного Байеса} и {\it SVM}, отмечается существенный прирост качества при
    использовании коллекций сбалансированного типа.

    % Про балансировку коллекций в том числе.
    Для решения подобной задачи воспользуемся готовым общедоступным корпусом
    Ю.~Рубцовой\footnote{
        Корпус коротких текстов на русском языке на основе <<постов>> сети
        Twitter: \url{study.mokoron.com}
    }, в котором каждое сообщение автоматически распределено в одну из тональных
    групп:
    {\it positive} и {\it negative}.
    Объем каждого класса такой коллекции составляет {\it $\approx$ 110 тыс.
    сообщений}

    % (Как производить балансировку)
    Для построения сбалансированной коллекции требуется существенно меньшее
    число сообщений чем предлагается в тональном корпусе.
    В связи с этим, выберем небольшой процент наиболее эмоциональных сообщений:
    \begin{enumerate}
        \item Пусть имеется лексикон $l$ на основе корпуса Ю.~Рубцовой для определения
            списка наиболее эмоциональных термов.
        \item Сообщение $m$ будем считать {\it наиболее эмоциональным},
            если для него выполнено следующее условие:
            \begin{gather}
                \max\limits_{i=1..N} |l(t_i)| > B
            \end{gather}
            Где $B$ -- величина порогового значения; \hspace{0.5pt}
            $t_i$ -- термы сообщения $m$; \hspace{0.5pt}
            $N$ -- общее количество термов в сообщении $m$;
    \end{enumerate}

    Таким образом были сбалансированы коллеции таблицы \ref{table:trainCollections}.
    Параметры дополнительно составленных коллекций представлены
    в таблице \ref{table:balancedTrainCollections}.

    \begin{table}[htp!]
\centering
\caption{Сбалансированные обучающие коллекции;
    $N_*$ -- размер класса коллекции;
    $\sum$ -- общее число сообщений в коллекции.
}
\label{table:balancedTrainCollections}
\begin{tabular}{ccccc}
    \hline
    \multicolumn{1}{c|}{\multirow{2}{*}{Название}} & \multicolumn{2}{c|}{BANK}                                & \multicolumn{2}{c}{TCC}               \\ \cline{2-5}
    \multicolumn{1}{c|}{}                          & \multicolumn{1}{c|}{$N_*$} & \multicolumn{1}{c|}{$\sum$} & \multicolumn{1}{c|}{$N_*$} & $\sum$   \\ \hline
    $B_{15}$                                       & $3’400$                    & $10’400$                    & $2’269$                    & $6’888$  \\
    $B_{16}$                                       & $6’756$                    & $20’268$                    & $4’870$                    & $14’610$ \\ \hline
\end{tabular}
\end{table}


    % Построение лексиконов.
    \subsection{Построение лексиконов}

Построение лексикона производится на основе {\it меры взаимной информации}
\cite{lexiconSO}:
\begin{gather}
    PMI(t_1, t_2) = log_2 \dfrac{P(t_1\cap t_2)}{P(t_1)\cdot P(t_2)}
    \label{eq:pmi}
\end{gather}

Такая мера показывает связь $t_1$ с $t_2$, т.е. какова вероятность их связи.
В качестве второго аргумента, рассмотрим метку, которая будет соответствовать
одному из тональных классов:
\begin{itemize}
    \item {\it Excellent} -- положительный оттенком;
    \item {\it Poor} -- негативный оттенком.
\end{itemize}

Таким образом, относительно каждого маркера по формуле \ref{eq:pmi} можно
установить степень связи $t_1$ с положительным и негативным оттенком.
На основе разности значений можно определить тональность терма, или его
{\it семантическую ориентацию} (формула \ref{eq:so}).
\begin{equation}
    \label{eq:so}
    SO(t) = PMI(t, Excellent) - PMI(t, Poor)
\end{equation}

От исходной коллекции $K$, на основе которой будет создан лексикон, требуется
чтобы все сообщения коллекции были размечены по тональным классам.
Т.е., для каждого сообщения должна быть определена метка {\it Excellent} или
{\it Poor}, которая характеризует тональность сообщения в целом.
Такую метку можно проставить автоматически, и применительно к сообщениям
сети {\it Twitter} на основе {\it эмотиконов} или {\it хэштегов} \cite{severyn}.
В результате лексикон строится следующим образом:
\begin{equation}
    S : \{ \left< t, SO(t) \right> | t \in K\}
\end{equation}


Лексиконы были составлены на основе следующих данных (параметры представлены
в таблице \ref{table:createdLexicons}):

\begin{itemize}
    \item $l_1$ -- корпус коротких текстов на русском языке Ю.~Рубцовой\footnote{
        \url{https://github.com/nicolay-r/tone-classifier/tree/2016_jan_contest/data/lexicons}
    };
    \item $l_2$ -- сообщений сети {\it Twitter} за январь 2016 года
        (проставление тональных меток на основе списка позитивных и негативных
        эмотиконов);
    \item $l_3$ -- лексикон $SentiRuLex$ созданный вручную экспертами
        \cite{expertLexicon}.
\end{itemize}

\subsection{Построение лексиконов}

Построение лексикона производится на основе {\it меры взаимной информации}
\cite{lexiconSO}:
\begin{gather}
    PMI(t_1, t_2) = log_2 \dfrac{P(t_1\cap t_2)}{P(t_1)\cdot P(t_2)}
    \label{eq:pmi}
\end{gather}

Такая мера показывает связь $t_1$ с $t_2$, т.е. какова вероятность их связи.
В качестве второго аргумента, рассмотрим метку, которая будет соответствовать
одному из тональных классов:
\begin{itemize}
    \item {\it Excellent} -- положительный оттенком;
    \item {\it Poor} -- негативный оттенком.
\end{itemize}

Таким образом, относительно каждого маркера по формуле \ref{eq:pmi} можно
установить степень связи $t_1$ с положительным и негативным оттенком.
На основе разности значений можно определить тональность терма, или его
{\it семантическую ориентацию} (формула \ref{eq:so}).
\begin{equation}
    \label{eq:so}
    SO(t) = PMI(t, Excellent) - PMI(t, Poor)
\end{equation}

От исходной коллекции $K$, на основе которой будет создан лексикон, требуется
чтобы все сообщения коллекции были размечены по тональным классам.
Т.е., для каждого сообщения должна быть определена метка {\it Excellent} или
{\it Poor}, которая характеризует тональность сообщения в целом.
Такую метку можно проставить автоматически, и применительно к сообщениям
сети {\it Twitter} на основе {\it эмотиконов} или {\it хэштегов} \cite{severyn}.
В результате лексикон строится следующим образом:
\begin{equation}
    S : \{ \left< t, SO(t) \right> | t \in K\}
\end{equation}


Лексиконы были составлены на основе следующих данных (параметры представлены
в таблице \ref{table:createdLexicons}):

\begin{itemize}
    \item $l_1$ -- корпус коротких текстов на русском языке Ю.~Рубцовой\footnote{
        \url{https://github.com/nicolay-r/tone-classifier/tree/2016_jan_contest/data/lexicons}
    };
    \item $l_2$ -- сообщений сети {\it Twitter} за январь 2016 года
        (проставление тональных меток на основе списка позитивных и негативных
        эмотиконов);
    \item $l_3$ -- лексикон $SentiRuLex$ созданный вручную экспертами
        \cite{expertLexicon}.
\end{itemize}

\subsection{Построение лексиконов}

Построение лексикона производится на основе {\it меры взаимной информации}
\cite{lexiconSO}:
\begin{gather}
    PMI(t_1, t_2) = log_2 \dfrac{P(t_1\cap t_2)}{P(t_1)\cdot P(t_2)}
    \label{eq:pmi}
\end{gather}

Такая мера показывает связь $t_1$ с $t_2$, т.е. какова вероятность их связи.
В качестве второго аргумента, рассмотрим метку, которая будет соответствовать
одному из тональных классов:
\begin{itemize}
    \item {\it Excellent} -- положительный оттенком;
    \item {\it Poor} -- негативный оттенком.
\end{itemize}

Таким образом, относительно каждого маркера по формуле \ref{eq:pmi} можно
установить степень связи $t_1$ с положительным и негативным оттенком.
На основе разности значений можно определить тональность терма, или его
{\it семантическую ориентацию} (формула \ref{eq:so}).
\begin{equation}
    \label{eq:so}
    SO(t) = PMI(t, Excellent) - PMI(t, Poor)
\end{equation}

От исходной коллекции $K$, на основе которой будет создан лексикон, требуется
чтобы все сообщения коллекции были размечены по тональным классам.
Т.е., для каждого сообщения должна быть определена метка {\it Excellent} или
{\it Poor}, которая характеризует тональность сообщения в целом.
Такую метку можно проставить автоматически, и применительно к сообщениям
сети {\it Twitter} на основе {\it эмотиконов} или {\it хэштегов} \cite{severyn}.
В результате лексикон строится следующим образом:
\begin{equation}
    S : \{ \left< t, SO(t) \right> | t \in K\}
\end{equation}


Лексиконы были составлены на основе следующих данных (параметры представлены
в таблице \ref{table:createdLexicons}):

\begin{itemize}
    \item $l_1$ -- корпус коротких текстов на русском языке Ю.~Рубцовой\footnote{
        \url{https://github.com/nicolay-r/tone-classifier/tree/2016_jan_contest/data/lexicons}
    };
    \item $l_2$ -- сообщений сети {\it Twitter} за январь 2016 года
        (проставление тональных меток на основе списка позитивных и негативных
        эмотиконов);
    \item $l_3$ -- лексикон $SentiRuLex$ созданный вручную экспертами
        \cite{expertLexicon}.
\end{itemize}

\input{parts/description/tables/lexicons}




    % Тестирование на коллекции 2015 года.
    \section{Тестирование на коллекции SentiRuEval-2015}

В тестировании участвуют прогоны с настройками, представленными в таблице
\ref{table:settings}.

\begin{table}[ht!]
\centering
\caption{Настройки векторизации сообщений}
\label{table:settings}
\begin{tabular}{cccccccccccc}
\hline
\multicolumn{1}{c|}{\multirow{2}{*}{№}} & \multicolumn{1}{c|}{\multirow{2}{*}{Термы}} & \multicolumn{1}{c|}{\multirow{2}{*}{\begin{tabular}[c]{@{}c@{}}Доп.\\ признаки\end{tabular}}} & \multicolumn{3}{c|}{$l_{1}$}                                                                                      & \multicolumn{3}{c|}{$l_{2}$}                                                                                     & \multicolumn{3}{c}{$l_{3}$}                                                                 \\ \cline{4-12}
\multicolumn{1}{c|}{}                   & \multicolumn{1}{c|}{}                       & \multicolumn{1}{c|}{}                                                                        & \multicolumn{1}{c|}{$\sum$} & \multicolumn{1}{c|}{$\max\limits_{1..N}$} & \multicolumn{1}{c|}{$\min\limits_{1..N}$} & \multicolumn{1}{c|}{$\sum$} & \multicolumn{1}{c|}{$\max\limits_{1..N}$} & \multicolumn{1}{c|}{$\min\limits_{1..N}$} & \multicolumn{1}{c|}{$\sum$} & \multicolumn{1}{c|}{$\max\limits_{1..N}$} & $\min\limits_{1..N}$ \\ \hline
1                                       & $\bullet$                                   &                                                                                              &                             &                                          &                                          &                            &                                          &                                          &                            &                                          &                     \\
2                                       & $\bullet$                                   & $\bullet$                                                                                    &                             &                                          &                                          &                            &                                          &                                          &                            &                                          &                     \\
3                                       & $\bullet$                                   & $\bullet$                                                                                    & $\bullet$                   &                                          &                                          &                            &                                          &                                          &                            &                                          &                     \\
4                                       & $\bullet$                                   & $\bullet$                                                                                    & $\bullet$                   &                                          &                                          & $\bullet$                  &                                          &                                          &                            &                                          &                     \\
5                                       & $\bullet$                                   & $\bullet$                                                                                    & $\bullet$                   &                                          &                                          & $\bullet$                  &                                          &                                          & $\bullet$                  &                                          &                     \\
6                                       & $\bullet$                                   & $\bullet$                                                                                    & $\bullet$                   & $\bullet$                                & $\bullet$                                & $\bullet$                  & $\bullet$                                & $\bullet$                                & $\bullet$                  & $\bullet$                                & $\bullet$           \\ \hline
\end{tabular}
\end{table}


Рассмотрим изменение результатов в каждом из прогонов для каждой задачи
(BANK, TKK), в зависимости от типа используемой обучающей коллекции.
Оценка качества работы классификаторов производится по метрике
$F_{1(macro)}^{PN}$.
Результаты приведены в таблице
\ref{table:results2015}.

\begin{table}[ht!]
\centering
\caption{Результаты $F_{1-macro}^{PN}$ тестирования на коллекции {\it SentiRuEval-2016};
        {\bf жирным шрифтом} отмечается лучший результат по каждой задаче.
        }
\label{table:results2016}
\begin{tabular}{ccccc}
\hline
\multicolumn{1}{c|}{\multirow{2}{*}{№}} & \multicolumn{2}{c|}{BANK} & \multicolumn{2}{c}{TCC}\\ \cline{2-5}
\multicolumn{1}{c|}{}                   & \multicolumn{1}{c|}{$I$} & \multicolumn{1}{c|}{$B$} & \multicolumn{1}{c|}{$I$} & $B$             \\ \hline
1                                       & ${48.77}$                              & $45.53$                              & $48.32$                             & $50.90$                    \\
2                                       & ${50.24}$                              & $47.36$                              & $49.48$                             & $50.69$                    \\
3                                       & ${50.28}$                              & $47.53$                              & $48.90$                             & $51.95$                    \\
4                                       & ${50.45}$                              & $47.13$                              & $49.31$                             & $52.72$                    \\
5                                       & ${49.72}$                              & $46.99$                              & $50.55$                             & $\bf{52.90}$               \\
6                                       & ${\bf 51.73}$                          & $50.25$                              & $51.66$                             & $52.63$                    \\ \hline
\end{tabular}
\end{table}
\cellcolor[HTML]{C0C0C0}
\cellcolor[HTML]{C0C0C0}
\cellcolor[HTML]{C0C0C0}
\cellcolor[HTML]{C0C0C0}
\cellcolor[HTML]{C0C0C0}
\cellcolor[HTML]{C0C0C0}

% Вывод о преимуществе применения балансировки.

Весьма неоднозначная картина поведения классификатора наблюдается в зависимости
от рассматриваемой задачи.
Интересно отметить, что сбалансированная коллекция улучшает результат для
задачи BANK (средний прирост $+3.02$), в то время как для классификации
коллекции задачи TCC, наоборот, лучше подходит использование несбалансированной
коллекции.

В остальном, добавление признаков (в том числе и на основе лексиконов)
стабильно улучшает качество работы классификатора.
Наибольший эффект улучшения наблюдается при вычислении
{\it минимума } и {\it максимума} (прогон №6).
% Благодаря введению таких признаков удалось добиться повышения качетва в среднем
% на \

Твиты коллекции TCC классифицируются несколько лучше, и аналогичная особенность
отмечается в \cite{tonalityAnalysis} что объясняется ухудшением ситуации на
Украине в период сбора сообщений для тестовой коллекции задачи BANK.
Так, например слово <<санкции>> может нести негативный характер в тестовой
выборке, в то время как в обучающей колекции аналогичное слово является
нейтральным.

    % Участие в соревнованиях 2016 года.
    \subsection{Участие в соревнованях SentiRuEval-2016}
    \subsubsection{Результаты}
В таблице \ref{table:tkkResult2015} приведены оценки качества работы
классификатора для тестовой коллекции {\it SentiRuEval-2016} \cite{dialog2016}
при использовании настроек предварительного тестирования.
Прогоны с такими настройками показали лучшие результаты среди других
вариаций настроек предложенного подхода (см. таблицы \ref{table:bankResult2016}-\ref{table:tkkResult2016}).

    \begin{table}[!ht]
    \centering
    \caption{Результаты прогонов соревнования (задача BANK, {\it SentiRuEval-2016})}
    \label{table:bankResult2016}
    \begin{tabular}{|c|c|c|c|c|}
    \hline
    \multirow{3}{*}{№} & \multicolumn{4}{c|}{BANK}                                                                                                                                                                                         \\ \cline{2-5}
                       & \multicolumn{2}{c|}{\begin{tabular}[c]{@{}c@{}}Не сбалансированная \\ коллекция (2015 год)\end{tabular}} & \multicolumn{2}{c|}{\begin{tabular}[c]{@{}c@{}}Расширенная сбалансированная \\ коллекция\end{tabular}} \\ \cline{2-5}
                       & $F_{macro}(neg, pos)$                               & $F_{micro}(neg, pos)$                              & $F_{macro}(neg, pos)$                              & $F_{micro}(neg, pos)$                             \\ \hline
    1                  & 0.384                                               & 0.4203                                             & {\bf 0.4536 (+18.1\%)}                                   & {\bf 0.4982 (+18,53\%)}                                 \\ \hline
    2                  & 0.3849                                              & 0.415                                              & {\bf 0.4672 (+20.9\%)}                                   & {\bf 0.5029 (+21,1\%)}                                 \\ \hline
    3                  & 0.3862                                              & 0.4218                                             & {\bf 0.4683 (+21.25\%)}                                  & {\bf 0.5022(+19.06\%)}                                  \\ \hline
    \end{tabular}
    \end{table}

    \begin{table}[!ht]
    \centering
    \caption{Результаты прогонов соревнования (задача TKK, {\it SentiRuEval-2016})}
    \label{table:tkkResult2016}
    \begin{tabular}{|c|c|c|c|c|}
    \hline
    \multirow{3}{*}{№} & \multicolumn{4}{c|}{TKK}                                                                                                                                                                                          \\ \cline{2-5}
                       & \multicolumn{2}{c|}{\begin{tabular}[c]{@{}c@{}}Не сбалансированная \\ коллекция (2015 год)\end{tabular}} & \multicolumn{2}{c|}{\begin{tabular}[c]{@{}c@{}}Расширенная сбалансированная \\ коллекция\end{tabular}} \\ \cline{2-5}
                       & $F_{macro}(neg, pos)$                               & $F_{micro}(neg, pos)$                              & $F_{macro}(neg, pos)$                              & $F_{micro}(neg, pos)$                             \\ \hline
    1                  & 0.4849                                              & 0.641                                              & {\bf 0.5103 (+5.2\%)}                                   & {\bf 0.6509 (+1.5\%) }                                  \\ \hline
    2                  & 0.4832                                              & 0.6473                                             & {\bf 0.5231 (+8.2\%)}                                    & {\bf 0.6508 (+0.5\%)}                                   \\ \hline
    3                  & 0.5099                                              & {\bf 0.677 (+2.0\%)}                                    & {\bf 0.5286 (+3.6\%)}                                    & 0.6632                                            \\ \hline
    \end{tabular}
    \end{table}

    \subsubsection{Улучшение результатов}

После проведения соревнований, в целях повышения качества классификации,
настройки прогонов изменялись в следующих направлениях:
\begin{enumerate}
    \item {\bf Настройка параметра $C$} штрафной функции SVM классификатора.
        По умолчанию $C=1$.
        Среди множества протестированных значений \{$1, 0.75, 0.5, 0.25$\},
        наибольший прирост достигается при {\bf $C = 0.5$} (см. таблицу \ref{table:cParameter}).
    \item {\bf Добавление новых признаков:} вычисление {\it максимальных} и
        {\it минимальных} значений (с учетом нормализации на основе формул
        \ref{eq:norm1}-\ref{eq:norm2}) среди всех термов сообщения по каждому
        из лексиконов.
        Пусть $l$ -- произвольный лексикон из всего множества $L$.
        Тогда относительно рассматриваемого лексикона $l$, для каждого сообщения
        $m = \{t_i\}_{i=1}^n$ вычисляются следующие признаки:
        \begin{equation}
            f_{max_l} = \max_{i=\overline{1 \ldots n}}l(t_i), t_i \in l \nonumber
        \end{equation}

        \begin{equation}
            f_{min_l} = \min\limits_{i=\overline{1 \ldots n}}l(t_i), t_i \in l \nonumber
        \end{equation}
\end{enumerate}

    \begin{table}[ht!]
    \centering
    \caption{Влияние настройки параметра Cost (С=0.5) ({\it SentiRuEval-2016})}
    \label{table:cParameter}
    \begin{tabular}{|c|c|c|c|c|}
    \hline
    \multirow{2}{*}{№} & \multicolumn{2}{c|}{\begin{tabular}[c]{@{}c@{}}BANK\\ (Расширенная сбалансированная\\ коллекция, C=0.5)\end{tabular}} & \multicolumn{2}{c|}{\begin{tabular}[c]{@{}c@{}}TKK\\ (Расширенная сбалансированная\\ коллекция, C=0.5)\end{tabular}} \\ \cline{2-5}
                       & $F_{macro}(neg, pos)$                                     & $F_{micro}(neg, pos)$                                     & $F_{macro}(neg, pos)$                                     & $F_{micro}(neg, pos)$                                    \\ \hline
    1                  & 0.4558 (+0.4\%)                                            & 0.5037 (+1.1\%)                                            & 0.5235 (+2.5\%)                                            & 0.6612 (+1.5\%)                                           \\ \hline
    2                  & {\bf 0.4795 (+2.6\%)}                                            & {\bf 0.5167 (+2.7\%)}                               & 0.5338 (+2.0\%)                                            & 0.6610 (+1.5\%)                                          \\ \hline
    3                  & 0.4768 (+1.8\%)                                            & 0.5135(+2.2\%)                                             & {\bf 0.5452 (+3.1\%) }                                           & {\bf 0.6733 (+1.5\%) }                                          \\ \hline
    \end{tabular}
    \end{table}


    Комбинация рассмотренных выше улучшений привела к настройке {\it финальных
прогонов} (результаты представлены в таблице \ref{table:finalResults}).
Во всех прогонах использовались русскоязычные термы и хэштеги, применялись
тональные префиксы, а также учитывались все признаки. Изменения в настройках
касались только числа используемых лексиконов, а также признаков построенных
на их основе (настройки прогонов):
    \begin{enumerate}
        \item Вычисление суммы, минимума, максимума на основе лексикона №1 (см. таблицу \ref{table:createdLexicons}).
        \item Прогон №1 + признаки суммы, минимума, максимума на основе лексикона №2.
        \item Прогон №2 + признаки суммы, минимума, максимума на основе лексикона №4.
        \item Прогон №3 + признаки минимума и максимума на основе лексиконов №3.
    \end{enumerate}

    \begin{table}[ht!]
    \centering
    \caption{Результаты финального тестирования {\it SentiRuEval-2016}}
    \label{table:finalResults}
    \begin{tabular}{|c|c|c|c|c|}
    \hline
    \multirow{2}{*}{№} & \multicolumn{2}{c|}{\begin{tabular}[c]{@{}c@{}}BANK\\ (Расширенная сбалансированная\\ коллекция, C=0.5)\end{tabular}} & \multicolumn{2}{c|}{\begin{tabular}[c]{@{}c@{}}TKK\\ (Расширенная сбалансированная\\ коллекция, C=0.5)\end{tabular}} \\ \cline{2-5}
                       & $F_{macro}(neg, pos)$                                     & $F_{micro}(neg, pos)$                                     & $F_{macro}(neg, pos)$                                     & $F_{micro}(neg, pos)$                                    \\ \hline
    1                  & 0.4955                                                    & 0.5388                                                    & 0.5259                                                    & 0.6662                                                   \\ \hline
    2                  & 0.5012                                                    & 0.5379                                                    & 0.5283                                                    & 0.6720                                                   \\ \hline
    3                  & \textbf{0.5239}                                           & \textbf{0.5514}                                           & \textbf{0.5453}                                           & \textbf{0.6970}                                          \\ \hline
    4                  & 0.4818                                                    & 0.5238                                                    & 0.5356                                                    & 0.6659                                                   \\ \hline
    \end{tabular}
    \end{table}

    \subsubsection{Вывод}
Использование метаинформации на основе лексиконов стабильно повышает качество
классификации. Наибольший прирост качества достигается в случае, если классификатор
был обучен на коллекции несбалансированного типа (см. таблицу \ref{table:conclusion}\footnote{
Тип обучающей коллекции обозначается следующим образом:
$A$ --- не сбалансированная;
$B$ --- сбалансированная;
$C$ --- расширенная.})\footnote{
В таблице рассматривается прирост качества 3-его прогона по отношению к 1-ому (согласно
таблицам \ref{table:bankResult2015}-\ref{table:tkkResult2015}, и
\ref{table:bankResult2016}-\ref{table:tkkResult2016}).
В скобках указывается общий прирост качества с учетом балансировки.
}.

\begin{table}[ht!]
\begin{adjustwidth}{-1.1cm}{}
\centering
\caption{Рост качества при использовании признаков на основе лексиконов в зависимости от типа обучающей коллекции}
\label{table:conclusion}
\begin{tabular}{|c|c|c|c|c|c|}
\hline
\multicolumn{2}{|c|}{\begin{tabular}[c]{@{}c@{}c@{}}Параметры \\ обучающей \\ коллекции\end{tabular}}                                                        & \multicolumn{2}{c|}{BANK}                                                                                            & \multicolumn{2}{c|}{TKK}                                                                                          \\ \hline
Год                      & Тип         & $F_{macro}(neg, pos)$                                    & $F_{micro}(neg, pos)$                                     & $F_{macro}(neg, pos)$                                   & $F_{micro}(neg, pos)$                                   \\ \hline
\multirow{2}{*}{2015} & $A$                                                    & +12.57\%                                                  & +9.8\%                                                     & +6.8\%                                                   & +3.9\%                                                   \\ \cline{2-6}
                                  & $B$                                                       & \begin{tabular}[c]{@{}c@{}}+3.3\%\\ (+19.0\%)\end{tabular} & \begin{tabular}[c]{@{}c@{}}+4.6\%\\ (+19.8\%)\end{tabular}  & \begin{tabular}[c]{@{}c@{}}+4\%\\ (+3.4\%)\end{tabular}   & \begin{tabular}[c]{@{}c@{}}+2.7\%\\ (+1.9\%)\end{tabular} \\ \hline
\multirow{3}{*}{2016} & $A$                                                    & ---                                                      & ---                                                       & +5.1\%                                                   & +4.6\%                                                   \\ \cline{2-6}
                                  & $B$                                                       & +0.5\%                                                    & +0.03\%                                                    & ---                                                     & ---                                                     \\ \cline{2-6}
& $C$ & \begin{tabular}[c]{@{}c@{}}+4.6\%\\ (+21.95)\end{tabular} & \begin{tabular}[c]{@{}c@{}}+1.9\%\\ (+19.48\%)\end{tabular} & \begin{tabular}[c]{@{}c@{}}+4.1\%\\ (+9.0\%)\end{tabular} & \begin{tabular}[c]{@{}c@{}}+1.8\%\\ (+3.4\%)\end{tabular} \\ \hline
\end{tabular}
\end{adjustwidth}
\end{table}
В таблице \ref{table:conclusion},
значения $(+21.95)$, и $(+19.48)$ последней строки указывают на общий прирост
качества с учетом расширенной балансировки по отношению к обычной балансировке
(тестирование в этих случаях на несбалансированной коллекции не проводилось,
ввиду результатов п. \ref{sec:test2015}, таблица \ref{table:bankResult2015}).
Увеличение числа признаков по каждому из лексиконов позволяет повысить показания
таблицы \ref{table:conclusion}.
В совокупности с использованием сбалансированной обучающей коллекции и настройкой
классификатора, в рамках этой работы были получены максимальные результаты
(см. таблицу \ref{table:finalResults}, прогон №3).
В таблице \ref{table:totalImprovement} представлен прирост качества в результате
использования расширенной сбалансированной коллекции в сочетании с признаками
на основе лексиконов. Наибольший прирост достигается для задачи {\it BANK}.

\begin{table}[!ht]
\centering
\caption{Прирост качества для каждой из задач (сравнение лучшего финального результата с результатами прогона №1, {\it SentiRuEval-2016})}
\label{table:totalImprovement}
\begin{tabular}{|c|c|c|}
\hline
Прирост качества & BANK   & TKK    \\ \hline
Общий            & +36.4\% & +12.4\% \\ \hline
\end{tabular}
\end{table}

Результаты всех участников соревнований представлены в таблице \ref{table:allResults}.
Про некоторых из участников известны настройки прогонов, которые они использовали
для получения соответствующих оценок.
Описание настроек рассматривается в таблице \ref{table:usersSettings}.\footnote{
Таблица с результатами прогонов всех участников соревнований, а также настройками прогонов:
\url{https://docs.google.com/spreadsheets/d/1rCaklClawfnnSnyk4q8CW4zWuO3P38DSrLw_f2wyyjg/edit\#gid=0}
}
Участник под номером 1 соответствует текущей работе.
Соответственно, в первой строке таблицы \ref{table:allResults} рассматривается
максимально полученный в рамках этой работы результат.
Вторая строка таблицы содержит результат, полученный во время проведения соревнований.\footnote{
Разница в результатах, записанная в формате процентов, вычисляется как отношение
результата победителя соревнований к соответвующему резульату участика №1.
}

%В таблице указаны наилучшие результаты для каждой задачи каждого из участников
%по каждой из задачи относительно показания $F_{macro}(neg, pos)$. Жирным шрифтом
%выделен участник, который достиг максимальных результатов по метрикам $F_{macro}(neg, pos)$.

\begin{table}[ht!]
\centering
\caption{
В таблице указаны наилучшие результаты для каждой задачи каждого из участников
по каждой из задачи относительно показания $F_{macro}(neg, pos)$. Жирным шрифтом
выделены результаты участника, который достиг максимальных результатов по
метрикам $F_{macro}(neg, pos)$.
}
\label{table:allResults}
\begin{tabular}{|c|c|c|c|c|}
\hline
                                                                   & \multicolumn{2}{c|}{BANK}                                                        & \multicolumn{2}{c|}{TKK}                                                                                \\ \cline{2-5}
\multirow{-2}{*}{\begin{tabular}[c]{@{}c@{}}Номер\\ участника\end{tabular}}                                                & $F_{macro}(neg, pos)$                  & $F_{micro}(neg, pos)$                   & $F_{macro}(neg, pos)$                                          & $F_{micro}(neg, pos)$                  \\ \hline
\rowcolor[HTML]{FFFFFF}
\cellcolor[HTML]{FFFFFF}{\color[HTML]{333333} }                    & {\color[HTML]{333333} 0.5239 (-5.3\%)} & {\color[HTML]{333333} 0.5514 (-6.6\%)}  & {\color[HTML]{333333} 0.5453 (-2.5\%)}                         & {\color[HTML]{333333} 0.6970 (+5.7\%)} \\ \cline{2-5}
\rowcolor[HTML]{FFFFFF}
\multirow{-2}{*}{\cellcolor[HTML]{FFFFFF}{\color[HTML]{333333} 1}} & 0.4683 (-17.8\%)                       & {\color[HTML]{333333} 0.5022 (-17.1\%)} & {\color[HTML]{333333} 0.5286 (-5.8\%)}                         & 0.6632 (+0.9\%)                        \\ \hline
\textbf{2}                                                         & \textbf{0.5517}                        & {\color[HTML]{333333} \textbf{0.5881}}  & \cellcolor[HTML]{FFFFFF}{\color[HTML]{333333} \textbf{0.5594}} & \textbf{0.6569}                        \\ \hline
3                                                                  & 0.3423                                 & 0.3524                                  & 0.3994                                                         & 0.3994                                 \\ \hline
4                                                                  & 0.3730                                 & 0.3967                                  & 0.4955                                                         & 0.6252                                 \\ \hline
5                                                                  & 0.3859                                 & 0.4640                                  & 0.3499                                                         & 0.4044                                 \\ \hline
6                                                                  & 0.2398                                 & 0.3127                                  & 0.3545                                                         & 0.5263                                 \\ \hline
7                                                                  & 0.471                                  & 0.5128                                  & 0.4842                                                         & 0.6374                                 \\ \hline
8                                                                  & 0.4492                                 & 0.4705                                  & 0.4871                                                         & 0.5745                                 \\ \hline
9                                                                  & 0.5195                                 & 0.5595                                  & 0.5489                                                         & 0.6822                                 \\ \hline
10                                                                 & 0.4659                                 & 0.5053                                  & 0.5055                                                         & 0.6254                                 \\ \hline
\end{tabular}
\end{table}

\begin{table}[ht!]
\centering
\caption{Настройки прогонов участников соревнований}
\label{table:usersSettings}
\begin{tabular}{|c|p{13cm}|}
\hline
\begin{tabular}[c]{@{}c@{}}Номер\\ участника\end{tabular} & \multicolumn{1}{c|}{Настройки прогона}                                                                                                                                                                                                                                                     \\ \hline
1               & Текущая работа (опубликована в формате статьи конференции {\it Диалог-2016} \cite{myArticle}). Описание подхода рассмотрено как в статье \cite{myArticle}, так и в п. \ref{sec:buildingApproachDescription}.  \\ \hline
2               & рекуррентная нейронная сетка ({\it LSTM}). В качестве признаков {\it WORD2VEC}, обученный на внешней коллекции. (Посты и комментарии из социальных сетей)                                                                                                                              \\ \hline
4               & Словарные признаки + признаки мета-классификаторов (логистическая регрессия, ридж-регрессия, классификатор на основе градиентного бустинга и классификатор на основе нейронной сети) и линейный {\it SVM} в качестве классификатора.                                                     \\ \hline
8               & поиск эмоциональных слов по словарю (200 тыс. словоформ), правила их  комбинирования на основе синтаксического анализа; применение онтологических правил, характерных для данной предметной области                                                                                        \\ \hline
9               & {\it SVM}: униграммы, биграммы, словарь РуСентиЛекс, учет частей речи, многозначных слов (автоматический словарь коннотаций по новостям для TKK задачи)                                                                                                                                          \\ \hline
10              & {\it SVM}, в качестве признаков использовались униграммы, подвергшиеся преобразованиям ({\it не + слово = один признак}, множественные повторения символов заменяются двукратным; ссылки, ответы, даты, числа – заменяются паттернами и другие преобразования). Подключение словаря РуСентиЛекс \\ \hline
\end{tabular}
\end{table}



    % Улучшения
    \subsection{Изменение настроек SVM классификатора}
Рассмотрим динамику изменения результатов, в зависимости от величины
отступа для разделения классов SVM классификатором.
В пакете LibSVM отступ изменяется на основе параметра $C$.
% Описать, чего мы хотим добиться
%Так, на основе резульататов таблиц \ref{table:results2015} и \ref{table:results2016}
%наблюдается стабильный рост качества при добавлении новых лексиконов и признаков
%на их основе, то интересно проверить, насколько будут меняться результаты и
%будут ли улучшения при меньших значениях отступа.

На рис. \ref{fig:cost} рассмотрены результаты прогонов с
изменением значения параметра, овечающего за величину отступа в диапазоне
$[0.1, 1]$ с шагом 0.1. Ранее, результаты таблицы \ref{table:results2016}
были получены при максимальном значении параметра в рассматриваемом диапазоне,
т.е. при $C= 1$.
Лучшие результаты по каждому были вынесены в таблицу \ref{table:cost}.

\begin{figure}[!htop] \centering
    \begin{subfigure}[b]{0.35\textwidth}
        % пересчитать
        \includegraphics[width=\textwidth]{pics/2016_ttk_imbalanced.png}
        \caption{$TCC_{16}, \hspace{0.2cm} I_{tcc}^{16}$}
        \label{fig:cost_ttk_2016_imb}
    \end{subfigure}
    ~
    \begin{subfigure}[b]{0.35\textwidth}
        \includegraphics[width=\textwidth]{pics/2016_ttk_balanced.png}
        \caption{$TCC_{16}, \hspace{0.2cm} B_{tcc}^{16}$}
        \label{fig:cost_ttk_2016_b}
    \end{subfigure}

    \begin{subfigure}[b]{0.35\textwidth}
        \includegraphics[width=\textwidth]{pics/2016_bank_imbalanced.png}
        \caption{$BANK_{16},\hspace{0.2cm} I_{bank}^{16}$}
        \label{fig:cost_bank_2016_imb}
    \end{subfigure}
    ~
    \begin{subfigure}[b]{0.35\textwidth}
        \includegraphics[width=\textwidth]{pics/2016_bank_balanced.png}
        \caption{$BANK_{16},\hspace{0.2cm} B_{bank}^{16}$}
        \label{fig:cost_bank_2016_b}
    \end{subfigure}

    \caption{
        Влияние {\it параметра штрафной функции SVM классификатора (C)}
        на результаты прогонов для задачи TCC и BANK на разных обучающих
        коллекциях;
        кривыми на графиках обозначены прогоны, номера которых соответствуют
        настройкам таблицы \ref{table:settings};
        значение параметра измерялось в пределах $[0.1, 1]$ с шагом $0.1$.
    }
    \label{fig:cost}
\end{figure}


\begin{table}[htp!]
\centering
\caption{Наилучшие результаты $F_{1-macro}^{PN}$ рис. \ref{fig:cost}
    для каждой задачи по каждому из прогонов.
    Помимо результата фиксируется значение параметра $Cost$, при котором
    достигается такой резульат.
}
\label{table:cost}
\begin{tabular}{ccccc}
\hline
\multicolumn{1}{c|}{\multirow{2}{*}{№}} & \multicolumn{2}{c|}{BANK}                                                   & \multicolumn{2}{c}{TCC}                                           \\ \cline{2-5}
\multicolumn{1}{c|}{}                   & \multicolumn{1}{c|}{$I_{bank_{/cost}}^{16}$} & \multicolumn{1}{c|}{$B_{bank_{/cost}}^{16}$} & \multicolumn{1}{c|}{$I_{tcc_{/cost}}^{16}$}  & $B_{tcc_{/cost}}^{16}$             \\ \hline
1                                       & $48.93_{0.6}$                                &  $46.71_{0.3}$                               &  $48.57_{0.7}$                                        & $53.50_{0.2}$                                  \\
2                                       & $50.69_{0.6}$                                &  $47.33_{0.5}$                               &  $49.48_{1.0}$                                       & $53.31_{0.3}$                                  \\
3                                       & $51.25_{0.8}$                                &  $48.00_{0.6}$                               &  $49.56_{0.5}$                                          & $53.46_{0.4}$                                  \\
4                                       & $51.52_{0.8}$                                &  $48.43_{0.5}$                               &  $49.84_{0.8}$                                         & $54.25_{0.4}$                                  \\
5                                       & $51.33_{0.7}$                                &  $48.39_{0.5}$                               &  $50.55_{1.0}$                                        & $54.26_{0.5}$                                  \\
6                                       & $\textbf{52.82}_{0.7}$                       &  $51.50_{0.7}$                               &  $52.02_{0.6}$                                        & $\textbf{55.46}_{0.3}$                         \\ \hline
\end{tabular}
\end{table}



Сравнивая результаты относительно использования разных типов обучающих коллекций,
для несбалансированных коллекций (рис. \ref{fig:cost_ttk_2016_imb},
\label{fig:cost_bank_2016_imb})
можно наблюдать резкий спад при использовании малого значения параметра ($C < 0.4$).
Что касается всех результатов,то положительный эффект от добавление признаков
на основе лексиконов сохраняется.
В большинстве случаев лучший результат достигается при подходе №6, что
можно было наблюдать и в таблицах \ref{table:results2015} и \ref{table:results2016}.
Наибольший отрыв такого подохода отмечается на рис. \ref{fig:cost_bank_2016_b}.
Подведя итог, посмотрев на работу классификационных модлелей с точки зрения
изменения настроек, можно рекомендовать использование лексиконов для составления
дополнительных признаков.


    % Сравнение с результатами участников.
    Результаты всех участников соревнований представлены в таблице \ref{table:allResults}.
Про некоторых из участников известны настройки прогонов, которые они использовали
для получения соответствующих оценок.
Описание настроек рассматривается в таблице \ref{table:usersSettings}.\footnote{
Таблица с результатами прогонов всех участников соревнований, а также настройками прогонов:
\url{https://docs.google.com/spreadsheets/d/1rCaklClawfnnSnyk4q8CW4zWuO3P38DSrLw_f2wyyjg/edit\#gid=0}
}
Участник под номером 1 соответствует текущей работе.
Соответственно, в первой строке таблицы \ref{table:allResults} рассматривается
максимально полученный в рамках этой работы результат.
Вторая строка таблицы содержит результат, полученный во время проведения соревнований.\footnote{
Разница в результатах, записанная в формате процентов, вычисляется как отношение
результата победителя соревнований к соответвующему резульату участика №1.
}

%В таблице указаны наилучшие результаты для каждой задачи каждого из участников
%по каждой из задачи относительно показания $F_{macro}(neg, pos)$. Жирным шрифтом
%выделен участник, который достиг максимальных результатов по метрикам $F_{macro}(neg, pos)$.

\begin{table}[ht!]
\centering
\caption{
В таблице указаны наилучшие результаты для каждой задачи каждого из участников
по каждой из задачи относительно показания $F_{macro}(neg, pos)$. Жирным шрифтом
выделены результаты участника, который достиг максимальных результатов по
метрикам $F_{macro}(neg, pos)$.
}
\label{table:allResults}
\begin{tabular}{|c|c|c|c|c|}
\hline
                                                                   & \multicolumn{2}{c|}{BANK}                                                        & \multicolumn{2}{c|}{TKK}                                                                                \\ \cline{2-5}
\multirow{-2}{*}{\begin{tabular}[c]{@{}c@{}}Номер\\ участника\end{tabular}}                                                & $F_{macro}(neg, pos)$                  & $F_{micro}(neg, pos)$                   & $F_{macro}(neg, pos)$                                          & $F_{micro}(neg, pos)$                  \\ \hline
\rowcolor[HTML]{FFFFFF}
\cellcolor[HTML]{FFFFFF}{\color[HTML]{333333} }                    & {\color[HTML]{333333} 0.5239 (-5.3\%)} & {\color[HTML]{333333} 0.5514 (-6.6\%)}  & {\color[HTML]{333333} 0.5453 (-2.5\%)}                         & {\color[HTML]{333333} 0.6970 (+5.7\%)} \\ \cline{2-5}
\rowcolor[HTML]{FFFFFF}
\multirow{-2}{*}{\cellcolor[HTML]{FFFFFF}{\color[HTML]{333333} 1}} & 0.4683 (-17.8\%)                       & {\color[HTML]{333333} 0.5022 (-17.1\%)} & {\color[HTML]{333333} 0.5286 (-5.8\%)}                         & 0.6632 (+0.9\%)                        \\ \hline
\textbf{2}                                                         & \textbf{0.5517}                        & {\color[HTML]{333333} \textbf{0.5881}}  & \cellcolor[HTML]{FFFFFF}{\color[HTML]{333333} \textbf{0.5594}} & \textbf{0.6569}                        \\ \hline
3                                                                  & 0.3423                                 & 0.3524                                  & 0.3994                                                         & 0.3994                                 \\ \hline
4                                                                  & 0.3730                                 & 0.3967                                  & 0.4955                                                         & 0.6252                                 \\ \hline
5                                                                  & 0.3859                                 & 0.4640                                  & 0.3499                                                         & 0.4044                                 \\ \hline
6                                                                  & 0.2398                                 & 0.3127                                  & 0.3545                                                         & 0.5263                                 \\ \hline
7                                                                  & 0.471                                  & 0.5128                                  & 0.4842                                                         & 0.6374                                 \\ \hline
8                                                                  & 0.4492                                 & 0.4705                                  & 0.4871                                                         & 0.5745                                 \\ \hline
9                                                                  & 0.5195                                 & 0.5595                                  & 0.5489                                                         & 0.6822                                 \\ \hline
10                                                                 & 0.4659                                 & 0.5053                                  & 0.5055                                                         & 0.6254                                 \\ \hline
\end{tabular}
\end{table}

\begin{table}[ht!]
\centering
\caption{Настройки прогонов участников соревнований}
\label{table:usersSettings}
\begin{tabular}{|c|p{13cm}|}
\hline
\begin{tabular}[c]{@{}c@{}}Номер\\ участника\end{tabular} & \multicolumn{1}{c|}{Настройки прогона}                                                                                                                                                                                                                                                     \\ \hline
1               & Текущая работа (опубликована в формате статьи конференции {\it Диалог-2016} \cite{myArticle}). Описание подхода рассмотрено как в статье \cite{myArticle}, так и в п. \ref{sec:buildingApproachDescription}.  \\ \hline
2               & рекуррентная нейронная сетка ({\it LSTM}). В качестве признаков {\it WORD2VEC}, обученный на внешней коллекции. (Посты и комментарии из социальных сетей)                                                                                                                              \\ \hline
4               & Словарные признаки + признаки мета-классификаторов (логистическая регрессия, ридж-регрессия, классификатор на основе градиентного бустинга и классификатор на основе нейронной сети) и линейный {\it SVM} в качестве классификатора.                                                     \\ \hline
8               & поиск эмоциональных слов по словарю (200 тыс. словоформ), правила их  комбинирования на основе синтаксического анализа; применение онтологических правил, характерных для данной предметной области                                                                                        \\ \hline
9               & {\it SVM}: униграммы, биграммы, словарь РуСентиЛекс, учет частей речи, многозначных слов (автоматический словарь коннотаций по новостям для TKK задачи)                                                                                                                                          \\ \hline
10              & {\it SVM}, в качестве признаков использовались униграммы, подвергшиеся преобразованиям ({\it не + слово = один признак}, множественные повторения символов заменяются двукратным; ссылки, ответы, даты, числа – заменяются паттернами и другие преобразования). Подключение словаря РуСентиЛекс \\ \hline
\end{tabular}
\end{table}



    % Вывод
    \section*{Заключение}
В статье был описан подход к решению задачи тональной классификации сообщений
сети {\it Twitter} с использованием лексиконов.
Они нашли свое применение в качестве дополнительных признаков в векторе
сообщений, а также для отбора наиболее тональных сообщений с целью увеличения
объема обучающих коллекций.

Качество работы классификатора было протестировано в системах анализа тональности
русского языка {\it SentiRuEval-2015} и {\it SentiRuEval-2016}.
Добавляя в сообщения признаки на их основе, а также применяя лексиконы
для расширения коллекций наиболее тональными сообщениями, удалось добиться
стабильного роста качества классификации.

На последней демонстрируется довольно высокий результат (3-е место) относительно
остальных участников тестирования.
После проведения настройки классификатора, рассматриваемый в статье подход
можно считать одним из наиболее успешных на сегодняшний день для проведения
тональной классификации сообщений различных областей русскоязычной сети
{\it Twitter}.


    % Список литературы
    \bibliographystyle{styles/utf8gost705u}  %% стилевой файл для оформления по ГОСТу
    \addcontentsline{toc}{section}{\large Библиографический список}
    \bibliography{biblio}     %% имя библиографической базы (bib-файла)
\end{document}
