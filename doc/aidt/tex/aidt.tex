\documentclass[a4paper,14pt]{extarticle}
\usepackage[T2A]{fontenc}
\usepackage[utf8]{inputenc}

\usepackage[english,russian]{babel} %используем русский и английский языки с переносами
\usepackage{amssymb,amsfonts,amsmath,mathtext,cite,enumerate,float} %подключаем нужные пакеты расширений

\usepackage[pdftex]{graphicx, color}
\usepackage{color}
\usepackage{listings}
% подсчет числа изображений, таблиц, ссылок
\usepackage[figure, table]{totalcount}
\usepackage{totcount}
\newtotcounter{citnum} %From the package documentation
\def\oldbibitem{} \let\oldbibitem=\bibitem
\def\bibitem{\stepcounter{citnum}\oldbibitem}

\usepackage{algorithm}
\usepackage{algpseudocode}

\usepackage{pdflscape}
\usepackage{longtable}
\usepackage{multirow}
\usepackage[table,xcdraw]{xcolor}
\usepackage{float}
\usepackage{booktabs}
\usepackage{cases}
\floatname{algorithm}{Листинг}

\setlength{\parindent}{1.25cm}      % Абзацный отступ: 1.25 cm
\usepackage{indentfirst}            % 1-й абзац имеет отступ

\DeclareGraphicsExtensions{.png,.pdf,.jpg,.mps,.bmp}
\graphicspath{{pictures/chapter1/}, {pictures/chapter2/}, {pictures/chapter3/}, {pictures/chapter4/}}
\usepackage{bmpsize}
\usepackage[section]{placeins}
\usepackage[nooneline]{caption} \captionsetup[table]{justification=raggedleft} \captionsetup[figure]{justification=justified,labelsep=endash}

\usepackage[left=2cm,right=2cm,top=2cm,bottom=2cm,bindingoffset=0cm]{geometry} % Меняем поля страницы
\usepackage{textcomp,eurosym}
\usepackage{setspace}
\onehalfspacing % Полуторный интервал

\renewcommand{\lstlistingname}{Листинг}

\usepackage{changepage}

\usepackage{tikz} %для рисования графиков
\usepackage{pgfplots}
\usepackage[hidelinks]{hyperref}
\usepackage{graphicx}
\usepackage{subcaption}

\usepackage{lastpage}

% Перенос ячеек в таблице
\newcommand{\specialcell}[2][c]{%
\begin{tabular}[#1]{@{}c@{}}#2\end{tabular}}

\begin{document}
    % Объявление команд
    \newcommand\twitter{{\it Twitter }}

    % Рассматриваемые подходы
    \section{Подходы к решению задачи тональной классификации с использованием лексиконов}
    % Добавить обзор статей
    \subsection{Автоматическое порождение тональных лексиконов}
    В работе \cite{severyn} описан способ построения лексикона
    на основе метода <<удаленного контроля>>. В качестве исходных сообщений,
    авторы подхода использовали корпус сообщений сети {\it Twitter}, содержащий
    для каждого сообщения {\it метки мнений} ({\it positive} и {\it negative}).
    Такие метки легли в основу обучения контроля полярности классификатора.

    % Описание модели (постановка задачи, раскрытие термина)
    Задача контроля полярности ставится следующим образом. Пусть имеется
    размеченные данные $ \{{\bf x}_i, {\bf y}_i \}_{i=1}^{n}$, на основе
    которых необходимо построить функцию принятия решения
    $f({\bf x}) \to {\bf y}$, которая бы на основе входных параметров
    определяла бы результирующую метку сообщения.
    В частности, авторы использовали линейную модель {\it SVM} классификатора, с
    функцией предсказания следующего вида:
    \begin{equation}
        f = sign(w^T{\bf x} + b)
    \end{equation}

    Где $w$ -- весовые коэффициенты, полученные на основе обучающей коллекции;
    $b$ -- поправочный коэффициент.

    % Алгоритм построения модели
    Авторы статьи предлагают следующий подход автоматического построения
    лексикона и его использования для создания классификационной модели:
    \begin{enumerate}
        \item Составление неразмеченного корпуса сообщений $C$ сети {\it Twitter}.
        \item Для каждого сообщения $t_i \in C$ использовать подсказки
            (хэштеги, эмотиконы) для получения метки ({\it positive} и {\it negative})
            $y_i \in \{-1; +1\}$. Использование эмотиконов вида <<:-)>>, <<:-(>>
            в качестве индикатора выражения автора сообщения в целом.
        \item Извлечение биграмм и униграмм особенности сообщения $t_i$ в
            вектор ${\bf x}_i \in R^{|L|}$, где $L$ -- лексикон, состоящий из
            термов формата биграмм и униграмм;
        \item Построить классификационную модель ${\bf w}$ на основе корпуса
            $C = \{{\bf x}_i, {\bf y}_i \}_{i=1}^{N}$ следующим образом:
        \begin{equation}
            {\bf w} = \sum\limits_{i=1}^{N}\alpha_i y_i {\bf x}_i
        \end{equation}
        Здесь ${\bf x}_i$ выступают в качестве опорных векторов; $y_i$ -- их метки;
        $\alpha_i$ -- параметр краевой задачи, который вносит вклад в
        $w$ в случае когда положителен.
        \item Каждый компонент $w_j$ обученной модели $w$, соответствует компоненту $l_i$
            лексикона $L$, что устанавливает связи с ассоциативной оценкой.
    \end{enumerate}

    % Алгоритм построения лексикона
    Используемый лексикон составлен на основе \twitter корпуса {\it Emoticon140}.
    Метки для корпуса расставлялись на основе эмотиконов, содержащихся в
    тексте сообщений.
    Так, сообщения содержащие эмотиконы типа <<:)>> считались положительными,
    а <<:(>> -- отрицательными.
    Объем корпуса составляет {\it $1.6$ млн. сообщений} с одинаковым распределением
    положительных и негативных сообщений.

    Для составления лексикона используется подход на основе вычисления
    точечной меры взаимоинформации (см. п. \ref{sec:soEvaluation}).
    Дополнительно авторами были составлены собственные лексиконы: MPQA, BingLiu, NRC.
    На этапе предварительного тестирования и настройки модели, отмечается прирост
    качества при увеличении числа используемых лексиконов.

    % Результаты
    Подход демонстрирует хорошие результаты качества работы классификационной
    модели. На соревнованиях {\it Semeval-2014} такой подход занял второе
    место.
    Применительно к коллекциям $SMS$ и $Twitter$, оценка качества работы на
    основе $F-$меры колеблется в диапазоне 66.8 -- 71\%.

    \subsection{Подход к решению задачи тональной классификации, предложенный: Saif M. Mohammad, Svetlana Kiritchenko,
        Xiaodan Zhu}

    Статья \cite{modernApproach} предлагает способы построения {\it SVM}
    классификаторов для решения следующих задач классификации:
    \begin{itemize}
        \item Определения тональной оценки для всего сообщения в целом;
        \item Выявления тональности термов сообщения.
    \end{itemize}

    % Описание подхода (только для случая оценки сообщения в целом)
        % Идея с автоматической генерацией лексикона
    Ключевой идеей повышения качества классификации являются лексиконы,
    которые созданы автоматически на основе коллекции сети {\it Twitter}.
    Для этого, авторы разделили все сообщения корпуса на тональные классы с
    помощью такой метаинформации в сообщениях, как хэштеги.
    Для этого были составлены два множества хэштегов: {\it positive} и
    {\it negative}.

    Объем коллекции, на основе которой составлен лексикон, составляет {\it $775$
    тыс. сообщений}. Для распределения сообщений на классы, авторы использовали
    следующую логику:
    \begin{enumerate}
        \item Если в сообщении встречался хотя бы один хэштег из множества {\it positive}, то
            сообщение считается положительным;
        \item Если в сообщении встречается хотя бы один хэштег из {\it negative} множества, то
            сообщение считается отрицательным.
    \end{enumerate}

    Для построения лексиконов, авторы использовали метод {\it <<тональности словосочетаний>>}
    (см. п. \ref{sec:soEvaluation}, формула \ref{eq:so}) \cite{lexiconSO}.

    В качестве классификатора, авторы использовали {\it SVM} с линейным ядром, и
    параметром штрафной функции $C = 5\cdot 10^{-3}$.
    Векторизация сообщения включала в себя набор дополнительных признаков:
    % Используемые признаки
    \begin{itemize}
        \item {\bf Учет регистра:} количество слов, записанных в верхнем регистре;
        \item {\bf Учет хэштегов:} число входящих в сообщение хэштегов (слов с префиксом <<\#>>);
        \item {\bf Символьные $n$-граммы:} присутствие или отсутствие последовательности
            подряд идущих одинаковых символов длиной в 3, 4, и 5 символов;
        \item {\bf На основе лексиконов:}
            Множество всех лексиконов включает в себя три лексикона, созданных
            в ручную ({\it NRC Emotion Lexicon, MPQA, Bing Liu Lexicon)}, а также
            два автоматически сгенерированных ({\it Hashtag Sentiment Lexicon,
            Sentiment140}). В качестве термов выступали биграммы, униграммы,
            Пусть $w$ -- рассматриваемый токен, $p$ -- полярность. Тогда,
            авторами были составлены следующие признаки:% функции
            \begin{itemize}
                \item Число токенов, для которых выполнено: $score(w, p) > 0$;
                \item Суммарное значение $\sum_{w \in tweet} score(w,p)$;
                \item Вычисление максимума $\max_{w \in tweet} score(w,p)$;
                \item Учет оценки последнего токена, при условии: $score(w,p) > 0$.
            \end{itemize}
        \item {\bf Пунктуация:}
            \begin{itemize}
                \item Подсчет числа последовательностей символов <<!>> и <<?>>,
                    а также подсчет случаев когда они оба символа встречаются в одной последовательности;
                \item Учет содержания знаков <<?>> или <<!>> в последнем терме.
            \end{itemize}
        \item {\bf Эмотиконы:}
            \begin{itemize}
                \item Присутствие или отсутствие положительных и негативных
                    эмотиконов в любой позиции сообщения;
                \item Признак, указывающий на наличие эмотикона (положительного
                    или негативного) в конце сообщения.
            \end{itemize}
        \item {\bf Удлиненные слова:} число слов, в которых символ повторяется
            более двух раз, например <<sooooo>>;
        \item {\bf Суффикс отрицания:} добавляет к слову суффикс {\it <<\_NEG>>},
            в случае, если перед ним имеется конструкция: {\it отрицание +
            знак пунктуации}.
            Под отрицанием понимаются слова вида: {\it<<no>>}, {\it <<shouldn't>>}.
            В качестве знаков пунктуации рассматриваются символы:
            <<,>>, <<.>>, <<:>>, <<;>>, <<!>>, <<?>>.
            Пример преобразования:
            \begin{center}
                \it
                \underline{shouldn't,} perfect \\
                perfect\_NEG
            \end{center}
    \end{itemize}

    % Результаты
    Среди всех команд, принимавших участие в соревновании {\it SemEval-2013 'Detecting
    Sentiment in Twitter'}, описанный подход занял первое место как в задаче
    определения тональности отдельного терма, так и сообщения в целом.
    По задаче оценки термов, авторам подхода удалось добиться оценки
    $F_{score}$ = $69.02\%$. На задаче тонального анализа на уровне сообщения,
    лучшая оценка несколько выше: $88.93\%$.


    % Улучшения
    \subsection{Изменение настроек SVM классификатора}
Рассмотрим динамику изменения результатов, в зависимости от величины
отступа для разделения классов SVM классификатором.
В пакете LibSVM отступ изменяется на основе параметра $C$.
% Описать, чего мы хотим добиться
%Так, на основе резульататов таблиц \ref{table:results2015} и \ref{table:results2016}
%наблюдается стабильный рост качества при добавлении новых лексиконов и признаков
%на их основе, то интересно проверить, насколько будут меняться результаты и
%будут ли улучшения при меньших значениях отступа.

На рис. \ref{fig:cost} рассмотрены результаты прогонов с
изменением значения параметра, овечающего за величину отступа в диапазоне
$[0.1, 1]$ с шагом 0.1. Ранее, результаты таблицы \ref{table:results2016}
были получены при максимальном значении параметра в рассматриваемом диапазоне,
т.е. при $C= 1$.
Лучшие результаты по каждому были вынесены в таблицу \ref{table:cost}.

\begin{figure}[!htop] \centering
    \begin{subfigure}[b]{0.35\textwidth}
        % пересчитать
        \includegraphics[width=\textwidth]{pics/2016_ttk_imbalanced.png}
        \caption{$TCC_{16}, \hspace{0.2cm} I_{tcc}^{16}$}
        \label{fig:cost_ttk_2016_imb}
    \end{subfigure}
    ~
    \begin{subfigure}[b]{0.35\textwidth}
        \includegraphics[width=\textwidth]{pics/2016_ttk_balanced.png}
        \caption{$TCC_{16}, \hspace{0.2cm} B_{tcc}^{16}$}
        \label{fig:cost_ttk_2016_b}
    \end{subfigure}

    \begin{subfigure}[b]{0.35\textwidth}
        \includegraphics[width=\textwidth]{pics/2016_bank_imbalanced.png}
        \caption{$BANK_{16},\hspace{0.2cm} I_{bank}^{16}$}
        \label{fig:cost_bank_2016_imb}
    \end{subfigure}
    ~
    \begin{subfigure}[b]{0.35\textwidth}
        \includegraphics[width=\textwidth]{pics/2016_bank_balanced.png}
        \caption{$BANK_{16},\hspace{0.2cm} B_{bank}^{16}$}
        \label{fig:cost_bank_2016_b}
    \end{subfigure}

    \caption{
        Влияние {\it параметра штрафной функции SVM классификатора (C)}
        на результаты прогонов для задачи TCC и BANK на разных обучающих
        коллекциях;
        кривыми на графиках обозначены прогоны, номера которых соответствуют
        настройкам таблицы \ref{table:settings};
        значение параметра измерялось в пределах $[0.1, 1]$ с шагом $0.1$.
    }
    \label{fig:cost}
\end{figure}


\begin{table}[htp!]
\centering
\caption{Наилучшие результаты $F_{1-macro}^{PN}$ рис. \ref{fig:cost}
    для каждой задачи по каждому из прогонов.
    Помимо результата фиксируется значение параметра $Cost$, при котором
    достигается такой резульат.
}
\label{table:cost}
\begin{tabular}{ccccc}
\hline
\multicolumn{1}{c|}{\multirow{2}{*}{№}} & \multicolumn{2}{c|}{BANK}                                                   & \multicolumn{2}{c}{TCC}                                           \\ \cline{2-5}
\multicolumn{1}{c|}{}                   & \multicolumn{1}{c|}{$I_{bank_{/cost}}^{16}$} & \multicolumn{1}{c|}{$B_{bank_{/cost}}^{16}$} & \multicolumn{1}{c|}{$I_{tcc_{/cost}}^{16}$}  & $B_{tcc_{/cost}}^{16}$             \\ \hline
1                                       & $48.93_{0.6}$                                &  $46.71_{0.3}$                               &  $48.57_{0.7}$                                        & $53.50_{0.2}$                                  \\
2                                       & $50.69_{0.6}$                                &  $47.33_{0.5}$                               &  $49.48_{1.0}$                                       & $53.31_{0.3}$                                  \\
3                                       & $51.25_{0.8}$                                &  $48.00_{0.6}$                               &  $49.56_{0.5}$                                          & $53.46_{0.4}$                                  \\
4                                       & $51.52_{0.8}$                                &  $48.43_{0.5}$                               &  $49.84_{0.8}$                                         & $54.25_{0.4}$                                  \\
5                                       & $51.33_{0.7}$                                &  $48.39_{0.5}$                               &  $50.55_{1.0}$                                        & $54.26_{0.5}$                                  \\
6                                       & $\textbf{52.82}_{0.7}$                       &  $51.50_{0.7}$                               &  $52.02_{0.6}$                                        & $\textbf{55.46}_{0.3}$                         \\ \hline
\end{tabular}
\end{table}



Сравнивая результаты относительно использования разных типов обучающих коллекций,
для несбалансированных коллекций (рис. \ref{fig:cost_ttk_2016_imb},
\label{fig:cost_bank_2016_imb})
можно наблюдать резкий спад при использовании малого значения параметра ($C < 0.4$).
Что касается всех результатов,то положительный эффект от добавление признаков
на основе лексиконов сохраняется.
В большинстве случаев лучший результат достигается при подходе №6, что
можно было наблюдать и в таблицах \ref{table:results2015} и \ref{table:results2016}.
Наибольший отрыв такого подохода отмечается на рис. \ref{fig:cost_bank_2016_b}.
Подведя итог, посмотрев на работу классификационных модлелей с точки зрения
изменения настроек, можно рекомендовать использование лексиконов для составления
дополнительных признаков.



    % Список литературы
    \clearpage
    \newpage
    \bibliographystyle{styles/utf8gost705u}  %% стилевой файл для оформления по ГОСТу
    \addcontentsline{toc}{section}{\large Список Литературы}
    \bibliography{biblio}     %% имя библиографической базы (bib-файла)
\end{document}
