\section{Тестирование на коллекции SentiRuEval-2015}
\label{sec:sentirueval2015}

Качество работы подготовленных моделей оценивается на основе $F_1$ меры:
\begin{equation}
    \label{eq:fmeasure}
    F_1 = 2 \cdot \dfrac{P \cdot R}{P + R}
\end{equation}

Такая мера позволяет одновременно учитывать результаты следующих параметров
относительно некоторого класса:
\begin{itemize}
    \item Точность ({\bf P}recision) -- количество сообщений, которое
        классификатор правильно отнес к соответствующему классу по отношению ко
        всему объему сообщений определенных системой в этот класс;
    \item Полнота ({\bf R}ecall) -- число найденных сообщений, которые
        действительно принадлежат соответствующему классу относительно всех
        сообщений соответствующего класса.
\end{itemize}

% Про макро/микро -усреднения при переходе к нескольким классам.
В случае, когда необходимо оценить качество работы по метрике $F_1$ относительно
несколькик классов, то применяются усреднения меры.
Различают {\it микро-} и {\it макро-} усреднения.
Для вычисления усредненной $F_{1}$ меры определяются
параметры полноты и точности с соответствующим усреднением относительно
интересующих нас классов, которые, в свою очередь, вычисляются на основе
таблиц контингентности.

%
% Если что, то можно вставить
%

Мaкроусреднение придает одинаковый вес каждому из усредняемых классов, в то
время как при микроусреднении вес учитывается на основе числа документов в
классе.
При использовании $F_{1}$-макро смещение среднего значения будет производиться в
сторону того класса, для которого классификатор сработал лучше; в тоже время,
при использовании $F_{1}$-микро, смещение будет произведено в сторону наибольшего
класса. \cite{micromacromeasures}

В данной задаче нас интересует качество определения тональных твитов, т.е.
сообщений соответсвующих положительному ({\it Positive}) и отрицательному
({\it Negative}) классам.
Будем рассматривать результаты с макроусреднением $F_1$ меры
(формула \ref{eq:fmacro12}).

\begin{equation}
    \label{eq:fmacro12}
    F_{1-macro}^{PN} = 2 \cdot \dfrac{P_{macro}^{PN} \cdot
        R_{macro}^{PN}}{P_{macro}^{PN} + R_{macro}^{PN}}
\end{equation}

В тестировании участвуют прогоны с настройками, представленными в таблице
\ref{table:settings}.
Все прогоны будут одновременно протестированы в двух областях:
\begin{itemize}
    \item BANK -- отзывы о банках;
    \item TCC -- отзывы о телекоммуникационных компаниях.
\end{itemize}

\begin{table}[ht!]
\centering
\caption{Настройки векторизации сообщений}
\label{table:settings}
\begin{tabular}{cccccccccccc}
\hline
\multicolumn{1}{c|}{\multirow{2}{*}{№}} & \multicolumn{1}{c|}{\multirow{2}{*}{Термы}} & \multicolumn{1}{c|}{\multirow{2}{*}{\begin{tabular}[c]{@{}c@{}}Доп.\\ признаки\end{tabular}}} & \multicolumn{3}{c|}{$l_{1}$}                                                                                      & \multicolumn{3}{c|}{$l_{2}$}                                                                                     & \multicolumn{3}{c}{$l_{3}$}                                                                 \\ \cline{4-12}
\multicolumn{1}{c|}{}                   & \multicolumn{1}{c|}{}                       & \multicolumn{1}{c|}{}                                                                        & \multicolumn{1}{c|}{$\sum$} & \multicolumn{1}{c|}{$\max\limits_{1..N}$} & \multicolumn{1}{c|}{$\min\limits_{1..N}$} & \multicolumn{1}{c|}{$\sum$} & \multicolumn{1}{c|}{$\max\limits_{1..N}$} & \multicolumn{1}{c|}{$\min\limits_{1..N}$} & \multicolumn{1}{c|}{$\sum$} & \multicolumn{1}{c|}{$\max\limits_{1..N}$} & $\min\limits_{1..N}$ \\ \hline
1                                       & $\bullet$                                   &                                                                                              &                             &                                          &                                          &                            &                                          &                                          &                            &                                          &                     \\
2                                       & $\bullet$                                   & $\bullet$                                                                                    &                             &                                          &                                          &                            &                                          &                                          &                            &                                          &                     \\
3                                       & $\bullet$                                   & $\bullet$                                                                                    & $\bullet$                   &                                          &                                          &                            &                                          &                                          &                            &                                          &                     \\
4                                       & $\bullet$                                   & $\bullet$                                                                                    & $\bullet$                   &                                          &                                          & $\bullet$                  &                                          &                                          &                            &                                          &                     \\
5                                       & $\bullet$                                   & $\bullet$                                                                                    & $\bullet$                   &                                          &                                          & $\bullet$                  &                                          &                                          & $\bullet$                  &                                          &                     \\
6                                       & $\bullet$                                   & $\bullet$                                                                                    & $\bullet$                   & $\bullet$                                & $\bullet$                                & $\bullet$                  & $\bullet$                                & $\bullet$                                & $\bullet$                  & $\bullet$                                & $\bullet$           \\ \hline
\end{tabular}
\end{table}


Рассмотрим качество работы классификационных моделей каждого из прогонов для
коллекций $BANK_{15}$ и $TCC_{15}$ в зависимости от типа обучающей коллекции
($I$ -- несбалансированной, $B$ -- сбалансированной).
Результаты проведенного тестирования зафиксированы в таблице \ref{table:results2015}.

\begin{table}[ht!]
\centering
\caption{Результаты $F_{1-macro}^{PN}$ тестирования на коллекции {\it SentiRuEval-2016};
        {\bf жирным шрифтом} отмечается лучший результат по каждой задаче.
        }
\label{table:results2016}
\begin{tabular}{ccccc}
\hline
\multicolumn{1}{c|}{\multirow{2}{*}{№}} & \multicolumn{2}{c|}{BANK} & \multicolumn{2}{c}{TCC}\\ \cline{2-5}
\multicolumn{1}{c|}{}                   & \multicolumn{1}{c|}{$I$} & \multicolumn{1}{c|}{$B$} & \multicolumn{1}{c|}{$I$} & $B$             \\ \hline
1                                       & ${48.77}$                              & $45.53$                              & $48.32$                             & $50.90$                    \\
2                                       & ${50.24}$                              & $47.36$                              & $49.48$                             & $50.69$                    \\
3                                       & ${50.28}$                              & $47.53$                              & $48.90$                             & $51.95$                    \\
4                                       & ${50.45}$                              & $47.13$                              & $49.31$                             & $52.72$                    \\
5                                       & ${49.72}$                              & $46.99$                              & $50.55$                             & $\bf{52.90}$               \\
6                                       & ${\bf 51.73}$                          & $50.25$                              & $51.66$                             & $52.63$                    \\ \hline
\end{tabular}
\end{table}
\cellcolor[HTML]{C0C0C0}
\cellcolor[HTML]{C0C0C0}
\cellcolor[HTML]{C0C0C0}
\cellcolor[HTML]{C0C0C0}
\cellcolor[HTML]{C0C0C0}
\cellcolor[HTML]{C0C0C0}

% Вывод о преимуществе применения балансировки.

Независимо от типа тестируемой коллекции, исходя из полученных результатов,
можно сказать что добавление признаков на основе лексиконов стабильно
повышает качество классификации.
Так, начиная с прогона №3 наблюдается прирост качества (см. результаты
прогонов с №3 -- №6, таблица \ref{table:results2015}.
Таким образом, заявленный прирост качества в статье близкого подхода
\cite{modernApproach} можно наблюдать и в тональной классификации русскоязычного
Твиттера.

Наибольший результат достигается в прогоне №6, где по каждому из лексиконов таблицы
\ref{table:createdLexicons} вычисляются все признаки: сумма, минимум и максимум.
Добавление последних двух признаков в векторизацию сообщений изменило
результат качества на +2.25 для $BANK_{15}$, и на +1.02 для $TKK_{15}$, если
сравнивать с подходом №5.

Если проанализировать результат с точки зрения влияния балансировки, то
здесь она приходится кстати, если говорить о коллекции $BANK_{15}$.
Средний прирост по каждому прогону, при сравнении результатов $I_{bank}^{15}$
с $B_{bank}^{15}$ составил +3.02.

Твиты коллекции TCC классифицируются несколько лучше. Эта особенность
отмечается в \cite{tonalityAnalysis} что объясняется ухудшением ситуации на
Украине в период сбора сообщений для тестовой коллекции $BANK_{15}$.
Так, например слово <<санкции>> может нести негативный характер в тестовой
выборке, в то время как в обучающей колекции аналогичное слово является
нейтральным.

\subsection{Сравнение с результатами участников}

Если сравнить лучший полученный в статье результат с участниками соревнований\footnote{
    Результаты участников соревнований SentiRuEval-2015:
    \url{https://docs.google.com/spreadsheets/d/1IxGFhGO4zS5t356FePIMdJQ5IU6-tkpetoOzoXpGLPs/edit\#gid=0}
},
(см. таблицу \ref{table:participants_results}),
то наилучший результат среди участников по метрике $F_{1-macro}^{PN}$ на коллекции $BANK_{15}$
составляет 35.98 (участник №4), а для коллекции $TCC_{15}$ -- 48.04 (участник №3).
\begin{table}[htp!]
\centering
\caption{Сравнение описанного в статье подхода с результатами остальных
    участников соревнований {\it SentiRuEval-2016}; {\bf жирным шрифтом}
    отмечен участник с наилучшими результатами}
\label{table:comparison}
\begin{tabular}{ccc}
\hline
Участник   & BANK           & TCC             \\ \hline
1          & 46.83          & 52.86           \\
\textbf{2} & \textbf{55.17} & \textbf{55.94}  \\
3          & 34.23          & 39.94           \\
4          & 37.30          & 49.55           \\
5          & 38.59          & 34.99           \\
6          & 23.98          & 35.45           \\
7          & 47.10          & 48.42           \\
8          & 44.92          & 48.71           \\
9          & 51.95          & 54.89           \\
10         & 46.59          & 50.55           \\ \hline
\end{tabular}
\end{table}


Таким образом, текущий подход мог бы показать первое место на прошедших
соревнованиях {\it SentiRuEval-2015}.
Единственный нюанс -- необходимо было бы использовать данные для составления
лексиконов, собранные до даты проведения соревнования.
Сравнивая лучший резульатат соревнования с прогоном №6
(см. таблицу \ref{table:results2015}), отрыв относительно победителя
по метрике $F_{1-macro}^{PN}$ составляет +6.23 для задачи BANK,
и +2.08 для задачи TCC.
