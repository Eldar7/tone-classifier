\section{Тестирование на коллекции SentiRuEval-2015}

В тестировании участвуют прогоны с настройками, представленными в таблице
\ref{table:settings}.

\begin{table}[ht!]
\centering
\caption{Настройки векторизации сообщений}
\label{table:settings}
\begin{tabular}{cccccccccccc}
\hline
\multicolumn{1}{c|}{\multirow{2}{*}{№}} & \multicolumn{1}{c|}{\multirow{2}{*}{Термы}} & \multicolumn{1}{c|}{\multirow{2}{*}{\begin{tabular}[c]{@{}c@{}}Доп.\\ признаки\end{tabular}}} & \multicolumn{3}{c|}{$l_{1}$}                                                                                      & \multicolumn{3}{c|}{$l_{2}$}                                                                                     & \multicolumn{3}{c}{$l_{3}$}                                                                 \\ \cline{4-12}
\multicolumn{1}{c|}{}                   & \multicolumn{1}{c|}{}                       & \multicolumn{1}{c|}{}                                                                        & \multicolumn{1}{c|}{$\sum$} & \multicolumn{1}{c|}{$\max\limits_{1..N}$} & \multicolumn{1}{c|}{$\min\limits_{1..N}$} & \multicolumn{1}{c|}{$\sum$} & \multicolumn{1}{c|}{$\max\limits_{1..N}$} & \multicolumn{1}{c|}{$\min\limits_{1..N}$} & \multicolumn{1}{c|}{$\sum$} & \multicolumn{1}{c|}{$\max\limits_{1..N}$} & $\min\limits_{1..N}$ \\ \hline
1                                       & $\bullet$                                   &                                                                                              &                             &                                          &                                          &                            &                                          &                                          &                            &                                          &                     \\
2                                       & $\bullet$                                   & $\bullet$                                                                                    &                             &                                          &                                          &                            &                                          &                                          &                            &                                          &                     \\
3                                       & $\bullet$                                   & $\bullet$                                                                                    & $\bullet$                   &                                          &                                          &                            &                                          &                                          &                            &                                          &                     \\
4                                       & $\bullet$                                   & $\bullet$                                                                                    & $\bullet$                   &                                          &                                          & $\bullet$                  &                                          &                                          &                            &                                          &                     \\
5                                       & $\bullet$                                   & $\bullet$                                                                                    & $\bullet$                   &                                          &                                          & $\bullet$                  &                                          &                                          & $\bullet$                  &                                          &                     \\
6                                       & $\bullet$                                   & $\bullet$                                                                                    & $\bullet$                   & $\bullet$                                & $\bullet$                                & $\bullet$                  & $\bullet$                                & $\bullet$                                & $\bullet$                  & $\bullet$                                & $\bullet$           \\ \hline
\end{tabular}
\end{table}


Рассмотрим изменение результатов в каждом из прогонов для каждой задачи
(BANK, TKK), в зависимости от типа используемой обучающей коллекции.
Оценка качества работы классификаторов производится по метрике
$F_{1(macro)}^{PN}$.
Результаты приведены в таблице
\ref{table:results2015}.

\begin{table}[ht!]
\centering
\caption{Результаты $F_{1-macro}^{PN}$ тестирования на коллекции {\it SentiRuEval-2016};
        {\bf жирным шрифтом} отмечается лучший результат по каждой задаче.
        }
\label{table:results2016}
\begin{tabular}{ccccc}
\hline
\multicolumn{1}{c|}{\multirow{2}{*}{№}} & \multicolumn{2}{c|}{BANK} & \multicolumn{2}{c}{TCC}\\ \cline{2-5}
\multicolumn{1}{c|}{}                   & \multicolumn{1}{c|}{$I$} & \multicolumn{1}{c|}{$B$} & \multicolumn{1}{c|}{$I$} & $B$             \\ \hline
1                                       & ${48.77}$                              & $45.53$                              & $48.32$                             & $50.90$                    \\
2                                       & ${50.24}$                              & $47.36$                              & $49.48$                             & $50.69$                    \\
3                                       & ${50.28}$                              & $47.53$                              & $48.90$                             & $51.95$                    \\
4                                       & ${50.45}$                              & $47.13$                              & $49.31$                             & $52.72$                    \\
5                                       & ${49.72}$                              & $46.99$                              & $50.55$                             & $\bf{52.90}$               \\
6                                       & ${\bf 51.73}$                          & $50.25$                              & $51.66$                             & $52.63$                    \\ \hline
\end{tabular}
\end{table}
\cellcolor[HTML]{C0C0C0}
\cellcolor[HTML]{C0C0C0}
\cellcolor[HTML]{C0C0C0}
\cellcolor[HTML]{C0C0C0}
\cellcolor[HTML]{C0C0C0}
\cellcolor[HTML]{C0C0C0}

% Вывод о преимуществе применения балансировки.

Весьма неоднозначная картина поведения классификатора наблюдается в зависимости
от рассматриваемой задачи.
Интересно отметить, что сбалансированная коллекция улучшает результат для
задачи BANK, в то время как для классификации коллекции задачи TCC, наоборот,
лучше подходит использование несбалансированной коллекции.

В остальном, использование признаков (в том числе и на основе лексиконов)
стабильно улучшает качество работы классификатора.
Наибольший эффект улучшения наблюдается при вычислении
{\it минимума } и {\it максимума} (прогон №6).
% Благодаря введению таких признаков удалось добиться повышения качетва в среднем
% на \

Твиты коллекции TCC классифицируются несколько лучше, и аналогичная особенность
отмечается в \cite{tonalityAnalysis} что объясняется ухудшением ситуации на
Украине в период сбора сообщений для тестовой коллекции задачи BANK.
Так, например слово <<санкции>> может нести негативный характер в тестовой
выборке, в то время как в обучающей колекции аналогичное слово является
нейтральным.

\subsection{Изменение настроек SVM классификатора}
Рассмотрим динамику изменения результатов, в зависимости от величины
отступа для разделения классов SVM классификатором.
В пакете LibSVM отступ изменяется на основе параметра $C$.
% Описать, чего мы хотим добиться
%Так, на основе резульататов таблиц \ref{table:results2015} и \ref{table:results2016}
%наблюдается стабильный рост качества при добавлении новых лексиконов и признаков
%на их основе, то интересно проверить, насколько будут меняться результаты и
%будут ли улучшения при меньших значениях отступа.

На рис. \ref{fig:cost} рассмотрены результаты прогонов с
изменением значения параметра, овечающего за величину отступа в диапазоне
$[0.1, 1]$ с шагом 0.1. Ранее, результаты таблицы \ref{table:results2016}
были получены при максимальном значении параметра в рассматриваемом диапазоне,
т.е. при $C= 1$.
Лучшие результаты по каждому были вынесены в таблицу \ref{table:cost}.

\begin{figure}[!htop] \centering
    \begin{subfigure}[b]{0.35\textwidth}
        % пересчитать
        \includegraphics[width=\textwidth]{pics/2016_ttk_imbalanced.png}
        \caption{$TCC_{16}, \hspace{0.2cm} I_{tcc}^{16}$}
        \label{fig:cost_ttk_2016_imb}
    \end{subfigure}
    ~
    \begin{subfigure}[b]{0.35\textwidth}
        \includegraphics[width=\textwidth]{pics/2016_ttk_balanced.png}
        \caption{$TCC_{16}, \hspace{0.2cm} B_{tcc}^{16}$}
        \label{fig:cost_ttk_2016_b}
    \end{subfigure}

    \begin{subfigure}[b]{0.35\textwidth}
        \includegraphics[width=\textwidth]{pics/2016_bank_imbalanced.png}
        \caption{$BANK_{16},\hspace{0.2cm} I_{bank}^{16}$}
        \label{fig:cost_bank_2016_imb}
    \end{subfigure}
    ~
    \begin{subfigure}[b]{0.35\textwidth}
        \includegraphics[width=\textwidth]{pics/2016_bank_balanced.png}
        \caption{$BANK_{16},\hspace{0.2cm} B_{bank}^{16}$}
        \label{fig:cost_bank_2016_b}
    \end{subfigure}

    \caption{
        Влияние {\it параметра штрафной функции SVM классификатора (C)}
        на результаты прогонов для задачи TCC и BANK на разных обучающих
        коллекциях;
        кривыми на графиках обозначены прогоны, номера которых соответствуют
        настройкам таблицы \ref{table:settings};
        значение параметра измерялось в пределах $[0.1, 1]$ с шагом $0.1$.
    }
    \label{fig:cost}
\end{figure}


\begin{table}[htp!]
\centering
\caption{Наилучшие результаты $F_{1-macro}^{PN}$ рис. \ref{fig:cost}
    для каждой задачи по каждому из прогонов.
    Помимо результата фиксируется значение параметра $Cost$, при котором
    достигается такой резульат.
}
\label{table:cost}
\begin{tabular}{ccccc}
\hline
\multicolumn{1}{c|}{\multirow{2}{*}{№}} & \multicolumn{2}{c|}{BANK}                                                   & \multicolumn{2}{c}{TCC}                                           \\ \cline{2-5}
\multicolumn{1}{c|}{}                   & \multicolumn{1}{c|}{$I_{bank_{/cost}}^{16}$} & \multicolumn{1}{c|}{$B_{bank_{/cost}}^{16}$} & \multicolumn{1}{c|}{$I_{tcc_{/cost}}^{16}$}  & $B_{tcc_{/cost}}^{16}$             \\ \hline
1                                       & $48.93_{0.6}$                                &  $46.71_{0.3}$                               &  $48.57_{0.7}$                                        & $53.50_{0.2}$                                  \\
2                                       & $50.69_{0.6}$                                &  $47.33_{0.5}$                               &  $49.48_{1.0}$                                       & $53.31_{0.3}$                                  \\
3                                       & $51.25_{0.8}$                                &  $48.00_{0.6}$                               &  $49.56_{0.5}$                                          & $53.46_{0.4}$                                  \\
4                                       & $51.52_{0.8}$                                &  $48.43_{0.5}$                               &  $49.84_{0.8}$                                         & $54.25_{0.4}$                                  \\
5                                       & $51.33_{0.7}$                                &  $48.39_{0.5}$                               &  $50.55_{1.0}$                                        & $54.26_{0.5}$                                  \\
6                                       & $\textbf{52.82}_{0.7}$                       &  $51.50_{0.7}$                               &  $52.02_{0.6}$                                        & $\textbf{55.46}_{0.3}$                         \\ \hline
\end{tabular}
\end{table}



Сравнивая результаты относительно использования разных типов обучающих коллекций,
для несбалансированных коллекций (рис. \ref{fig:cost_ttk_2016_imb},
\label{fig:cost_bank_2016_imb})
можно наблюдать резкий спад при использовании малого значения параметра ($C < 0.4$).
Что касается всех результатов,то положительный эффект от добавление признаков
на основе лексиконов сохраняется.
В большинстве случаев лучший результат достигается при подходе №6, что
можно было наблюдать и в таблицах \ref{table:results2015} и \ref{table:results2016}.
Наибольший отрыв такого подохода отмечается на рис. \ref{fig:cost_bank_2016_b}.
Подведя итог, посмотрев на работу классификационных модлелей с точки зрения
изменения настроек, можно рекомендовать использование лексиконов для составления
дополнительных признаков.


