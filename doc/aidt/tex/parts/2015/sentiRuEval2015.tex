\section{Тестирование на коллекции SentiRuEval-2015}

В тестировании участвуют прогоны с настройками, представленными в таблице
\ref{table:settings}.

\begin{table}[ht!]
\centering
\caption{Настройки векторизации сообщений}
\label{table:settings}
\begin{tabular}{cccccccccccc}
\hline
\multicolumn{1}{c|}{\multirow{2}{*}{№}} & \multicolumn{1}{c|}{\multirow{2}{*}{Термы}} & \multicolumn{1}{c|}{\multirow{2}{*}{\begin{tabular}[c]{@{}c@{}}Доп.\\ признаки\end{tabular}}} & \multicolumn{3}{c|}{$l_{1}$}                                                                                      & \multicolumn{3}{c|}{$l_{2}$}                                                                                     & \multicolumn{3}{c}{$l_{3}$}                                                                 \\ \cline{4-12}
\multicolumn{1}{c|}{}                   & \multicolumn{1}{c|}{}                       & \multicolumn{1}{c|}{}                                                                        & \multicolumn{1}{c|}{$\sum$} & \multicolumn{1}{c|}{$\max\limits_{1..N}$} & \multicolumn{1}{c|}{$\min\limits_{1..N}$} & \multicolumn{1}{c|}{$\sum$} & \multicolumn{1}{c|}{$\max\limits_{1..N}$} & \multicolumn{1}{c|}{$\min\limits_{1..N}$} & \multicolumn{1}{c|}{$\sum$} & \multicolumn{1}{c|}{$\max\limits_{1..N}$} & $\min\limits_{1..N}$ \\ \hline
1                                       & $\bullet$                                   &                                                                                              &                             &                                          &                                          &                            &                                          &                                          &                            &                                          &                     \\
2                                       & $\bullet$                                   & $\bullet$                                                                                    &                             &                                          &                                          &                            &                                          &                                          &                            &                                          &                     \\
3                                       & $\bullet$                                   & $\bullet$                                                                                    & $\bullet$                   &                                          &                                          &                            &                                          &                                          &                            &                                          &                     \\
4                                       & $\bullet$                                   & $\bullet$                                                                                    & $\bullet$                   &                                          &                                          & $\bullet$                  &                                          &                                          &                            &                                          &                     \\
5                                       & $\bullet$                                   & $\bullet$                                                                                    & $\bullet$                   &                                          &                                          & $\bullet$                  &                                          &                                          & $\bullet$                  &                                          &                     \\
6                                       & $\bullet$                                   & $\bullet$                                                                                    & $\bullet$                   & $\bullet$                                & $\bullet$                                & $\bullet$                  & $\bullet$                                & $\bullet$                                & $\bullet$                  & $\bullet$                                & $\bullet$           \\ \hline
\end{tabular}
\end{table}


Рассмотрим изменение результатов в каждом из прогонов для каждой задачи
(BANK, TKK), в зависимости от типа используемой обучающей коллекции.
Оценка качества работы классификаторов производится по метрике
$F_{1(macro)}^{PN}$.
Результаты приведены в таблице
\ref{table:results2015}.

\begin{table}[ht!]
\centering
\caption{Результаты $F_{1-macro}^{PN}$ тестирования на коллекции {\it SentiRuEval-2016};
        {\bf жирным шрифтом} отмечается лучший результат по каждой задаче.
        }
\label{table:results2016}
\begin{tabular}{ccccc}
\hline
\multicolumn{1}{c|}{\multirow{2}{*}{№}} & \multicolumn{2}{c|}{BANK} & \multicolumn{2}{c}{TCC}\\ \cline{2-5}
\multicolumn{1}{c|}{}                   & \multicolumn{1}{c|}{$I$} & \multicolumn{1}{c|}{$B$} & \multicolumn{1}{c|}{$I$} & $B$             \\ \hline
1                                       & ${48.77}$                              & $45.53$                              & $48.32$                             & $50.90$                    \\
2                                       & ${50.24}$                              & $47.36$                              & $49.48$                             & $50.69$                    \\
3                                       & ${50.28}$                              & $47.53$                              & $48.90$                             & $51.95$                    \\
4                                       & ${50.45}$                              & $47.13$                              & $49.31$                             & $52.72$                    \\
5                                       & ${49.72}$                              & $46.99$                              & $50.55$                             & $\bf{52.90}$               \\
6                                       & ${\bf 51.73}$                          & $50.25$                              & $51.66$                             & $52.63$                    \\ \hline
\end{tabular}
\end{table}
\cellcolor[HTML]{C0C0C0}
\cellcolor[HTML]{C0C0C0}
\cellcolor[HTML]{C0C0C0}
\cellcolor[HTML]{C0C0C0}
\cellcolor[HTML]{C0C0C0}
\cellcolor[HTML]{C0C0C0}

% Вывод о преимуществе применения балансировки.

Весьма неоднозначная картина поведения классификатора наблюдается в зависимости
от рассматриваемой задачи.
Интересно отметить, что сбалансированная коллекция улучшает результат для
задачи BANK (средний прирост $+3.02$), в то время как для классификации
коллекции задачи TCC, наоборот, лучше подходит использование несбалансированной
коллекции.

В остальном, добавление признаков (в том числе и на основе лексиконов)
стабильно улучшает качество работы классификатора.
Наибольший эффект улучшения наблюдается при вычислении
{\it минимума } и {\it максимума} (прогон №6).
% Благодаря введению таких признаков удалось добиться повышения качетва в среднем
% на \

Твиты коллекции TCC классифицируются несколько лучше, и аналогичная особенность
отмечается в \cite{tonalityAnalysis} что объясняется ухудшением ситуации на
Украине в период сбора сообщений для тестовой коллекции задачи BANK.
Так, например слово <<санкции>> может нести негативный характер в тестовой
выборке, в то время как в обучающей колекции аналогичное слово является
нейтральным.
