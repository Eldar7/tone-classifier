\section{Обзор близких подходов}
    \label{sec:review}
    % \cite{severyn} (Automatic Learning)
    %   -- упоминается про использование SVM c биграммами для достижения
    %   хороших результатов. (Wand and Manning, 2012)
    %   Третий абзац -- пояснение про лексиконы (что это отображение термов в
    %   тональные оценки)
    %   -- описывает работу (Mohammad, 2013) про использование признаков на
    %   основе лексиконов для улучшения качества работы
    %   -- используют для составления лексиконов \cite{lexiconSO}
    %   -- Последний абзац -- что используется для оценки подхода.

    % \cite{modernApproach} (Building The State Of The Art)
    %   -- прирост в 5 единиц при использовании лексиконов в качестве признаков !! (Abstract)
    %   -- объясняет области применения подхода (для оценки терма и сообщения в целом).
    %   -- используют для составления лексиконов \cite{lexiconSO}
    %   -- объясняет про задачу, для которой будет применен построенных
    %   классификатор

    % \cite{svmCompareVsNB}

    % \cite{dialog2015}
    %   -- посмотреть, какие наиболее частые подходы векторизации,
    %   чтобы сделать вывод о том, что нужно использовать

    % \cite{tonalityAnalysis}
    %   -- обзорная статья про SentiRuEval

    % \cite{svmAdvantages}
    %   -- сравнение SVM, NB, ME для решения задачи тональной классификации
    %   в сети Twitter.

    % В общем нужно посмотреть на эту задачу со стороны эффективности
    % использования тех или иных корпусов даных для обучения, и посмотреть
    % как будут изменяться результаты в зависимости от используемого корпуса.

    % Привести аналогию с PerLab

    % 1. Написать про зарубежные соревнования (SemEval)
    На протяжении нескольких лет в рамках зарубежной концеренции {\it SemEval},
    проводятся соревнования по тональной классификации сообщений в сети {\it Twitter}.
    Начиная с {\it SemEval-2013}, в контесте рассматривалось два типа задач: $A$
    -- классификация на уровне фраз, и $B$ -- на уровне всего сообщения.
    Наилучший результат по метрике $F_{1(macro)}$ для задачи $B$ составил
    $69\%$ \cite{semEval2013}.
    Соревнования были продолжены в 2014 году \cite{semEval2014}, на которых
    лучший подход прошлогоднего участника продолжил удерживать высокие позиции в
    общем рейтинге по задаче $B$, и показал результат в $69.84\%$
    \cite{modernApproach}.

    % 2. Описать про смежные подходы зарубежных соревовнования (SemEval)
    Авторы подхода с наилучшим результатом {\it SemEval-2013} \cite{tonalityAnalysis}
    построили дополнительный корпус на основе данных сети {\it Twitter} и
    авторазметки.
    Сообщения в Твиттере содержат большое количество метаинформации (\#хэштеги,
    эмотиконы), пользуясь которой можно принять решение о тональности.
    Полученная таким образом размеченная коллекция используется в целях
    обучения классификатора, а также для составления {\it лексикона}.

    Лексиконы представляют собой словарь пар $\left<w, s\right>$, в котором для каждого
    входящего в него словосочетания $w$ определена тональная окраска $s \in R$.
    Это позволяет отобразить фразы сообщения в численную эмоциональную оценку и,
    как следствие, получить приближенную тональную окраску сообщения.
    Как построенный лексикон (на основе $PMI$ \cite{lexiconSO}), так и лексиконы
    на основе других корпусов\footnote{
        Лексиконы на основе корпусов {\it MPQA, Bing Liu, NRC-Emoticon, Sentiment-140},
        использовались для добавления признаков в работах
        \cite{tonalityAnalysis,modernApproach}.
    }
    использовались авторами \cite{tonalityAnalysis} для составления признаков.

    % SentiRuEval-2014
    Схожий подход, примененный в 2014 году демонстрирует прирост качетства при
    использовании нескольких различных признаков (суммы, максимума) на основе
    лексиконов \cite{modernApproach}.
    Авторы также приводят ряд признаков, которые позволяют отделить тональные
    сообщения от нейтральных\footnote{
        Учет количества слов в верхнем регистре;
        учет слов с большим числом повторения букв;
        признаки на основе пунктуации;
        и т.д.
    }, что положительно сказывается на работе классификатора.
    Результатом таких исследований стало 2-e место на соревнованиях {\it SemEval-2014}.

    % 3. Описать про соревнования в России
    Последние годы в России проводятся соревнования {\it SentiRuEval} в области
    тональной классификации сообщений сети {\it Twitter}~\cite{dialog2015,dialog2016}.
    % 4. Описать про предлагаемый подход.
    В этой статье рассмотрено построение классификатора для русскоязычной сети
    {\it Twitter} с использованием идей подходов \cite{tonalityAnalysis, modernApproach}.
    %Как и в случае с {\it SemEval}, организаторы предоставляют коллекции для обучения
    %классификаторов.
    Ставится задача применения таких идей для расширения обучающих коллекций,
    а также использование признаков на основе созданных лексиконов.
    Оценка подхода будет произведена на тестовых коллекциях
    {\it SentiRuEval-2015/2016}.
