\subsection{Построение лексиконов}

Построение лексикона производится на основе {\it меры взаимной информации}
\cite{lexiconSO}:
\begin{gather}
    PMI(t_1, t_2) = log_2 \dfrac{P(t_1\cap t_2)}{P(t_1)\cdot P(t_2)}
    \label{eq:pmi}
\end{gather}

Такая мера показывает связь $t_1$ с $t_2$, т.е. какова вероятность их связи.
В качестве второго аргумента, рассмотрим метку, которая будет соответствовать
одному из тональных классов:
\begin{itemize}
    \item {\it Excellent} -- положительный оттенком;
    \item {\it Poor} -- негативный оттенком.
\end{itemize}

Таким образом, относительно каждого маркера по формуле \ref{eq:pmi} можно
установить степень связи $t_1$ с положительным и негативным оттенком.
На основе разности значений можно определить тональность терма, или его
{\it семантическую ориентацию} (формула \ref{eq:so}).
\begin{equation}
    \label{eq:so}
    SO(t) = PMI(t, Excellent) - PMI(t, Poor)
\end{equation}

От исходной коллекции $K$, на основе которой будет создан лексикон, требуется
чтобы все сообщения коллекции были размечены по тональным классам.
Т.е., для каждого сообщения должна быть определена метка {\it Excellent} или
{\it Poor}, которая характеризует тональность сообщения в целом.
Такую метку можно проставить автоматически, и применительно к сообщениям
сети Twitter на основе {\it эмотиконов} или {\it хэштегов} \cite{severyn}.
В результате лексикон строится следующим образом:
\begin{equation}
    S : \{ \left< t, SO(t) \right> | t \in K\}
\end{equation}


Лексиконы были составлены на основе следующих данных (параметры представлены
в таблице \ref{table:createdLexicons}):

\begin{itemize}
    \item $l_1$ -- корпус коротких текстов на русском языке Ю.~Рубцовой\footnote{
        \url{https://github.com/nicolay-r/tone-classifier/tree/2016_jan_contest/data/lexicons}
    };
    \item $l_2$ -- сообщений сети {\it Twitter} за январь 2016 года
        (проставление тональных меток на основе списка позитивных и негативных
        эмотиконов);
    \item $l_3$ -- лексикон $SentiRuLex$ созданный вручную экспертами
        \cite{expertLexicon}.
\end{itemize}

\subsection{Построение лексиконов}

Построение лексикона производится на основе {\it меры взаимной информации}
\cite{lexiconSO}:
\begin{gather}
    PMI(t_1, t_2) = log_2 \dfrac{P(t_1\cap t_2)}{P(t_1)\cdot P(t_2)}
    \label{eq:pmi}
\end{gather}

Такая мера показывает связь $t_1$ с $t_2$, т.е. какова вероятность их связи.
В качестве второго аргумента, рассмотрим метку, которая будет соответствовать
одному из тональных классов:
\begin{itemize}
    \item {\it Excellent} -- положительный оттенком;
    \item {\it Poor} -- негативный оттенком.
\end{itemize}

Таким образом, относительно каждого маркера по формуле \ref{eq:pmi} можно
установить степень связи $t_1$ с положительным и негативным оттенком.
На основе разности значений можно определить тональность терма, или его
{\it семантическую ориентацию} (формула \ref{eq:so}).
\begin{equation}
    \label{eq:so}
    SO(t) = PMI(t, Excellent) - PMI(t, Poor)
\end{equation}

От исходной коллекции $K$, на основе которой будет создан лексикон, требуется
чтобы все сообщения коллекции были размечены по тональным классам.
Т.е., для каждого сообщения должна быть определена метка {\it Excellent} или
{\it Poor}, которая характеризует тональность сообщения в целом.
Такую метку можно проставить автоматически, и применительно к сообщениям
сети {\it Twitter} на основе {\it эмотиконов} или {\it хэштегов} \cite{severyn}.
В результате лексикон строится следующим образом:
\begin{equation}
    S : \{ \left< t, SO(t) \right> | t \in K\}
\end{equation}


Лексиконы были составлены на основе следующих данных (параметры представлены
в таблице \ref{table:createdLexicons}):

\begin{itemize}
    \item $l_1$ -- корпус коротких текстов на русском языке Ю.~Рубцовой\footnote{
        \url{https://github.com/nicolay-r/tone-classifier/tree/2016_jan_contest/data/lexicons}
    };
    \item $l_2$ -- сообщений сети {\it Twitter} за январь 2016 года
        (проставление тональных меток на основе списка позитивных и негативных
        эмотиконов);
    \item $l_3$ -- лексикон $SentiRuLex$ созданный вручную экспертами
        \cite{expertLexicon}.
\end{itemize}

\subsection{Построение лексиконов}

Построение лексикона производится на основе {\it меры взаимной информации}
\cite{lexiconSO}:
\begin{gather}
    PMI(t_1, t_2) = log_2 \dfrac{P(t_1\cap t_2)}{P(t_1)\cdot P(t_2)}
    \label{eq:pmi}
\end{gather}

Такая мера показывает связь $t_1$ с $t_2$, т.е. какова вероятность их связи.
В качестве второго аргумента, рассмотрим метку, которая будет соответствовать
одному из тональных классов:
\begin{itemize}
    \item {\it Excellent} -- положительный оттенком;
    \item {\it Poor} -- негативный оттенком.
\end{itemize}

Таким образом, относительно каждого маркера по формуле \ref{eq:pmi} можно
установить степень связи $t_1$ с положительным и негативным оттенком.
На основе разности значений можно определить тональность терма, или его
{\it семантическую ориентацию} (формула \ref{eq:so}).
\begin{equation}
    \label{eq:so}
    SO(t) = PMI(t, Excellent) - PMI(t, Poor)
\end{equation}

От исходной коллекции $K$, на основе которой будет создан лексикон, требуется
чтобы все сообщения коллекции были размечены по тональным классам.
Т.е., для каждого сообщения должна быть определена метка {\it Excellent} или
{\it Poor}, которая характеризует тональность сообщения в целом.
Такую метку можно проставить автоматически, и применительно к сообщениям
сети {\it Twitter} на основе {\it эмотиконов} или {\it хэштегов} \cite{severyn}.
В результате лексикон строится следующим образом:
\begin{equation}
    S : \{ \left< t, SO(t) \right> | t \in K\}
\end{equation}


Лексиконы были составлены на основе следующих данных (параметры представлены
в таблице \ref{table:createdLexicons}):

\begin{itemize}
    \item $l_1$ -- корпус коротких текстов на русском языке Ю.~Рубцовой\footnote{
        \url{https://github.com/nicolay-r/tone-classifier/tree/2016_jan_contest/data/lexicons}
    };
    \item $l_2$ -- сообщений сети {\it Twitter} за январь 2016 года
        (проставление тональных меток на основе списка позитивных и негативных
        эмотиконов);
    \item $l_3$ -- лексикон $SentiRuLex$ созданный вручную экспертами
        \cite{expertLexicon}.
\end{itemize}

\subsection{Построение лексиконов}

Построение лексикона производится на основе {\it меры взаимной информации}
\cite{lexiconSO}:
\begin{gather}
    PMI(t_1, t_2) = log_2 \dfrac{P(t_1\cap t_2)}{P(t_1)\cdot P(t_2)}
    \label{eq:pmi}
\end{gather}

Такая мера показывает связь $t_1$ с $t_2$, т.е. какова вероятность их связи.
В качестве второго аргумента, рассмотрим метку, которая будет соответствовать
одному из тональных классов:
\begin{itemize}
    \item {\it Excellent} -- положительный оттенком;
    \item {\it Poor} -- негативный оттенком.
\end{itemize}

Таким образом, относительно каждого маркера по формуле \ref{eq:pmi} можно
установить степень связи $t_1$ с положительным и негативным оттенком.
На основе разности значений можно определить тональность терма, или его
{\it семантическую ориентацию} (формула \ref{eq:so}).
\begin{equation}
    \label{eq:so}
    SO(t) = PMI(t, Excellent) - PMI(t, Poor)
\end{equation}

От исходной коллекции $K$, на основе которой будет создан лексикон, требуется
чтобы все сообщения коллекции были размечены по тональным классам.
Т.е., для каждого сообщения должна быть определена метка {\it Excellent} или
{\it Poor}, которая характеризует тональность сообщения в целом.
Такую метку можно проставить автоматически, и применительно к сообщениям
сети {\it Twitter} на основе {\it эмотиконов} или {\it хэштегов} \cite{severyn}.
В результате лексикон строится следующим образом:
\begin{equation}
    S : \{ \left< t, SO(t) \right> | t \in K\}
\end{equation}


Лексиконы были составлены на основе следующих данных (параметры представлены
в таблице \ref{table:createdLexicons}):

\begin{itemize}
    \item $l_1$ -- корпус коротких текстов на русском языке Ю.~Рубцовой\footnote{
        \url{https://github.com/nicolay-r/tone-classifier/tree/2016_jan_contest/data/lexicons}
    };
    \item $l_2$ -- сообщений сети {\it Twitter} за январь 2016 года
        (проставление тональных меток на основе списка позитивных и негативных
        эмотиконов);
    \item $l_3$ -- лексикон $SentiRuLex$ созданный вручную экспертами
        \cite{expertLexicon}.
\end{itemize}

\input{parts/description/tables/lexicons}



