\subsection{Построение лексиконов}

Построение лексикона производится на основе {\it меры взаимной информации}
\cite{lexiconSO}:
\begin{gather}
    PMI(t_1, t_2) = log_2 \dfrac{P(t_1\cap t_2)}{P(t_1)\cdot P(t_2)}
    \label{eq:pmi}
\end{gather}

Такая мера показывает связь $t_1$ с $t_2$, т.е. какова вероятность их связи.
В качестве второго аргумента, рассмотрим метку, которая будет соответствовать
одному из тональных классов:
\begin{itemize}
    \item {\it Excellent} -- положительный оттенком;
    \item {\it Poor} -- негативный оттенком.
\end{itemize}

Таким образом, относительно каждого маркера по формуле \ref{eq:pmi} можно
установить степень связи $t_1$ с положительным и негативным оттенком.
На основе разности значений можно определить тональность терма, или его
{\it семантическую ориентацию} (формула \ref{eq:so}).
\begin{equation}
    \label{eq:so}
    SO(t) = PMI(t, Excellent) - PMI(t, Poor)
\end{equation}

От исходной коллекции $K$, на основе которой будет создан лексикон, {\bf требуется
чтобы все сообщения коллекции были размечены по тональным классам}.
Т.е., для каждого сообщения должна быть определена метка {\it Excellent} или
{\it Poor}, которая характеризует тональность сообщения в целом.
Такую метку можно проставить автоматически, и применительно к сообщениям
сети {\it Twitter} на основе {\it эмотиконов} или {\it хэштегов} \cite{severyn}.
В результате лексикон строится следующим образом:
\begin{equation}
    S : \{ \left< t, SO(t) \right> | t \in K\}
\end{equation}

\subsubsection{Составленные лексиконы}
На основе описанного подхода были составлены следующие лексиконы
(параметры представлены в таблице \ref{table:createdLexicons}).
\subsection{Построение лексиконов}

Построение лексикона производится на основе {\it меры взаимной информации}
\cite{lexiconSO}:
\begin{gather}
    PMI(t_1, t_2) = log_2 \dfrac{P(t_1\cap t_2)}{P(t_1)\cdot P(t_2)}
    \label{eq:pmi}
\end{gather}

Такая мера показывает связь $t_1$ с $t_2$, т.е. какова вероятность их связи.
В качестве второго аргумента, рассмотрим метку, которая будет соответствовать
одному из тональных классов:
\begin{itemize}
    \item {\it Excellent} -- положительный оттенком;
    \item {\it Poor} -- негативный оттенком.
\end{itemize}

Таким образом, относительно каждого маркера по формуле \ref{eq:pmi} можно
установить степень связи $t_1$ с положительным и негативным оттенком.
На основе разности значений можно определить тональность терма, или его
{\it семантическую ориентацию} (формула \ref{eq:so}).
\begin{equation}
    \label{eq:so}
    SO(t) = PMI(t, Excellent) - PMI(t, Poor)
\end{equation}

От исходной коллекции $K$, на основе которой будет создан лексикон, требуется
чтобы все сообщения коллекции были размечены по тональным классам.
Т.е., для каждого сообщения должна быть определена метка {\it Excellent} или
{\it Poor}, которая характеризует тональность сообщения в целом.
Такую метку можно проставить автоматически, и применительно к сообщениям
сети {\it Twitter} на основе {\it эмотиконов} или {\it хэштегов} \cite{severyn}.
В результате лексикон строится следующим образом:
\begin{equation}
    S : \{ \left< t, SO(t) \right> | t \in K\}
\end{equation}


Лексиконы были составлены на основе следующих данных (параметры представлены
в таблице \ref{table:createdLexicons}):

\begin{itemize}
    \item $l_1$ -- корпус коротких текстов на русском языке Ю.~Рубцовой\footnote{
        \url{https://github.com/nicolay-r/tone-classifier/tree/2016_jan_contest/data/lexicons}
    };
    \item $l_2$ -- сообщений сети {\it Twitter} за январь 2016 года
        (проставление тональных меток на основе списка позитивных и негативных
        эмотиконов);
    \item $l_3$ -- лексикон $SentiRuLex$ созданный вручную экспертами
        \cite{expertLexicon}.
\end{itemize}

\subsection{Построение лексиконов}

Построение лексикона производится на основе {\it меры взаимной информации}
\cite{lexiconSO}:
\begin{gather}
    PMI(t_1, t_2) = log_2 \dfrac{P(t_1\cap t_2)}{P(t_1)\cdot P(t_2)}
    \label{eq:pmi}
\end{gather}

Такая мера показывает связь $t_1$ с $t_2$, т.е. какова вероятность их связи.
В качестве второго аргумента, рассмотрим метку, которая будет соответствовать
одному из тональных классов:
\begin{itemize}
    \item {\it Excellent} -- положительный оттенком;
    \item {\it Poor} -- негативный оттенком.
\end{itemize}

Таким образом, относительно каждого маркера по формуле \ref{eq:pmi} можно
установить степень связи $t_1$ с положительным и негативным оттенком.
На основе разности значений можно определить тональность терма, или его
{\it семантическую ориентацию} (формула \ref{eq:so}).
\begin{equation}
    \label{eq:so}
    SO(t) = PMI(t, Excellent) - PMI(t, Poor)
\end{equation}

От исходной коллекции $K$, на основе которой будет создан лексикон, требуется
чтобы все сообщения коллекции были размечены по тональным классам.
Т.е., для каждого сообщения должна быть определена метка {\it Excellent} или
{\it Poor}, которая характеризует тональность сообщения в целом.
Такую метку можно проставить автоматически, и применительно к сообщениям
сети {\it Twitter} на основе {\it эмотиконов} или {\it хэштегов} \cite{severyn}.
В результате лексикон строится следующим образом:
\begin{equation}
    S : \{ \left< t, SO(t) \right> | t \in K\}
\end{equation}


Лексиконы были составлены на основе следующих данных (параметры представлены
в таблице \ref{table:createdLexicons}):

\begin{itemize}
    \item $l_1$ -- корпус коротких текстов на русском языке Ю.~Рубцовой\footnote{
        \url{https://github.com/nicolay-r/tone-classifier/tree/2016_jan_contest/data/lexicons}
    };
    \item $l_2$ -- сообщений сети {\it Twitter} за январь 2016 года
        (проставление тональных меток на основе списка позитивных и негативных
        эмотиконов);
    \item $l_3$ -- лексикон $SentiRuLex$ созданный вручную экспертами
        \cite{expertLexicon}.
\end{itemize}

\subsection{Построение лексиконов}

Построение лексикона производится на основе {\it меры взаимной информации}
\cite{lexiconSO}:
\begin{gather}
    PMI(t_1, t_2) = log_2 \dfrac{P(t_1\cap t_2)}{P(t_1)\cdot P(t_2)}
    \label{eq:pmi}
\end{gather}

Такая мера показывает связь $t_1$ с $t_2$, т.е. какова вероятность их связи.
В качестве второго аргумента, рассмотрим метку, которая будет соответствовать
одному из тональных классов:
\begin{itemize}
    \item {\it Excellent} -- положительный оттенком;
    \item {\it Poor} -- негативный оттенком.
\end{itemize}

Таким образом, относительно каждого маркера по формуле \ref{eq:pmi} можно
установить степень связи $t_1$ с положительным и негативным оттенком.
На основе разности значений можно определить тональность терма, или его
{\it семантическую ориентацию} (формула \ref{eq:so}).
\begin{equation}
    \label{eq:so}
    SO(t) = PMI(t, Excellent) - PMI(t, Poor)
\end{equation}

От исходной коллекции $K$, на основе которой будет создан лексикон, требуется
чтобы все сообщения коллекции были размечены по тональным классам.
Т.е., для каждого сообщения должна быть определена метка {\it Excellent} или
{\it Poor}, которая характеризует тональность сообщения в целом.
Такую метку можно проставить автоматически, и применительно к сообщениям
сети {\it Twitter} на основе {\it эмотиконов} или {\it хэштегов} \cite{severyn}.
В результате лексикон строится следующим образом:
\begin{equation}
    S : \{ \left< t, SO(t) \right> | t \in K\}
\end{equation}


Лексиконы были составлены на основе следующих данных (параметры представлены
в таблице \ref{table:createdLexicons}):

\begin{itemize}
    \item $l_1$ -- корпус коротких текстов на русском языке Ю.~Рубцовой\footnote{
        \url{https://github.com/nicolay-r/tone-classifier/tree/2016_jan_contest/data/lexicons}
    };
    \item $l_2$ -- сообщений сети {\it Twitter} за январь 2016 года
        (проставление тональных меток на основе списка позитивных и негативных
        эмотиконов);
    \item $l_3$ -- лексикон $SentiRuLex$ созданный вручную экспертами
        \cite{expertLexicon}.
\end{itemize}

\input{parts/description/tables/lexicons}




$l_1$ -- на основе корпуса сообщений сети {\it Twitter} Ю.~Рубцовой.
Корпус состоит и распространяется в формате двух независимых коллекций
сообщений: {\it positive} и {\it negative}.

В таблице \ref{table:rubtsova_lexicon} приведены примеры наиболее тональных
термов составленного лексикона $l_2$.
%Наиболее эмоциональными термами получились эмотиконы (приведены в строках 1-3),
%значение которых изменяется по модулю от 6 до 15.
%Затем идут слова и хэштеги, максимальная оценка которых по модулю
%составляет $\approx6.5$.
\begin{table}[htp!]
\centering
\caption{Пример наиболее тональных термов лексикона $l_1$}
\label{table:rubtsova_lexicon}
\begin{tabular}{cccc}
\hline
\multicolumn{1}{c|}{Положительные}& \multicolumn{1}{c|}{оценка} & \multicolumn{1}{c|}{Отрицательные} & оценка \\ \hline
\#улыбнуло            & +4.71 & теракт   & -6.88 \\
позаимствовать        & +4.42 & некролог & -6.20 \\
крутотень             & +4.35 & погибший & -6.16 \\
еееее                 & +4.35 & \#сми    & -5.12 \\
бесподобный           & +4.21 & траур    & -5.08 \\
ржач                  & +4.21 & критический & -5.03 \\
позитив               & +4.05 & поч      & -4.89 \\ \hline
%улыбнуть              & +4.04 & хнык     & -4.81 \\ \hline
\end{tabular}
\end{table}


$l_2$ -- на основе сообщений сети {\it Twitter}, собранных в течение января {\it 2016} года.
Сообщения сети извлекались с помощью {\it Streaming Twitter API}\footnote{
    Из всего потока принимались только русскоязычные сообщения;
}.
Определение тонального класса каждого из сообщений производилось на
основе содержащихся в нем эмотиконов.

Для этого были составлены два множества эмотиконов (положительные и отрицательные).
Сообщение отмечалось меткой {\it Excellent}, если оно содержало
только эмотиконы положительного множества.
Аналогично в случае, если все эмотиконы сообщения принадлежали
отрицательному множеству, то сообщение отмечалось меткой {\it Poor}.
Сообщение не рассматривалось, если в нем отсутствовали эмотиконы.

В таблице \ref{table:jan_lexicon} приведены примеры наиболее тональных термов
составленного лексикона $l_2$.
\begin{table}[htp!] \centering
\caption{Пример наиболее тональных термов лексикона $l_2$}
\label{table:jan_lexicon}
\begin{tabular}{cccc}
\hline
\multicolumn{1}{c|}{Положительные}& \multicolumn{1}{c|}{оценка} & \multicolumn{1}{c|}{Отрицательные} & оценка \\ \hline
XD                                & +7.43                       & злостный                           & -7.62 \\
\#badoo                           & +7.23                       & ухудшение                          & -6.18 \\
\#happynewyear                    & +4.39                       & хнык                               & -6.10 \\
XDD                               & +4.08                       & ужест                              & -6.03 \\
выздоровление                     & +3.79                       & противоречить                      & -5.85 \\
бодрый                            & +3.74                       & нерешенный                         & -5.81 \\ \hline
%активный                          & +3.73                       & виновник                           & -5.74 \\ \hline
\end{tabular}
\end{table}


$l_3$ -- словарь оценочных слов $SentiRuLex$, порожденный
автоматически на основе извлечения информации из нескольких источников,
а затем проверенный вручную экспертами.

Из нескольких источников разных предметных областей были
извлечены списки оценочных слов (и словосочетаний) с проставленными
весами оценочности:
\begin{itemize}
    \item Оценочные слова тезауруса РуТез;
    \item Сленговые слова сети Twitter;
    \item Слова с позитивными или негативными ассоциациями из корпуса новостей.
\end{itemize}
Содержимое списков было сопоставлено с тезаурусом РуТез и все понятия
были извлечены для анализа экспертом.
Эксперт выполнял следующие задачи:
проверка и уточнение тональной оценки слов исходного списка,
уточние оценки значений слова,
и проверка полноту создаваемого словаря (за счет списка синонимов из
близких понятий тезауруса).

Каждая единицa словаря (т.е. слово или словосочетание) состоит из набора
атрибутов, представленного в табл. \ref{table:sentiRuLexItemFormat}.

\begin{table}[htp!]
\centering
\caption{Формат представления единицы словаря {\it SentiRuLex}}
\label{table:sentiRuLexItemFormat}
\begin{tabular}{lcc}
\hline
\multicolumn{1}{l|}{Атрибут}                                                 & \multicolumn{1}{c|}{Возможные значения} & Пример    \\ \hline
Слово или фраза                                                              & строка                                  & пресный   \\
Часть речи                                                                   & Adj, Noun                               & Adj       \\
Лемматизированная форма                                                      & строка                                  & пресный   \\
Тональность                                                                  & positive, negative, neutral             & negative  \\
Источник тональности                                                         & opinion, emotion, fact                  & emotion   \\
\begin{tabular}[c]{@{}l@{}}Отсылки к понятиям\\ тезауруса РуТез\end{tabular} & строка (возможно пустая)                & Невкусный \\ \hline
\end{tabular}
\end{table}


Для построения лексикона были рассмотрены только те единицы словаря,
которые описывали слово (т.е. словосочетания игнорировались).
Из всех атрибутов таблицы \ref{table:sentiRuLexItemFormat} использовались
только {\it лемматизированная форма} и {\it тональность}.
Атрибут тональности преобразовывался в {\it числовую тональность}
(-1 -- negative,
0 -- neutral,
1 -- positive).
Поскольку некоторые слова могут быть представлены
в нескольких различных тональностях, то будем рассматривать только те из них,
тональность которых определена однозначно в рамках всего словаря.

В результате, список пар $\left< \text{лемма, числовая тональность} \right>$
образует результирующий лексикон $l_3$, объем которого составляет $\approx10$ тыс. слов
(см. табл. \ref{table:createdLexicons}).

