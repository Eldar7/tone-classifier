\subsection{Коллекции данных для обучения}
    \label{sec:train}
    % Здесь рассказываем про коллекции, которые использовались несбалансированные для обучения коллекции
    Для обучения классификатора предполагается использовать соответствующие
    коллекции данных соревнований {\it SentiRuEval} (см. таблицу
    \ref{table:trainCollections}).
    Коллекции $I_{bank}^{16}, \hspace{0.1cm} I_{tcc}^{16}$ являются объединением
    размеченных экспертами данных за 2015 и 2016 года.

    \begin{table}[htp!]
\caption{Обучающие коллеции предоставленные организаторами.
        {\bf $N_+, N_0, N_-$} -- число сообщений положительного, нейтрального и
        негативного классов соответственно;
        {\bf $\sum$} -- общее число сообщений в коллекции;
        нейтральный класс является {\bf наиболее частотным}.
    }
\label{table:trainCollections}
\centering
\begin{tabular}{lcccc}
\hline
\multicolumn{1}{c|}{Название} & \multicolumn{1}{c|}{$N_+$} & \multicolumn{1}{c|}{$N_0$} & \multicolumn{1}{c|}{$N_-$} & $\sum$            \\ \hline
    $I_{bank}^{15}$           & 356                        & \textbf{3\hspace{2pt}482}  & 1\hspace{2pt}077           & 4\hspace{2pt}915  \\
    $I_{tcc}^{15}$            & 956                        & \textbf{2\hspace{2pt}269}  & 1\hspace{2pt}634           & 4\hspace{2pt}859  \\
    $I_{bank}^{16}$           & 1\hspace{2pt}354           & \textbf{4\hspace{2pt}870}  & 2\hspace{2pt}550           & 8\hspace{2pt}783  \\
    $I_{tcc}^{16}$            & 704                        & \textbf{6\hspace{2pt}756}  & 1\hspace{2pt}741           & 9\hspace{2pt}102  \\ \hline
\end{tabular}
\end{table}


    Поскольку в предоставляемых
    данных число тональных сообщений существенно уступает объему класса
    нейтральных сообщений, то дополнительно планируется создать {\it сбалансированную
    обучающую коллекцию}.
    В работе \cite{diploma2015}, применительно к классификаторам {\it
    Наивного Байеса} и {\it SVM}, отмечается существенный прирост качества при
    использовании коллекций сбалансированного типа.

    % Про балансировку коллекций в том числе.
    Для решения подобной задачи воспользуемся готовым общедоступным корпусом Ю.~Рубцовой
    \cite{rubtsovaCollection}, в
    котором каждое сообщение автоматически распределено в одну из тональных групп:
    {\it positive} и {\it negative}.
    Объем каждого класса такой коллекции составляет {\it $\approx$ 110 тыс.
    сообщений}

    % (Как производить балансировку)
    Для построения сбалансированной коллекции требуется существенно меньшее
    число сообщений чем предлагается в тональном корпусе.
    В связи с этим, выберем небольшой процент наиболее эмоциональных сообщений:
    \begin{enumerate}
        \item Построим лексикон $l$ на основе корпуса Ю.~Рубцовой для определения
            списка наиболее эмоциональных термов.
        \item Сообщение $m$ будем считать {\it наиболее эмоциональным},
            если для него выполнено следующее условие:
            \begin{gather}
                \max\limits_{i=1..N} |l(t_i)| > B
            \end{gather}
            Где $B$ -- величина порогового значения; \hspace{0.5pt}
            $t_i$ -- термы сообщения $m$; \hspace{0.5pt}
            $N$ -- общее количество термов в сообщении $m$;
    \end{enumerate}

    Таким образом были сбалансированы коллеции таблицы \ref{table:trainCollections}.
    Параметры дополнительно составленных коллекций представлены
    в таблице \ref{table:balancedTrainCollections}.

    \begin{table}[htp!]
\centering
\caption{Сбалансированные обучающие коллекции;
    $N_*$ -- размер класса коллекции;
    $\sum$ -- общее число сообщений в коллекции.
}
\label{table:balancedTrainCollections}
\begin{tabular}{ccccc}
    \hline
    \multicolumn{1}{c|}{\multirow{2}{*}{Название}} & \multicolumn{2}{c|}{BANK}                                & \multicolumn{2}{c}{TCC}               \\ \cline{2-5}
    \multicolumn{1}{c|}{}                          & \multicolumn{1}{c|}{$N_*$} & \multicolumn{1}{c|}{$\sum$} & \multicolumn{1}{c|}{$N_*$} & $\sum$   \\ \hline
    $B_{15}$                                       & $3’400$                    & $10’400$                    & $2’269$                    & $6’888$  \\
    $B_{16}$                                       & $6’756$                    & $20’268$                    & $4’870$                    & $14’610$ \\ \hline
\end{tabular}
\end{table}

