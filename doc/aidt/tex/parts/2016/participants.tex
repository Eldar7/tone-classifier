\subsection{Сравнение с результатами участников}
Каждому участнику предлагалось протестировать несколько прогонов.
По каждому из участнику, из всех полученных результатов были отобраны
наилучшие и внесены в таблицу \ref{table:comparison}.
Описание подходов участников соревнований представлено в таблице
\ref{table:participants}.
Для сравнения с участниками, результаты текущего подхода в таблице
\ref{table:comparison} представлены в двух вариантах:
\begin{enumerate}
    \item На момент участия в соревнованиях: настройки прогона описаны
        в таблице \ref{table:participants} (участник №1);
    \item После окончания соревнований: прогон №6 с применением настроек
        классификатора (наилучший результат таблицы \ref{table:cost}).
\end{enumerate}
\begin{table}[htp!]
\centering
\caption{Сравнение описанного в статье подхода с результатами остальных
    участников соревнований {\it SentiRuEval-2016}; {\bf жирным шрифтом}
    отмечен участник с наилучшими результатами}
\label{table:comparison}
\begin{tabular}{ccc}
\hline
Участник   & BANK           & TCC             \\ \hline
1          & 46.83          & 52.86           \\
\textbf{2} & \textbf{55.17} & \textbf{55.94}  \\
3          & 34.23          & 39.94           \\
4          & 37.30          & 49.55           \\
5          & 38.59          & 34.99           \\
6          & 23.98          & 35.45           \\
7          & 47.10          & 48.42           \\
8          & 44.92          & 48.71           \\
9          & 51.95          & 54.89           \\
10         & 46.59          & 50.55           \\ \hline
\end{tabular}
\end{table}


\begin{table}[htp!]
\centering
\caption{Настройки прогонов участников соревнований}
\label{table:participants}
\begin{tabular}{cp{13.2cm}}
\hline
Участник        & \multicolumn{1}{|c}{Настройки лучших прогонов}                                                                                                                                                                                                                     \\ \hline
1               & Настройки подхода №5, таблица \ref{table:settings} (статья конференции {\it Диалог-2016} \cite{myArticle})                                                                                                                                                                                                        \\
2               & Рекуррентная нейронная сетка ({\it LSTM}); в качестве признаков {\it Word2Vec}, обученный на внешней коллекции (посты и комментарии из социальных сетей) \cite{neuralNetworks}.                                                                                                                                            \\
4               & Словарные признаки + признаки мета-классификаторов (логистическая регрессия, ридж-регрессия, классификатор на основе градиентного бустинга и классификатор на основе нейронной сети) и линейный {\it SVM} в качестве классификатора.                                                                 \\
8               & Поиск эмоциональных слов по словарю (200 тыс. словоформ), правила их комбинирования на основе синтаксического анализа; применение онтологических правил, характерных для данной предметной области                                                                                                  \\
9               & {\it SVM}: униграммы, биграммы, словарь {\it SentiRuLex}, учет частей речи, многозначных слов (автоматический словарь коннотаций по новостям для TKK задачи)                                                                                                                                         \\
10              & {\it SVM}, в качестве признаков использовались униграммы, подвергшиеся преобразованиям ({\it не + слово = один признак}, множественные повторения символов заменяются двукратным; ссылки, ответы, даты, числа – заменяются паттернами и другие преобразования); подключение словаря {\it SentiRuLex}.\\ \hline
\end{tabular}
\end{table}



Практически все участники соревнований отдали предпочтения использованию
SVM классификатора.
Для улучшения качества работы наблюдаются схожие с рассматриваемой в данной
работе методики -- использование преобразований текста и
дополнительных признаков, часть из которых построена на основе лексиконов.
Так, при использовании словаря оценочной лексики {\it SentiRuLex} в подходах 9 и 10
(см. таблицу \ref{participants}) можно наблюдать примерную схожесть с результатами,
полученных в этой статье (подход №1).
Следует обратить внимание на высокие показания по микромере для задачи {\it TCC},
которые составляют  $F_{1-micro}^{PN} \approx 68.0$.

Исключением является участник №2, который классифицировал сообщения с помощью
нейронных сетей \cite{neuralNetworks}.
За последнее вреия произошел резкий скачок в этой области, что привело
к бурному развитию этого направления.
Для достижения наилучшего результата среди всех участников, авторы
экспериментировали и отправили три различных решения:
\begin{enumerate}
    \item Построение сверточной нейронной сети, и добавление признаков на
        основе {\it Word2Vec};
    \item Применение рекуррентной нейронной сетки с добавлением признаков
        на основе {\it Word2Vec}, который был обучен на внешней коллекции
        постов и комментариев социальных сетей;
    \item Объединение результатов двух предыдущих решений + результата
        SVM с полиноминальным ядром над усредненным word2vec.
\end{enumerate}

Наилучший результат был достигнут при использовании при использовании сети
второго типа
