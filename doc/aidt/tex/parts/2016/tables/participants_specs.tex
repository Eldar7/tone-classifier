\begin{table}[htp!]
\centering
\caption{Настройки прогонов участников соревнований}
\label{table:participants}
\begin{tabular}{cp{13.2cm}}
\hline
Участник        & \multicolumn{1}{|c}{Настройки прогона}                                                                                                                                                                                                                     \\ \hline
1               & Текущая работа (статья конференции {\it Диалог-2016} \cite{myArticle})                                                                                                                                                                                                        \\
2               & Рекуррентная нейронная сетка ({\it LSTM}); в качестве признаков {\it Word2Vec}, обученный на внешней коллекции (посты и комментарии из социальных сетей) \cite{neuralNetworks}.                                                                                                                                            \\
4               & Словарные признаки + признаки мета-классификаторов (логистическая регрессия, ридж-регрессия, классификатор на основе градиентного бустинга и классификатор на основе нейронной сети) и линейный {\it SVM} в качестве классификатора.                                                                 \\
8               & Поиск эмоциональных слов по словарю (200 тыс. словоформ), правила их комбинирования на основе синтаксического анализа; применение онтологических правил, характерных для данной предметной области                                                                                                  \\
9               & {\it SVM}: униграммы, биграммы, словарь {\it SentiRuLex}, учет частей речи, многозначных слов (автоматический словарь коннотаций по новостям для TKK задачи)                                                                                                                                         \\
10              & {\it SVM}, в качестве признаков использовались униграммы, подвергшиеся преобразованиям ({\it не + слово = один признак}, множественные повторения символов заменяются двукратным; ссылки, ответы, даты, числа – заменяются паттернами и другие преобразования); подключение словаря {\it SentiRuLex}.\\ \hline
\end{tabular}
\end{table}

