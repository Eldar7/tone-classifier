\section{Участие в соревнованях SentiRuEval-2016}
\label{sec:sentirueval2016}

Посмотрим насколько изменятся результаты, если применить классификатор к
коллекции данных {\it SentiRuEval-2016}, используя настройки таблицы
\ref{table:settings}.
Для обучения классификатора применим все коллекции, представленные в
п. \ref{sec:train}.
Полученные результаты представлены в таблице \ref{table:results2016}.

\begin{table}[ht!]
\centering
\caption{Результаты $F_{1-macro}^{PN}$ тестирования на коллекции {\it SentiRuEval-2016};
        {\bf жирным шрифтом} отмечается лучший результат по каждой задаче.
        }
\label{table:results2016}
\begin{tabular}{ccccc}
\hline
\multicolumn{1}{c|}{\multirow{2}{*}{№}} & \multicolumn{2}{c|}{BANK} & \multicolumn{2}{c}{TCC}\\ \cline{2-5}
\multicolumn{1}{c|}{}                   & \multicolumn{1}{c|}{$I$} & \multicolumn{1}{c|}{$B$} & \multicolumn{1}{c|}{$I$} & $B$             \\ \hline
1                                       & ${48.77}$                              & $45.53$                              & $48.32$                             & $50.90$                    \\
2                                       & ${50.24}$                              & $47.36$                              & $49.48$                             & $50.69$                    \\
3                                       & ${50.28}$                              & $47.53$                              & $48.90$                             & $51.95$                    \\
4                                       & ${50.45}$                              & $47.13$                              & $49.31$                             & $52.72$                    \\
5                                       & ${49.72}$                              & $46.99$                              & $50.55$                             & $\bf{52.90}$               \\
6                                       & ${\bf 51.73}$                          & $50.25$                              & $51.66$                             & $52.63$                    \\ \hline
\end{tabular}
\end{table}
\cellcolor[HTML]{C0C0C0}
\cellcolor[HTML]{C0C0C0}
\cellcolor[HTML]{C0C0C0}
\cellcolor[HTML]{C0C0C0}
\cellcolor[HTML]{C0C0C0}
\cellcolor[HTML]{C0C0C0}


Анализируя полученные результаты можно сказать, что увеличения числа признаков
положительно сказывается на качестве работы классификатора, независимо от
типа задачи.

Что касается использования сбалансированных обучающих коллекций, то
однозначный прирост наблюдается в задаче TCC, в частности при обучении на
$B_{tcc}^{16}$.
Для задачи BANK наилучший результат достигается при использовании $I_{bank}^{16}$.
Возможно что данные, которые использовались для расширения в $B_{bank}^{16}$
привели к переобучению, что выражается падением качества классификации.
