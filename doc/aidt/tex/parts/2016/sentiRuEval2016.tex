\section{Участие в соревнованях SentiRuEval-2016}
\label{sec:sentirueval2016}

В январе 2016 годa соревнования по тональной классификации сообщений
сети Твиттер были продолжены.
В качестве областей были выбраны все те же коллекции отзывов о банках,
и телекоммуникационных компаниях. В период с сентября 2015 года и до начала
2016 года составлялись тестовые коллекции $BANK_{16}$ и $TCC_{16}$ (см.
таблицу \ref{table:testCollections}).

Таким образом, весь набор необходимых коллекций для тестирования, рассмотрим
результаты, которые были получены при применении разных прогонов таблицы
\ref{table:settings}.
Полученные результаты представлены в таблице \ref{table:results2016}.
Для обучения классификатора были рассмотрены все коллекции, представленные в
таблицах \ref{table:trainCollections} и \ref{table:balancedTrainCollections}.

\begin{table}[ht!]
\centering
\caption{Результаты $F_{1-macro}^{PN}$ тестирования на коллекции {\it SentiRuEval-2016};
        {\bf жирным шрифтом} отмечается лучший результат по каждой задаче.
        }
\label{table:results2016}
\begin{tabular}{ccccc}
\hline
\multicolumn{1}{c|}{\multirow{2}{*}{№}} & \multicolumn{2}{c|}{BANK} & \multicolumn{2}{c}{TCC}\\ \cline{2-5}
\multicolumn{1}{c|}{}                   & \multicolumn{1}{c|}{$I$} & \multicolumn{1}{c|}{$B$} & \multicolumn{1}{c|}{$I$} & $B$             \\ \hline
1                                       & ${48.77}$                              & $45.53$                              & $48.32$                             & $50.90$                    \\
2                                       & ${50.24}$                              & $47.36$                              & $49.48$                             & $50.69$                    \\
3                                       & ${50.28}$                              & $47.53$                              & $48.90$                             & $51.95$                    \\
4                                       & ${50.45}$                              & $47.13$                              & $49.31$                             & $52.72$                    \\
5                                       & ${49.72}$                              & $46.99$                              & $50.55$                             & $\bf{52.90}$               \\
6                                       & ${\bf 51.73}$                          & $50.25$                              & $51.66$                             & $52.63$                    \\ \hline
\end{tabular}
\end{table}
\cellcolor[HTML]{C0C0C0}
\cellcolor[HTML]{C0C0C0}
\cellcolor[HTML]{C0C0C0}
\cellcolor[HTML]{C0C0C0}
\cellcolor[HTML]{C0C0C0}
\cellcolor[HTML]{C0C0C0}


В результате можно наблюдать схожую картину, что и в п. \ref{sec:sentirueval2015}.
Во всех задачах наблюдается рост результата если увеличивать число признаков
на основе лексиконов (прогоны с №3 по №6, таблица \ref{table:results2016}).

Объем используемых коллекций и наличие актуальных данных положительно сказываются
на качестве работы классификатора.
Именно поэтому обучающие коллекции 2016 года
(результаты в серых столбцах таблицы \ref{table:results2016})
позволяют достичь значительно лучших результатов, что особенно
наблюдается в задаче BANK (низкие результаты при использовании $I_{bank}^{15}$
и $B_{bank}^{16}$).
Поэтому далее будем проводить анализ результатов относительно обучающих
коллекций 2016 года.
%$I_{bank/tcc}^{16}$ и $B_{bank/tcc}^{16}$.

При тестировании на задаче BANK, максимальный результат наблюдается
в подходе №6, и составляет $F_{1-macro}^{PN} = 51.73$.
Как и в тестировании п. \ref{sec:sentirueval2015}, результат последнего
дает средний прирост в $+1.6$ на коллекции $I_{bank}^{16}$ и в $+3.0$
при обучении на $B_{bank}^{16}$ при сравнении с прогонами №3 -- №5.

Для задачи TCC ситуация в целом аналогична банковской коллекции, за исключением
того что чуть лучший результат достигается в подходе №5, где
$F_{1-macro}^{PN} = 52.90$,
что на +0.27 лучше результата последнего прогона.
В итоге, результат последнего подхода №6 в очередной раз показывает полезность
применения нескольких разных признаков на основе лексиконов.

Сравнивая результаты при использовании разных типов обучающих коллекций,
прирост можно наблюдать на задаче TCC; при использовании $B_{tcc}^{16}$
составляет +2,26 в среднем для каждого прогона при сравнении с $I_{tcc}^{16}$.
Для задачи BANK ситуация обратная, поскольку разность в среднем по каждому
из прогонов качество при обучении на $B_{bank}^{16}$  и $I_{bank}^{16}$
составляет -2.73.
В следующем разделе мы рассмотрим, насколько изменятся эта разница после
настройки SVM классификатора.
