\section{Участие в соревнованях SentiRuEval-2016}
В таблице \ref{table:tkkResult2015} приведены оценки качества работы
классификатора для тестовой коллекции {\it SentiRuEval-2016} \cite{dialog2016}
при использовании настроек предварительного тестирования.
Прогоны с такими настройками показали лучшие результаты среди других
вариаций настроек предложенного подхода (см. таблицы \ref{table:bankResult2016}-\ref{table:tkkResult2016}).

    \begin{table}[!ht]
    \centering
    \caption{Результаты прогонов соревнования (задача BANK, {\it SentiRuEval-2016})}
    \label{table:bankResult2016}
    \begin{tabular}{|c|c|c|c|c|}
    \hline
    \multirow{3}{*}{№} & \multicolumn{4}{c|}{BANK --- сообщения о банковских компаниях}                                                                                                                                                                                         \\ \cline{2-5}
                       & \multicolumn{2}{c|}{\begin{tabular}[c]{@{}c@{}}Не сбалансированная \\ коллекция (2015 год)\end{tabular}} & \multicolumn{2}{c|}{\begin{tabular}[c]{@{}c@{}}Расширенная сбалансированная \\ коллекция\end{tabular}} \\ \cline{2-5}
                       & $F_{macro}(neg, pos)$                               & $F_{micro}(neg, pos)$                              & $F_{macro}(neg, pos)$                              & $F_{micro}(neg, pos)$                             \\ \hline
    1                  & 0.384                                               & 0.4203                                             & {\bf 0.4536 (+18.1\%)}                                   & {\bf 0.4982 (+18,53\%)}                                 \\ \hline
    2                  & 0.3849                                              & 0.415                                              & {\bf 0.4672 (+20.9\%)}                                   & {\bf 0.5029 (+21,1\%)}                                 \\ \hline
    3                  & 0.3862                                              & 0.4218                                             & {\bf 0.4683 (+21.25\%)}                                  & {\bf 0.5022(+19.06\%)}                                  \\ \hline
    \end{tabular}
    \end{table}

    \begin{table}[!ht]
    \centering
    \caption{Результаты прогонов соревнования (задача TKK, {\it SentiRuEval-2016})}
    \label{table:tkkResult2016}
    \begin{tabular}{|c|c|c|c|c|}
    \hline
    \multirow{3}{*}{№} & \multicolumn{4}{c|}{TKK --- сообщения о телекоммуникационных компаниях}                                                                                                                                                                                          \\ \cline{2-5}
                       & \multicolumn{2}{c|}{\begin{tabular}[c]{@{}c@{}}Не сбалансированная \\ коллекция (2015 год)\end{tabular}} & \multicolumn{2}{c|}{\begin{tabular}[c]{@{}c@{}}Расширенная сбалансированная \\ коллекция\end{tabular}} \\ \cline{2-5}
                       & $F_{macro}(neg, pos)$                               & $F_{micro}(neg, pos)$                              & $F_{macro}(neg, pos)$                              & $F_{micro}(neg, pos)$                             \\ \hline
    1                  & 0.4849                                              & 0.641                                              & {\bf 0.5103 (+5.2\%)}                                   & {\bf 0.6509 (+1.5\%) }                                  \\ \hline
    2                  & 0.4832                                              & 0.6473                                             & {\bf 0.5231 (+8.2\%)}                                    & {\bf 0.6508 (+0.5\%)}                                   \\ \hline
    3                  & 0.5099                                              & {\bf 0.677 (+2.0\%)}                                    & {\bf 0.5286 (+3.6\%)}                                    & 0.6632                                            \\ \hline
    \end{tabular}
    \end{table}

    \subsection{Улучшение результатов}

После проведения соревнований, в целях повышения качества классификации,
настройки прогонов изменялись в следующих направлениях:
\begin{enumerate}
    \item {\bf Настройка параметра $C$} штрафной функции SVM классификатора.
        По умолчанию $C=1$.
        Среди множества протестированных значений \{$1, 0.75, 0.5, 0.25$\},
        наибольший прирост достигается при {\bf $C = 0.5$} (см. таблицу \ref{table:cParameter}).
    \item {\bf Добавление новых признаков:} вычисление {\it максимальных} и
        {\it минимальных} значений (с учетом нормализации на основе формул
        \ref{eq:norm1}-\ref{eq:norm2}) среди всех термов сообщения по каждому
        из лексиконов.
        Пусть $l$ -- произвольный лексикон из всего множества $L$.
        Тогда относительно рассматриваемого лексикона $l$, для каждого сообщения
        $m = \{t_i\}_{i=1}^n$ вычисляются следующие признаки:
        \begin{equation}
            f_{max_l} = \max_{i=\overline{1 \ldots n}}l(t_i), t_i \in l \nonumber
        \end{equation}

        \begin{equation}
            f_{min_l} = \min\limits_{i=\overline{1 \ldots n}}l(t_i), t_i \in l \nonumber
        \end{equation}
\end{enumerate}

    \begin{table}[ht!]
    \centering
    \caption{Влияние настройки параметра Cost (С=0.5) ({\it SentiRuEval-2016})}
    \label{table:cParameter}
    \begin{tabular}{|c|c|c|c|c|}
    \hline
    \multirow{2}{*}{№} & \multicolumn{2}{c|}{\begin{tabular}[c]{@{}c@{}}BANK\\ (Расширенная сбалансированная\\ коллекция, C=0.5)\end{tabular}} & \multicolumn{2}{c|}{\begin{tabular}[c]{@{}c@{}}TKK\\ (Расширенная сбалансированная\\ коллекция, C=0.5)\end{tabular}} \\ \cline{2-5}
                       & $F_{macro}(neg, pos)$                                     & $F_{micro}(neg, pos)$                                     & $F_{macro}(neg, pos)$                                     & $F_{micro}(neg, pos)$                                    \\ \hline
    1                  & 0.4558 (+0.4\%)                                            & 0.5037 (+1.1\%)                                            & 0.5235 (+2.5\%)                                            & 0.6612 (+1.5\%)                                           \\ \hline
    2                  & {\bf 0.4795 (+2.6\%)}                                            & {\bf 0.5167 (+2.7\%)}                               & 0.5338 (+2.0\%)                                            & 0.6610 (+1.5\%)                                          \\ \hline
    3                  & 0.4768 (+1.8\%)                                            & 0.5135(+2.2\%)                                             & {\bf 0.5452 (+3.1\%) }                                           & {\bf 0.6733 (+1.5\%) }                                          \\ \hline
    \end{tabular}
    \end{table}


    Комбинация рассмотренных выше улучшений привела к настройке {\it финальных
прогонов} (результаты представлены в таблице \ref{table:finalResults}).
Во всех прогонах использовались русскоязычные термы и хэштеги, применялись
тональные префиксы, а также учитывались все признаки. Изменения в настройках
касались только числа используемых лексиконов, а также признаков построенных
на их основе (настройки прогонов):
    \begin{enumerate}
        \item Вычисление суммы, минимума, максимума на основе лексикона №1 (см. таблицу \ref{table:createdLexicons}).
        \item Прогон №1 + признаки суммы, минимума, максимума на основе лексикона №2.
        \item Прогон №2 + признаки суммы, минимума, максимума на основе лексикона №4.
        \item Прогон №3 + признаки минимума и максимума на основе лексиконов №3.
    \end{enumerate}

    \begin{table}[ht!]
    \centering
    \caption{Результаты финального тестирования {\it SentiRuEval-2016}}
    \label{table:finalResults}
    \begin{tabular}{|c|c|c|c|c|}
    \hline
    \multirow{2}{*}{№} & \multicolumn{2}{c|}{\begin{tabular}[c]{@{}c@{}}BANK\\ (Расширенная сбалансированная\\ коллекция, C=0.5)\end{tabular}} & \multicolumn{2}{c|}{\begin{tabular}[c]{@{}c@{}}TKK\\ (Расширенная сбалансированная\\ коллекция, C=0.5)\end{tabular}} \\ \cline{2-5}
                       & $F_{macro}(neg, pos)$                                     & $F_{micro}(neg, pos)$                                     & $F_{macro}(neg, pos)$                                     & $F_{micro}(neg, pos)$                                    \\ \hline
    1                  & 0.4955                                                    & 0.5388                                                    & 0.5259                                                    & 0.6662                                                   \\ \hline
    2                  & 0.5012                                                    & 0.5379                                                    & 0.5283                                                    & 0.6720                                                   \\ \hline
    3                  & \textbf{0.5239}                                           & \textbf{0.5514}                                           & \textbf{0.5453}                                           & \textbf{0.6970}                                          \\ \hline
    4                  & 0.4818                                                    & 0.5238                                                    & 0.5356                                                    & 0.6659                                                   \\ \hline
    \end{tabular}
    \end{table}

