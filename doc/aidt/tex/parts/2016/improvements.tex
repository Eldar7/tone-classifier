\subsection{Изменение настроек SVM классификатора}
Рассмотрим динамику изменения результатов, в зависимости от величины
отступа для разделения классов SVM классификатором.
В пакете LibSVM отступ изменяется на основе параметра $C$.
% Описать, чего мы хотим добиться
%Так, на основе резульататов таблиц \ref{table:results2015} и \ref{table:results2016}
%наблюдается стабильный рост качества при добавлении новых лексиконов и признаков
%на их основе, то интересно проверить, насколько будут меняться результаты и
%будут ли улучшения при меньших значениях отступа.

На рис. \ref{fig:cost} рассмотрены результаты прогонов с
изменением значения параметра, овечающего за величину отступа в диапазоне
$[0.1, 1]$ с шагом 0.1. Ранее, результаты таблицы \ref{table:results2016}
были получены при максимальном значении параметра в рассматриваемом диапазоне,
т.е. при $C= 1$.
Лучшие результаты по каждому были вынесены в таблицу \ref{table:cost}.

\begin{figure}[!htop] \centering
    \begin{subfigure}[b]{0.35\textwidth}
        % пересчитать
        \includegraphics[width=\textwidth]{pics/2016_ttk_imbalanced.png}
        \caption{$TCC_{16}, \hspace{0.2cm} I_{tcc}^{16}$}
        \label{fig:cost_ttk_2016_imb}
    \end{subfigure}
    ~
    \begin{subfigure}[b]{0.35\textwidth}
        \includegraphics[width=\textwidth]{pics/2016_ttk_balanced.png}
        \caption{$TCC_{16}, \hspace{0.2cm} B_{tcc}^{16}$}
        \label{fig:cost_ttk_2016_b}
    \end{subfigure}

    \begin{subfigure}[b]{0.35\textwidth}
        \includegraphics[width=\textwidth]{pics/2016_bank_imbalanced.png}
        \caption{$BANK_{16},\hspace{0.2cm} I_{bank}^{16}$}
        \label{fig:cost_bank_2016_imb}
    \end{subfigure}
    ~
    \begin{subfigure}[b]{0.35\textwidth}
        \includegraphics[width=\textwidth]{pics/2016_bank_balanced.png}
        \caption{$BANK_{16},\hspace{0.2cm} B_{bank}^{16}$}
        \label{fig:cost_bank_2016_b}
    \end{subfigure}

    \caption{
        Влияние {\it параметра штрафной функции SVM классификатора (C)}
        на результаты прогонов для задачи TCC и BANK на разных обучающих
        коллекциях;
        кривыми на графиках обозначены прогоны, номера которых соответствуют
        настройкам таблицы \ref{table:settings};
        значение параметра измерялось в пределах $[0.1, 1]$ с шагом $0.1$.
    }
    \label{fig:cost}
\end{figure}


\begin{table}[htp!]
\centering
\caption{Наилучшие результаты $F_{1-macro}^{PN}$ рис. \ref{fig:cost}
    для каждой задачи по каждому из прогонов.
    Помимо результата фиксируется значение параметра $Cost$, при котором
    достигается такой резульат.
}
\label{table:cost}
\begin{tabular}{ccccc}
\hline
\multicolumn{1}{c|}{\multirow{2}{*}{№}} & \multicolumn{2}{c|}{BANK}                                                   & \multicolumn{2}{c}{TCC}                                           \\ \cline{2-5}
\multicolumn{1}{c|}{}                   & \multicolumn{1}{c|}{$I_{bank_{/cost}}^{16}$} & \multicolumn{1}{c|}{$B_{bank_{/cost}}^{16}$} & \multicolumn{1}{c|}{$I_{tcc_{/cost}}^{16}$}  & $B_{tcc_{/cost}}^{16}$             \\ \hline
1                                       & $48.93_{0.6}$                                &  $46.71_{0.3}$                               &  $48.57_{0.7}$                                        & $53.50_{0.2}$                                  \\
2                                       & $50.69_{0.6}$                                &  $47.33_{0.5}$                               &  $49.48_{1.0}$                                       & $53.31_{0.3}$                                  \\
3                                       & $51.25_{0.8}$                                &  $48.00_{0.6}$                               &  $49.56_{0.5}$                                          & $53.46_{0.4}$                                  \\
4                                       & $51.52_{0.8}$                                &  $48.43_{0.5}$                               &  $49.84_{0.8}$                                         & $54.25_{0.4}$                                  \\
5                                       & $51.33_{0.7}$                                &  $48.39_{0.5}$                               &  $50.55_{1.0}$                                        & $54.26_{0.5}$                                  \\
6                                       & $\textbf{52.82}_{0.7}$                       &  $51.50_{0.7}$                               &  $52.02_{0.6}$                                        & $\textbf{55.46}_{0.3}$                         \\ \hline
\end{tabular}
\end{table}



Сравнивая результаты относительно использования разных типов обучающих коллекций,
для несбалансированных коллекций (рис. \ref{fig:cost_ttk_2016_imb},
\label{fig:cost_bank_2016_imb})
можно наблюдать резкий спад при использовании малого значения параметра ($C < 0.4$).
Что касается всех результатов,то положительный эффект от добавление признаков
на основе лексиконов сохраняется.
В большинстве случаев лучший результат достигается при подходе №6, что
можно было наблюдать и в таблицах \ref{table:results2015} и \ref{table:results2016}.
Наибольший отрыв такого подохода отмечается на рис. \ref{fig:cost_bank_2016_b}.
Подведя итог, посмотрев на работу классификационных модлелей с точки зрения
изменения настроек, можно рекомендовать использование лексиконов для составления
дополнительных признаков.

