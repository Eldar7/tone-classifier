\section{Введение}
% Актуальность, Проблема.
Огромное количество людей по всему миру пользуются микроблоговой сетью
{\it Twitter}.
Среди сообщений сети встречаются такие, авторы которых выражают мнениe и оценку
качества в различных сферах услуг.
Рост скорости появления информации ведет к разработке автоматических систем
тонального анализа.

% Постановка задачи.
Формат большинства сообщений сети представляет собой короткотекстовые
сообщения. Поэтому под задачей тональной классификации понимается оценка всего
сообщения по отношению к компаниям, в области которых это сообщение написано.
Оценка сообщения может быть положительной, нейтральной, либо негативной.
В русскоязычной сети на протяжении уже нескольких лет, наибольший интерес прикован к
{\it отзывам о банках} и {\it отзывам о телекоммуникационных компаниях}~\cite{tonalityAnalysis}.

% Описание.
В этой работе будет рассмотрен подход, основанный на использовании
словарей тональной окраски термов (лексиконов) для устранения проблемы недостатка
данных для обучения модели.
Будут рассмотрены признаки, в том числе и на основе лексиконов, которые позволяют
существенно повысить качество классификации.

% Про результаты.
Подход протестирован на прошедших соревнованиях {\it SentiRuEval-2016}
в рамках конференции {\it Dialogue-2016},
продемонстрировав 3-e место среди всех участников~\cite{dialog2016}.
Рассмотренные в данной статье улучшения позволили приблизиться к результату
победителя.
