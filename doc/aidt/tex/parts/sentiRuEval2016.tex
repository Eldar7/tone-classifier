\subsection{Участие в соревнованях SentiRuEval-2016}
    \subsubsection{Результаты}
В таблице \ref{table:tkkResult2015} приведены оценки качества работы
классификатора для тестовой коллекции {\it SentiRuEval-2016} \cite{dialog2016}
при использовании настроек предварительного тестирования.
Прогоны с такими настройками показали лучшие результаты среди других
вариаций настроек предложенного подхода (см. таблицы \ref{table:bankResult2016}-\ref{table:tkkResult2016}).

    \begin{table}[!ht]
    \centering
    \caption{Результаты прогонов соревнования (задача BANK, {\it SentiRuEval-2016})}
    \label{table:bankResult2016}
    \begin{tabular}{|c|c|c|c|c|}
    \hline
    \multirow{3}{*}{№} & \multicolumn{4}{c|}{BANK --- сообщения о банковских компаниях}                                                                                                                                                                                         \\ \cline{2-5}
                       & \multicolumn{2}{c|}{\begin{tabular}[c]{@{}c@{}}Не сбалансированная \\ коллекция (2015 год)\end{tabular}} & \multicolumn{2}{c|}{\begin{tabular}[c]{@{}c@{}}Расширенная сбалансированная \\ коллекция\end{tabular}} \\ \cline{2-5}
                       & $F_{macro}(neg, pos)$                               & $F_{micro}(neg, pos)$                              & $F_{macro}(neg, pos)$                              & $F_{micro}(neg, pos)$                             \\ \hline
    1                  & 0.384                                               & 0.4203                                             & {\bf 0.4536 (+18.1\%)}                                   & {\bf 0.4982 (+18,53\%)}                                 \\ \hline
    2                  & 0.3849                                              & 0.415                                              & {\bf 0.4672 (+20.9\%)}                                   & {\bf 0.5029 (+21,1\%)}                                 \\ \hline
    3                  & 0.3862                                              & 0.4218                                             & {\bf 0.4683 (+21.25\%)}                                  & {\bf 0.5022(+19.06\%)}                                  \\ \hline
    \end{tabular}
    \end{table}

    \begin{table}[!ht]
    \centering
    \caption{Результаты прогонов соревнования (задача TKK, {\it SentiRuEval-2016})}
    \label{table:tkkResult2016}
    \begin{tabular}{|c|c|c|c|c|}
    \hline
    \multirow{3}{*}{№} & \multicolumn{4}{c|}{TKK --- сообщения о телекоммуникационных компаниях}                                                                                                                                                                                          \\ \cline{2-5}
                       & \multicolumn{2}{c|}{\begin{tabular}[c]{@{}c@{}}Не сбалансированная \\ коллекция (2015 год)\end{tabular}} & \multicolumn{2}{c|}{\begin{tabular}[c]{@{}c@{}}Расширенная сбалансированная \\ коллекция\end{tabular}} \\ \cline{2-5}
                       & $F_{macro}(neg, pos)$                               & $F_{micro}(neg, pos)$                              & $F_{macro}(neg, pos)$                              & $F_{micro}(neg, pos)$                             \\ \hline
    1                  & 0.4849                                              & 0.641                                              & {\bf 0.5103 (+5.2\%)}                                   & {\bf 0.6509 (+1.5\%) }                                  \\ \hline
    2                  & 0.4832                                              & 0.6473                                             & {\bf 0.5231 (+8.2\%)}                                    & {\bf 0.6508 (+0.5\%)}                                   \\ \hline
    3                  & 0.5099                                              & {\bf 0.677 (+2.0\%)}                                    & {\bf 0.5286 (+3.6\%)}                                    & 0.6632                                            \\ \hline
    \end{tabular}
    \end{table}

    \subsubsection{Улучшение результатов}

После проведения соревнований, в целях повышения качества классификации,
настройки прогонов изменялись в следующих направлениях:
\begin{enumerate}
    \item {\bf Настройка параметра $C$} штрафной функции SVM классификатора.
        По умолчанию $C=1$.
        Среди множества протестированных значений \{$1, 0.75, 0.5, 0.25$\},
        наибольший прирост достигается при {\bf $C = 0.5$} (см. таблицу \ref{table:cParameter}).
    \item {\bf Добавление новых признаков:} вычисление {\it максимальных} и
        {\it минимальных} значений (с учетом нормализации на основе формул
        \ref{eq:norm1}-\ref{eq:norm2}) среди всех термов сообщения по каждому
        из лексиконов.
        Пусть $l$ -- произвольный лексикон из всего множества $L$.
        Тогда относительно рассматриваемого лексикона $l$, для каждого сообщения
        $m = \{t_i\}_{i=1}^n$ вычисляются следующие признаки:
        \begin{equation}
            f_{max_l} = \max_{i=\overline{1 \ldots n}}l(t_i), t_i \in l \nonumber
        \end{equation}

        \begin{equation}
            f_{min_l} = \min\limits_{i=\overline{1 \ldots n}}l(t_i), t_i \in l \nonumber
        \end{equation}
\end{enumerate}

    \begin{table}[ht!]
    \centering
    \caption{Влияние настройки параметра Cost (С=0.5) ({\it SentiRuEval-2016})}
    \label{table:cParameter}
    \begin{tabular}{|c|c|c|c|c|}
    \hline
    \multirow{2}{*}{№} & \multicolumn{2}{c|}{\begin{tabular}[c]{@{}c@{}}BANK\\ (Расширенная сбалансированная\\ коллекция, C=0.5)\end{tabular}} & \multicolumn{2}{c|}{\begin{tabular}[c]{@{}c@{}}TKK\\ (Расширенная сбалансированная\\ коллекция, C=0.5)\end{tabular}} \\ \cline{2-5}
                       & $F_{macro}(neg, pos)$                                     & $F_{micro}(neg, pos)$                                     & $F_{macro}(neg, pos)$                                     & $F_{micro}(neg, pos)$                                    \\ \hline
    1                  & 0.4558 (+0.4\%)                                            & 0.5037 (+1.1\%)                                            & 0.5235 (+2.5\%)                                            & 0.6612 (+1.5\%)                                           \\ \hline
    2                  & {\bf 0.4795 (+2.6\%)}                                            & {\bf 0.5167 (+2.7\%)}                               & 0.5338 (+2.0\%)                                            & 0.6610 (+1.5\%)                                          \\ \hline
    3                  & 0.4768 (+1.8\%)                                            & 0.5135(+2.2\%)                                             & {\bf 0.5452 (+3.1\%) }                                           & {\bf 0.6733 (+1.5\%) }                                          \\ \hline
    \end{tabular}
    \end{table}


    Комбинация рассмотренных выше улучшений привела к настройке {\it финальных
прогонов} (результаты представлены в таблице \ref{table:finalResults}).
Во всех прогонах использовались русскоязычные термы и хэштеги, применялись
тональные префиксы, а также учитывались все признаки. Изменения в настройках
касались только числа используемых лексиконов, а также признаков построенных
на их основе (настройки прогонов):
    \begin{enumerate}
        \item Вычисление суммы, минимума, максимума на основе лексикона №1 (см. таблицу \ref{table:createdLexicons}).
        \item Прогон №1 + признаки суммы, минимума, максимума на основе лексикона №2.
        \item Прогон №2 + признаки суммы, минимума, максимума на основе лексикона №4.
        \item Прогон №3 + признаки минимума и максимума на основе лексиконов №3.
    \end{enumerate}

    \begin{table}[ht!]
    \centering
    \caption{Результаты финального тестирования {\it SentiRuEval-2016}}
    \label{table:finalResults}
    \begin{tabular}{|c|c|c|c|c|}
    \hline
    \multirow{2}{*}{№} & \multicolumn{2}{c|}{\begin{tabular}[c]{@{}c@{}}BANK\\ (Расширенная сбалансированная\\ коллекция, C=0.5)\end{tabular}} & \multicolumn{2}{c|}{\begin{tabular}[c]{@{}c@{}}TKK\\ (Расширенная сбалансированная\\ коллекция, C=0.5)\end{tabular}} \\ \cline{2-5}
                       & $F_{macro}(neg, pos)$                                     & $F_{micro}(neg, pos)$                                     & $F_{macro}(neg, pos)$                                     & $F_{micro}(neg, pos)$                                    \\ \hline
    1                  & 0.4955                                                    & 0.5388                                                    & 0.5259                                                    & 0.6662                                                   \\ \hline
    2                  & 0.5012                                                    & 0.5379                                                    & 0.5283                                                    & 0.6720                                                   \\ \hline
    3                  & \textbf{0.5239}                                           & \textbf{0.5514}                                           & \textbf{0.5453}                                           & \textbf{0.6970}                                          \\ \hline
    4                  & 0.4818                                                    & 0.5238                                                    & 0.5356                                                    & 0.6659                                                   \\ \hline
    \end{tabular}
    \end{table}

    \subsubsection{Вывод}
Использование метаинформации на основе лексиконов стабильно повышает качество
классификации. Наибольший прирост качества достигается в случае, если классификатор
был обучен на коллекции несбалансированного типа (см. таблицу \ref{table:conclusion}\footnote{
Тип обучающей коллекции обозначается следующим образом:
$A$ --- не сбалансированная;
$B$ --- сбалансированная;
$C$ --- расширенная.})\footnote{
В таблице рассматривается прирост качества 3-его прогона по отношению к 1-ому (согласно
таблицам \ref{table:bankResult2015}-\ref{table:tkkResult2015}, и
\ref{table:bankResult2016}-\ref{table:tkkResult2016}).
В скобках указывается общий прирост качества с учетом балансировки.
}.

\begin{table}[ht!]
\begin{adjustwidth}{-1.1cm}{}
\centering
\caption{Рост качества при использовании признаков на основе лексиконов в зависимости от типа обучающей коллекции}
\label{table:conclusion}
\begin{tabular}{|c|c|c|c|c|c|}
\hline
\multicolumn{2}{|c|}{\begin{tabular}[c]{@{}c@{}c@{}}Параметры \\ обучающей \\ коллекции\end{tabular}}                                                        & \multicolumn{2}{c|}{BANK}                                                                                            & \multicolumn{2}{c|}{TKK}                                                                                          \\ \hline
Год                      & Тип         & $F_{macro}(neg, pos)$                                    & $F_{micro}(neg, pos)$                                     & $F_{macro}(neg, pos)$                                   & $F_{micro}(neg, pos)$                                   \\ \hline
\multirow{2}{*}{2015} & $A$                                                    & +12.57\%                                                  & +9.8\%                                                     & +6.8\%                                                   & +3.9\%                                                   \\ \cline{2-6}
                                  & $B$                                                       & \begin{tabular}[c]{@{}c@{}}+3.3\%\\ (+19.0\%)\end{tabular} & \begin{tabular}[c]{@{}c@{}}+4.6\%\\ (+19.8\%)\end{tabular}  & \begin{tabular}[c]{@{}c@{}}+4\%\\ (+3.4\%)\end{tabular}   & \begin{tabular}[c]{@{}c@{}}+2.7\%\\ (+1.9\%)\end{tabular} \\ \hline
\multirow{3}{*}{2016} & $A$                                                    & ---                                                      & ---                                                       & +5.1\%                                                   & +4.6\%                                                   \\ \cline{2-6}
                                  & $B$                                                       & +0.5\%                                                    & +0.03\%                                                    & ---                                                     & ---                                                     \\ \cline{2-6}
& $C$ & \begin{tabular}[c]{@{}c@{}}+4.6\%\\ (+21.95)\end{tabular} & \begin{tabular}[c]{@{}c@{}}+1.9\%\\ (+19.48\%)\end{tabular} & \begin{tabular}[c]{@{}c@{}}+4.1\%\\ (+9.0\%)\end{tabular} & \begin{tabular}[c]{@{}c@{}}+1.8\%\\ (+3.4\%)\end{tabular} \\ \hline
\end{tabular}
\end{adjustwidth}
\end{table}
В таблице \ref{table:conclusion},
значения $(+21.95)$, и $(+19.48)$ последней строки указывают на общий прирост
качества с учетом расширенной балансировки по отношению к обычной балансировке
(тестирование в этих случаях на несбалансированной коллекции не проводилось,
ввиду результатов п. \ref{sec:test2015}, таблица \ref{table:bankResult2015}).
Увеличение числа признаков по каждому из лексиконов позволяет повысить показания
таблицы \ref{table:conclusion}.
В совокупности с использованием сбалансированной обучающей коллекции и настройкой
классификатора, в рамках этой работы были получены максимальные результаты
(см. таблицу \ref{table:finalResults}, прогон №3).
В таблице \ref{table:totalImprovement} представлен прирост качества в результате
использования расширенной сбалансированной коллекции в сочетании с признаками
на основе лексиконов. Наибольший прирост достигается для задачи {\it BANK}.

\begin{table}[!ht]
\centering
\caption{Прирост качества для каждой из задач (сравнение лучшего финального результата с результатами прогона №1, {\it SentiRuEval-2016})}
\label{table:totalImprovement}
\begin{tabular}{|c|c|c|}
\hline
Прирост качества & BANK   & TKK    \\ \hline
Общий            & +36.4\% & +12.4\% \\ \hline
\end{tabular}
\end{table}

Результаты всех участников соревнований представлены в таблице \ref{table:allResults}.
Про некоторых из участников известны настройки прогонов, которые они использовали
для получения соответствующих оценок.
Описание настроек рассматривается в таблице \ref{table:usersSettings}.\footnote{
Таблица с результатами прогонов всех участников соревнований, а также настройками прогонов:
\url{https://docs.google.com/spreadsheets/d/1rCaklClawfnnSnyk4q8CW4zWuO3P38DSrLw_f2wyyjg/edit\#gid=0}
}
Участник под номером 1 соответствует текущей работе.
Соответственно, в первой строке таблицы \ref{table:allResults} рассматривается
максимально полученный в рамках этой работы результат.
Вторая строка таблицы содержит результат, полученный во время проведения соревнований.\footnote{
Разница в результатах, записанная в формате процентов, вычисляется как отношение
результата победителя соревнований к соответвующему резульату участика №1.
}

%В таблице указаны наилучшие результаты для каждой задачи каждого из участников
%по каждой из задачи относительно показания $F_{macro}(neg, pos)$. Жирным шрифтом
%выделен участник, который достиг максимальных результатов по метрикам $F_{macro}(neg, pos)$.

\begin{table}[ht!]
\centering
\caption{
В таблице указаны наилучшие результаты для каждой задачи каждого из участников
по каждой из задачи относительно показания $F_{macro}(neg, pos)$. Жирным шрифтом
выделены результаты участника, который достиг максимальных результатов по
метрикам $F_{macro}(neg, pos)$.
}
\label{table:allResults}
\begin{tabular}{|c|c|c|c|c|}
\hline
                                                                   & \multicolumn{2}{c|}{BANK}                                                        & \multicolumn{2}{c|}{TKK}                                                                                \\ \cline{2-5}
\multirow{-2}{*}{\begin{tabular}[c]{@{}c@{}}Номер\\ участника\end{tabular}}                                                & $F_{macro}(neg, pos)$                  & $F_{micro}(neg, pos)$                   & $F_{macro}(neg, pos)$                                          & $F_{micro}(neg, pos)$                  \\ \hline
\rowcolor[HTML]{FFFFFF}
\cellcolor[HTML]{FFFFFF}{\color[HTML]{333333} }                    & {\color[HTML]{333333} 0.5239 (-5.3\%)} & {\color[HTML]{333333} 0.5514 (-6.6\%)}  & {\color[HTML]{333333} 0.5453 (-2.5\%)}                         & {\color[HTML]{333333} 0.6970 (+5.7\%)} \\ \cline{2-5}
\rowcolor[HTML]{FFFFFF}
\multirow{-2}{*}{\cellcolor[HTML]{FFFFFF}{\color[HTML]{333333} 1}} & 0.4683 (-17.8\%)                       & {\color[HTML]{333333} 0.5022 (-17.1\%)} & {\color[HTML]{333333} 0.5286 (-5.8\%)}                         & 0.6632 (+0.9\%)                        \\ \hline
\textbf{2}                                                         & \textbf{0.5517}                        & {\color[HTML]{333333} \textbf{0.5881}}  & \cellcolor[HTML]{FFFFFF}{\color[HTML]{333333} \textbf{0.5594}} & \textbf{0.6569}                        \\ \hline
3                                                                  & 0.3423                                 & 0.3524                                  & 0.3994                                                         & 0.3994                                 \\ \hline
4                                                                  & 0.3730                                 & 0.3967                                  & 0.4955                                                         & 0.6252                                 \\ \hline
5                                                                  & 0.3859                                 & 0.4640                                  & 0.3499                                                         & 0.4044                                 \\ \hline
6                                                                  & 0.2398                                 & 0.3127                                  & 0.3545                                                         & 0.5263                                 \\ \hline
7                                                                  & 0.471                                  & 0.5128                                  & 0.4842                                                         & 0.6374                                 \\ \hline
8                                                                  & 0.4492                                 & 0.4705                                  & 0.4871                                                         & 0.5745                                 \\ \hline
9                                                                  & 0.5195                                 & 0.5595                                  & 0.5489                                                         & 0.6822                                 \\ \hline
10                                                                 & 0.4659                                 & 0.5053                                  & 0.5055                                                         & 0.6254                                 \\ \hline
\end{tabular}
\end{table}

\begin{table}[ht!]
\centering
\caption{Настройки прогонов участников соревнований}
\label{table:usersSettings}
\begin{tabular}{|c|p{13cm}|}
\hline
\begin{tabular}[c]{@{}c@{}}Номер\\ участника\end{tabular} & \multicolumn{1}{c|}{Настройки прогона}                                                                                                                                                                                                                                                     \\ \hline
1               & Текущая работа (опубликована в формате статьи конференции {\it Диалог-2016} \cite{myArticle}). Описание подхода рассмотрено как в статье \cite{myArticle}, так и в п. \ref{sec:buildingApproachDescription}.  \\ \hline
2               & рекуррентная нейронная сетка ({\it LSTM}). В качестве признаков {\it WORD2VEC}, обученный на внешней коллекции. (Посты и комментарии из социальных сетей)                                                                                                                              \\ \hline
4               & Словарные признаки + признаки мета-классификаторов (логистическая регрессия, ридж-регрессия, классификатор на основе градиентного бустинга и классификатор на основе нейронной сети) и линейный {\it SVM} в качестве классификатора.                                                     \\ \hline
8               & поиск эмоциональных слов по словарю (200 тыс. словоформ), правила их  комбинирования на основе синтаксического анализа; применение онтологических правил, характерных для данной предметной области                                                                                        \\ \hline
9               & {\it SVM}: униграммы, биграммы, словарь РуСентиЛекс, учет частей речи, многозначных слов (автоматический словарь коннотаций по новостям для TKK задачи)                                                                                                                                          \\ \hline
10              & {\it SVM}, в качестве признаков использовались униграммы, подвергшиеся преобразованиям ({\it не + слово = один признак}, множественные повторения символов заменяются двукратным; ссылки, ответы, даты, числа – заменяются паттернами и другие преобразования). Подключение словаря РуСентиЛекс \\ \hline
\end{tabular}
\end{table}


