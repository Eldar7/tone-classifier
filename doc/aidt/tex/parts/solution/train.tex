\subsection{Коллекции данных для обучения}
    % Здесь рассказываем про коллекции, которые использовались несбалансированные для обучения коллекции
    Для обучения классификатора предполагается использовать соответствующие
    коллекции данных соревнований {\it SentiRuEval}.

    [Вставить таблицу с обучающими коллекциями!]

    Поскольку в предоставляемых
    данных число тональных сообщений существенно уступает объему класса
    нейтральных сообщений, то дополнительно планируется создать {\it сбалансированную
    обучающую коллекцию}.
    В работе \cite{diploma2015}, применительно к классификаторам {\it
    Наивного Байеса} и {\it SVM}, отмечается существенный прирост качества при
    использовании коллекций сбалансированного типа.

    % Про балансировку коллекций в том числе.
    Для решения подобной задачи воспользуемся готовым общедоступным корпусом Ю.~Рубцовой
    \cite{rubtsovaCollection}, в
    котором каждое сообщение автоматически распределено в одну из тональных групп:
    {\it positive} и {\it negative}.
    Объем каждого класса такой коллекции составляет {\it $\approx$ 110 тыс.
    сообщений}

    % (Как производить балансировку)
    Для построения сбалансированной коллекции требуется существенно меньшее
    число сообщений чем предлагается в тональном корпусе.
    В связи с этим, выберем небольшой процент наиболее эмоциональных сообщений:
    \begin{enumerate}
        \item Построим лексикон $l$ на основе корпуса Ю.~Рубцовой для определения
            списка наиболее эмоциональных термов.
        \item Сообщение $m$ будем считать {\it наиболее эмоциональным},
            если для него выполнено следующее условие:
            \begin{gather}
                \max\limits_{i=1..N} |l(t_i)| > B
            \end{gather}
            Где $B$ -- величина порогового значения; \hspace{0.5pt}
            $t_i$ -- термы сообщения $m$; \hspace{0.5pt}
            $N$ -- общее количество термов в сообщении $m$;
    \end{enumerate}
