\section*{Вывод}
% Описать результат сравнения bag of words с tf-idf.
%
Использование метаинформации на основе лексиконов стабильно повышает качество
классификации. Наибольший прирост качества достигается в случае, если классификатор
был обучен на коллекции несбалансированного типа (см. таблицу \ref{table:conclusion}\footnote{
Тип обучающей коллекции обозначается следующим образом:
$A$ --- не сбалансированная;
$B$ --- сбалансированная;
$C$ --- расширенная.})\footnote{
В таблице рассматривается прирост качества 3-его прогона по отношению к 1-ому (согласно
таблицам \ref{table:bankResult2015}-\ref{table:tkkResult2015}, и
\ref{table:bankResult2016}-\ref{table:tkkResult2016}).
В скобках указывается общий прирост качества с учетом балансировки.
}.

\begin{table}[ht!]
\begin{adjustwidth}{-1.1cm}{}
\centering
\caption{Рост качества при использовании признаков на основе лексиконов в зависимости от типа обучающей коллекции}
\label{table:conclusion}
\begin{tabular}{|c|c|c|c|c|c|}
\hline
\multicolumn{2}{|c|}{\begin{tabular}[c]{@{}c@{}c@{}}Параметры \\ обучающей \\ коллекции\end{tabular}}                                                        & \multicolumn{2}{c|}{BANK}                                                                                            & \multicolumn{2}{c|}{TKK}                                                                                          \\ \hline
Год                      & Тип         & $F_{macro}(neg, pos)$                                    & $F_{micro}(neg, pos)$                                     & $F_{macro}(neg, pos)$                                   & $F_{micro}(neg, pos)$                                   \\ \hline
\multirow{2}{*}{2015} & $A$                                                    & +12.57\%                                                  & +9.8\%                                                     & +6.8\%                                                   & +3.9\%                                                   \\ \cline{2-6}
                                  & $B$                                                       & \begin{tabular}[c]{@{}c@{}}+3.3\%\\ (+19.0\%)\end{tabular} & \begin{tabular}[c]{@{}c@{}}+4.6\%\\ (+19.8\%)\end{tabular}  & \begin{tabular}[c]{@{}c@{}}+4\%\\ (+3.4\%)\end{tabular}   & \begin{tabular}[c]{@{}c@{}}+2.7\%\\ (+1.9\%)\end{tabular} \\ \hline
\multirow{3}{*}{2016} & $A$                                                    & ---                                                      & ---                                                       & +5.1\%                                                   & +4.6\%                                                   \\ \cline{2-6}
                                  & $B$                                                       & +0.5\%                                                    & +0.03\%                                                    & ---                                                     & ---                                                     \\ \cline{2-6}
& $C$ & \begin{tabular}[c]{@{}c@{}}+4.6\%\\ (+21.95)\end{tabular} & \begin{tabular}[c]{@{}c@{}}+1.9\%\\ (+19.48\%)\end{tabular} & \begin{tabular}[c]{@{}c@{}}+4.1\%\\ (+9.0\%)\end{tabular} & \begin{tabular}[c]{@{}c@{}}+1.8\%\\ (+3.4\%)\end{tabular} \\ \hline
\end{tabular}
\end{adjustwidth}
\end{table}
В таблице \ref{table:conclusion},
значения $(+21.95)$, и $(+19.48)$ последней строки указывают на общий прирост
качества с учетом расширенной балансировки по отношению к обычной балансировке
(тестирование в этих случаях на несбалансированной коллекции не проводилось,
ввиду результатов п. \ref{sec:test2015}, таблица \ref{table:bankResult2015}).
Увеличение числа признаков по каждому из лексиконов позволяет повысить показания
таблицы \ref{table:conclusion}.
В совокупности с использованием сбалансированной обучающей коллекции и настройкой
классификатора, в рамках этой работы были получены максимальные результаты
(см. таблицу \ref{table:finalResults}, прогон №3).
В таблице \ref{table:totalImprovement} представлен прирост качества в результате
использования расширенной сбалансированной коллекции в сочетании с признаками
на основе лексиконов. Наибольший прирост достигается для задачи {\it BANK}.

\begin{table}[!ht]
\centering
\caption{Прирост качества для каждой из задач (сравнение лучшего финального результата с результатами прогона №1, {\it SentiRuEval-2016})}
\label{table:totalImprovement}
\begin{tabular}{|c|c|c|}
\hline
Прирост качества & BANK   & TKK    \\ \hline
Общий            & +36.4\% & +12.4\% \\ \hline
\end{tabular}
\end{table}
