\subsection{Оценка качества классификационной модели}
    % Посмотреть обзоры из обзора соревнований.
    \subsubsection{Точность и полнота}
    Чтобы произвести оценку качества работы классификатора на некотором наборе
    сообщений, необходимо чтобы для этого набора существовали эталонные
    значения. Применительно к задаче тональной классификации, под значениями
    понимается класс, к которому необходимо отнести соответствующее сообщение.

    Таким образом, для каждого сообщения ответ может быть получен как со стороны
    классификатора, так и группой экспертов. Все возможные случаи ответов
    для фиксированного класса $A$ удобнее всего представить в таблице
    \ref{table:contingent}. Такое представление носит название
    {\it таблицы контингентности}.

    \begin{table}[H]
        \centering
        \caption{Таблица контингентности для класса $A$}
        \label{table:contingent}
        \begin{tabular}{|c|c|c|c|}
            \hline
            \multicolumn{2}{|c|}{\multirow{2}{*}{Принадлежность сообщений к классу $A$}} &     \multicolumn{2}{c|}{эксперты}                     \\ \cline{3-4}
            \multicolumn{2}{|c|}{}                                              &   положительная             & отрицательная                           \\ \hline
            \multirow{2}{*}{классификатор}          & положительная             & {\cellcolor[HTML]{9AFF99} $TP$} & {\cellcolor[HTML]{FFCCC9} $FP$}     \\ \cline{2-4}
                                                    & отрицательная             & {\cellcolor[HTML]{FFCCC9} $FN$} & {\cellcolor[HTML]{9AFF99} $TN$}     \\ \hline
        \end{tabular}
     \end{table}

     На основе таблицы \ref{table:contingent} можно рассчитать следующие
     характеристики качества работы классификатора для соответствующего класса:
    \begin{itemize}
        \item {\bf Полнота} --- число найденных сообщений, которые
            действительно принадлежат соответствующему классу относительно всех
            сообщений соответствующего класса:

            \begin{equation}
                \label{eq:recall}
                R_A = \dfrac{TP}{TP + FN} \hspace{20pt}
            \end{equation}

        \item {\bf Точность} --- количество сообщений, которое
            классификатор правильно отнес к соответствующему классу по отношению
            ко всему объему сообщений определенных системой в этот класс:

            \begin{equation}
                \label{eq:precision}
                P_A = \dfrac{TP}{TP + FP} \hspace{20pt}
            \end{equation}
    \end{itemize}

        На практике возникает необходимость в метрике, которая бы позволяла одновременно
    обе характеристики: точность и полноту. Для этого предусмотрена $F-\text{мера}$, которая
    в общем случае вычисляется по формуле:
    \begin{equation}
        \label{eq:fmeasurecommon}
        F(\beta) = \dfrac{(1+\beta^2) P \cdot R}{\beta^2 \cdot P + R}
    \end{equation}

    В случае, если $\beta = 1$, то формула \ref{eq:fmeasurecommon} преобразуется
    к гармоническому среднему:
    \begin{equation}
        \label{eq:fmeasure}
        F_1 = \dfrac{2 \cdot P R}{P + R}
    \end{equation}

    \subsubsection{$F_1-micro$ и $F_1-macro$ меры качества}
    Рассмотрим случай, когда необходимо рассмотреть параметры качества работы
    классификатора относительно нескольких классов одновременно.
    Пусть имеется $K$ классов, относительно которых будут вычисляться параметры
    полноты, точности, и $F-\text{меры}$.
    Относительно каждого класса можно составить таблицы контингентности,
    аналогичные таблице \ref{table:contingent}.

    Одним из методов вычисления среднего значения параметров точности и полноты
    является {\it микроусреднение} \cite{micromacromeasures}:
    \begin{itemize}
        \item {\it Микроусреднением полноты} --- является обобщением формулы
            \ref{eq:recall} на случай нескольких классов:
            \begin{equation}
                R_{micro_{(1, \ldots, N)}} = \dfrac{\sum\limits_{i=1}^N TP_i}{\sum\limits_{i=1}^N TP_i + \sum\limits_{i=1}^N FP_i}
            \end{equation}
        \item {\it Микроусреднением точности} --- называется обобщением формулы
            \ref{eq:precision} на случай нескольких классов:
            \begin{equation}
                P_{micro_{(1, \ldots, N)}} = \dfrac{\sum\limits_{i=1}^N TP_i}{\sum\limits_{i=1}^N TP_i + \sum\limits_{i=1}^N FN_i}
            \end{equation}
    \end{itemize}

%    далее, подставляя параметры $P_{micro}$, $P_{macro}$ в формулу \ref{eq:fmeasure},
%    получим {\it микроусредненную меру $F_1$}:

    Другой метод усреднения значений полноты и точности называется {\it макроусреднением} \cite{micromacromeasures}.
    Параметры на основе такого подхода, вычисляются следующим образом:
    \begin{itemize}
        \item {\it Макроусреднение полноты} --- вычисление среднего значения параметров
            полноты каждого из классов:
            \begin{equation}
                R_{macro_{(1, \ldots, N)}} = \dfrac{\sum\limits_{i=1}^N R_i}{N}
            \end{equation}
        \item {\it макроусреднение точности} --- вычисление среднего значения параметров
            точности каждого из классов:
            \begin{equation}
                P_{macro_{(1, \ldots, N)}} = \dfrac{\sum\limits_{i=1}^N P_i}{N}
            \end{equation}
    \end{itemize}

    Таким образом, на основе каждого из метода усреднений может быть вычислена
    $F_1$ мера:
    \begin{equation}
        \label{eq:fmacro12}
        F_{1_{macro_{(1, \ldots, N)}}} = \dfrac{2 \cdot P_{macro_{(1,\ldots, N)}} R_{macro_{(1,\ldots, N)}} }{P_{macro_{(1,\ldots, N)}} + R_{macro_{(1,\ldots, N)}}}
    \end{equation}
    \begin{equation}
        \label{eq:fmicro12}
        F_{1_{micro_{(1, \ldots, N)}}} = \dfrac{2 \cdot P_{micro_{(1,\ldots, N)}} R_{micro_{(1,\ldots, N)}} }{P_{micro_{(1,\ldots, N)}} + R_{micro_{(1,\ldots, N)}}}
    \end{equation}

    % про ф-меру сюда же

    Мaкроусреднение придает одинаковый вес каждому из усредняемых классов, в то
    время как при микроусреднении вес учитывается на основе числа документов в
    классе. Поскольку $F_{macro}$ мера игнорирует параметр $TN$, то смещение
    среднего значения будет производиться в сторону того класса, для которого классификатор сработал
    лучше (большее значение $TP$); в тоже время, при использовании $F_{micro}$,
    смещение будет произведено в сторону наибольшего класса. \cite{micromacromeasures}

    Таким образом, результаты полученные на основе микроусреднения являются мерой
    эффективности на коллекциях большого объема. Для получения аналогичного эффекта
    на коллекциях малого объема, необходимо использовать макроусреднение. \cite{micromacromeasuresdifferences}

