\newpage
\part*{\large \centering ЗАКЛЮЧЕНИЕ}
\addcontentsline{toc}{part}{ЗАКЛЮЧЕНИЕ}

В данной работе исследован метод машинного обучения в качестве подхода к решению
задачи тональной классификации.
Для построения боллее точной классификационной модели применялись признаки
на основе словарей оценочной лексики.
Исследована возможность австоматического порождения таких словарей на основе
сообщений сети {\it Twitter}.

Качество работы классификатора было протестировано в системах анализа тональности
русского языка {\it SentiRuEval-2015} и {\it SentiRuEval-2016}.
На последней демонстрируется довольно высокий результат (3-е место)
относительно остальных участников тестирования.
После проведения настройки классификатора и введения дополнительных признаков на
основе словарей оценочной лексики, рассматриваемый в статье подход можно считать
одним из наиболее успешных на сегодняшний день для проведения тональной
классификации сообщений различных областей русскоязычной сети {\it Twitter}.
