\clearpage
\section{Технико-экономическое обоснование}
    %Разработка программного обеспечения~---~достаточно трудоемкий и длительный
    %процесс, требующий выполнения большого числа разнообразных операций.
    %Организация и планирование процесса разработки программного продукта или
    %программного комплекса при традиционном методе планирования предусматривает
    %выполнение следующих работ:
    %\begin{itemize}
    %    \item формирование состава выполняемых работ и группировка их по стадиям разработки;
    %    \item расчет трудоемкости выполнения работ;
    %    \item установление профессионального состава и расчет количества исполнителей;
    %    \item определение продолжительности выполнения отдельных этапов разработки;
    %    \item построение календарного графика выполнения разработки;
    %    \item контроль выполнения календарного графика.
    %\end{itemize}

    %Далее приведен перечень и состав работ при разработке программного средства
    %для автоматического установления связей между сообщениями твиттера и новостными
    %статьями.
    %Отметим, что процесс разработки программного продукта характеризуется совместной
    %работой разработчиков постановки задач и разработчиков программного обеспечения.

    %Укрупненный состав работ по стадиям разработки программного продукта:
    %\begin{enumerate}
    %    \item Техническое задание;
    %    \item Эскизный проект;
    %        \begin{itemize}
    %            \item Предварительная разработка структуры входных и выходных данных,
    %            \item Разработка общего описания алгоритмов реализации решения задач,
    %            \item Разработка пояснительной записки,
    %            \item Консультации разработчиков постановки задач,
    %            \item Согласование и утверждение эскизного проекта;
    %        \end{itemize}
    %    \item Технический проект;
    %    \item Рабочий проект;
    %    \item Внедрение.
    %\end{enumerate}

    %Трудоемкость разработки программной продукции зависит от ряда факторов,
    %основными из которых являются следующие: степень новизны разрабатываемого
    %программного комплекса, сложность алгоритма его функционирования, объем
    %используемой информации, вид ее представления и способ обработки, а также
    %уровень используемого алгоритмического языка программирования.
    %Чем выше уровень языка, тем трудоемкость меньше.

    По степени новизны разрабатываемый проект относится к \textit{группе новизны A}
    – разработка программных комплексов, требующих использования принципиально
    новых методов их создания, проведения НИР и т.п.

    По степени сложности алгоритма функционирования проект относится к
    \textit{2 группе сложности} - программная продукция, реализующая
    учетно-статистические алгоритмы.

    По виду представления исходной информации и способа ее контроля программный
    продукт относится к \textit{группе 12} - исходная информация представлена в
    форме документов, имеющих различный формат и структуру и \textit{группе 22}
    - требуется печать документов одинаковой формы и содержания, вывод массивов
    данных на машинные носители.

    \subsection{Трудоемкость разработки программной продукции}
    \label{subsec:trud}
        Трудоемкость разработки программной продукции~($\tau_{PP}$) может быть
        определена как сумма величин трудоемкости выполнения отдельных стадий
        разработки программного продукта из выражения:
        $$\tau_{PP} = \tau_{TZ} + \tau_{EP} + \tau_{TP} + \tau_{RP} + \tau_{V},$$
        где $\tau_{TZ}$~---~трудоемкость разработки технического задания на создание программного продукта;
        $\tau_{EP}$~---~трудоемкость разработки эскизного проекта программного продукта;
        $\tau_{TP}$~---~трудоемкость разработки технического проекта программного продукта;
        $\tau_{RP}$~---~трудоемкость разработки рабочего проекта программного продукта;
        $\tau_{V}$~---~трудоемкость внедрения разработанного программного продукта.

        \subsubsection{Трудоемкость разработки технического задания}
            Расчёт трудоёмкости разработки технического задания~($\tau_{PP}$)~[чел.-дни] производится по формуле:
            $$\tau_{TZ} = T^Z_{RZ} + T^Z_{RP},$$
            где $T^Z_{RZ}$~---~затраты времени разработчика постановки задачи на разработку ТЗ,~[чел.-дни];
            $T^Z_{RP}$~---~затраты времени разработчика программного обеспечения на разработку ТЗ,~[чел.-дни].
            Их значения рассчитываются по формулам:
            \begin{equation}
                T^Z_{RZ} = t_Z \cdot  K^Z_{RZ}   \nonumber
            \end{equation}
            \begin{equation}
                T^Z_{RP} = t_Z \cdot  K^Z_{RP}   \nonumber
            \end{equation}
            где $t_Z$~--~норма времени на разработку ТЗ на программный продукт
            (зависит от функционального назначения и степени новизны разрабатываемого
            программного продукта),~[чел.-дни].
            В нашем случае по таблице получаем значение~(группа новизны – А, функциональное назначение – технико-экономическое планирование):
            $$t_Z = 79.$$
            $K^Z_{RZ}$~---~коэффициент, учитывающий удельный вес трудоемкости работ,
            выполняемых разработчиком постановки задачи на стадии ТЗ.
            В нашем случае~(совместная разработка с разработчиком ПО):
            $$K^Z_{RZ} = 0.65.$$
            $K^Z_{RP}$~---~коэффициент, учитывающий удельный вес трудоемкости работ,
            выполняемых разработчиком программного обеспечения на стадии ТЗ.
            В нашем случае~(совместная разработка с разработчиком постановки задач):
            $$K^Z_{RP} = 0.35.$$
            Тогда:
            $$\tau_{TZ} = 79 \cdot ~(0.35 + 0.65) = 79.$$

        \subsubsection{Трудоемкость разработки эскизного проекта}
            Расчёт трудоёмкости разработки эскизного проекта~($\tau_{EP}$)~[чел.-дни] производится по формуле:
            $$\tau_{EP} = T^E_{RZ} + T^E_{RP},$$
            где $T^E_{RZ}$~---~затраты времени разработчика постановки задачи на разработку эскизного проекта~(ЭП),~[чел.-дни];
            $T^E_{RP}$~---~затраты времени разработчика программного обеспечения на разработку ЭП,~[чел.-дни].
            Их значения рассчитываются по формулам:
            $$T^E_{RZ} = t_E \cdot  K^E_{RZ},$$
            $$T^E_{RP} = t_E \cdot  K^E_{RP},$$
            где $t_E$~--~норма времени на разработку ЭП на программный продукт~(зависит от функционального назначения и степени новизны разрабатываемого программного продукта),~[чел.-дни].
            В нашем случае по таблице получаем значение~(группа новизны – А, функциональное назначение – технико-экономическое планирование):
            $$t_E = 175.$$
            $K^E_{RZ}$~---~коэффициент, учитывающий удельный вес трудоемкости работ, выполняемых разработчиком постановки задачи на стадии ЭП.
            В нашем случае~(совместная разработка с разработчиком ПО):
            $$K^E_{RZ} = 0.7.$$
            $K^E_{RP}$~---~коэффициент, учитывающий удельный вес трудоемкости работ, выполняемых разработчиком программного обеспечения на стадии ТЗ.
            В нашем случае~(совместная разработка с разработчиком постановки задач):
            $$K^E_{RP} = 0.3.$$
            Тогда:
            $$\tau_{EP} = 175 \cdot ~(0.3 + 0.7) = 175.$$

        \subsubsection{Трудоемкость разработки технического проекта}
            Трудоёмкость разработки технического проекта~($\tau_{TP}$)~[чел.-дни]
            зависит от функционального назначения программного продукта, количества
            разновидностей форм входной и выходной информации и определяется по формуле:
            $$\tau_{TP} = (t^T_{RZ} + t^T_{RP})\cdot K_V\cdot K_R,$$
            где $t^T_{RZ}$~---~норма времени, затрачиваемого на разработку технического проекта~(ТП) разработчиком постановки задач,~[чел.-дни];
            $t^T_{RP}$~---~норма времени, затрачиваемого на разработку ТП разработчиком ПО,~[чел.-дни].
            По таблице принимаем~(функциональное назначение~---~технико-экономическое планирование,
            количество разновидностей форм входной информации~---~1~(результаты качества работы классификатора),
            количество разновидностей форм выходной информации~---~2~(тональная оценка сообщений, оценка работы классификатора)):
            $$t^T_{RZ} = 38, \hspace{1cm} t^T_{RP} = 9$$
            $K_R$~---~коэффициент учета режима обработки информации. По таблице принимаем~(группа новизны~---~А, режим обработки информации~---~реальный масштаб времени):
            $$K_R = 1.45$$
            $K_V$~---~коэффициент учета вида используемой информации, определяется по формуле:
            $$K_V = \dfrac {K_P\cdot n_P + K_{NS}\cdot n_{NS} + K_B\cdot n_B} {n_P + n_{NS} + n_B },$$
            где $K_P$~---~коэффициент учета вида используемой информации для переменной информации;
            $K_{NS}$~---~коэффициент учета вида используемой информации для нормативно-справочной информации;
            $K_B$~---~коэффициент учета вида используемой информации для баз данных;
            $n_P$~---~количество наборов данных переменной информации;
            $n_{NS}$~---~количество наборов данных нормативно-справочной информации;
            $n_B$~---~количество баз данных.
            Коэффициенты находим по таблице~(группа новизны - А):
            $$K_P=1.70, \hspace{1cm} K_{NS}=1.45, \hspace{1cm} K_B=4.37.$$
            Количество наборов данных, используемых в рамках задачи:
            $$n_P=3, \hspace{1cm} n_{NS}=0, \hspace{1cm} n_B=1.$$
            Находим значение $K_V$:
            $$K_V = \dfrac{1.70\cdot 3+1.45\cdot 0+4.37\cdot 1}{3+0+1}=2.3675.$$
            Тогда:
            $$\tau_{TP} = (38+9)\cdot 2.3675\cdot 1.67 = 185.2$$

        \subsubsection{Трудоемкость разработки рабочего проекта}
            Трудоёмкость разработки рабочего проекта~($\tau_{RP}$)~[чел.-дни]
            зависит от функционального назначения программного продукта, количества
            разновидностей форм входной и выходной информации, сложности алгоритма
            функционирования, сложности контроля информации, степени использования
            готовых программных модулей, уровня алгоритмического языка
            программирования и определяется по формуле:
            $$\tau_{RP} = (t^R_{RZ} + t^R_{RP})\cdot K_K\cdot K_R\cdot K_Y \cdot K_Z\cdot K_{IA},$$
            где $t^R_{RZ}$~---~норма времени, затраченного на разработку рабочего проекта на алгоритмическом языке высокого уровня разработчиком постановки задач,~[чел.-дни].
            $t^R_{RP}$~---~норма времени, затраченного на разработку рабочего проекта на алгоритмическом языке высокого уровня разработчиком ПО,~[чел.-дни].
            По таблице принимаем~(функциональное назначение~---~технико-экономическое планирование,
            количество разновидностей форм входной информации~---~1,
            количество разновидностей форм выходной информации~---~2:
            $$t^R_{RZ} = 11,\hspace{1cm} t^R_{RP} = 68.$$
            $K_K$~---~коэффициент учета сложности контроля информации.
            По таблице принимаем~(степень сложности контроля входной информации~---~11, степень сложности контроля выходной информации~---~22):
            $$K_K = 1.07.$$
            $K_R$~---~коэффициент учета режима обработки информации.
            По таблице принимаем~(группа новизны~---~А, режим обработки информации~---~реальный масштаб времени):
            $$K_R = 1.75.$$
            $K_Y$~---~коэффициент учета уровня используемого алгоритмического языка программирования. По таблице принимаем значение~(интерпретаторы, языковые описатели):
            $$K_Y = 0.8.$$
            $K_Z$~---~коэффициент учета степени использования готовых программных модулей. По таблице принимаем~(использование готовых программных модулей составляет около 50%%):
            $$K_Z = 0.6.$$
            $K_{IA}$~---~коэффициент учета вида используемой информации и сложности алгоритма программного продукта, его значение определяется по формуле:
            $$K_IA = \dfrac {K'_P\cdot n_P + K'_{NS}\cdot n_{NS} + K'_B\cdot n_B} {n_P + n_{NS} + n_B },$$
            где $K'_P$~---~коэффициент учета сложности алгоритма ПП и вида используемой информации для переменной информации;
            $K'_{NS}$~---~коэффициент учета сложности алгоритма ПП и вида используемой информации для нормативно-справочной информации;
            $K'_B$~---~коэффициент учета сложности алгоритма ПП и вида используемой информации для баз данных.
            $n_P$~---~количество наборов данных переменной информации;
            $n_{NS}$~---~количество наборов данных нормативно-справочной информации;
            $n_B$~---~количество баз данных.
            Коэффициенты находим по таблице~(группа новизны - А):
            $$K'_P=2.02, \hspace{1cm} K'_{NS}=1.21, \hspace{1cm} K'_B=1.05.$$
            Количество наборов данных, используемых в рамках задачи:
            $$n_P=3, \hspace{1cm} n_{NS}=0,\hspace{1cm} n_B=1.$$
            Находим значение $K_{IA}$:
            $$K_{IA} = \dfrac{2.02\cdot 3+1.21\cdot 0+1.05\cdot 1}{3+0+1}=1.7775.$$
            Тогда:
            $$\tau_{RP} = (11+68)\cdot 1.00\cdot 1.75\cdot 0.8\cdot 0.6\cdot 1.7775 = 126.2$$


        \subsubsection{Трудоемкость выполнения стадии <<Внедрение>>}
            Расчёт трудоёмкости разработки технического проекта~($\tau_{V}$)~[чел.-дни] производится по формуле:
            $$\tau_{V} = (t^V_{RZ} + t^V_{RP})\cdot K_K\cdot K_R\cdot K_Z,$$
            где $t^V_{RZ}$~---~норма времени, затрачиваемого разработчиком постановки задач на выполнение процедур внедрения программного продукта,~[чел.-дни];
            $t^V_{RP}$~---~норма времени, затрачиваемого разработчиком программного обеспечения на выполнение процедур внедрения программного продукта,~[чел.-дни].
            По таблице принимаем~(функциональное назначение~---~технико-экономическое планирование,
            количество разновидностей форм входной информации~---~1,
            количество разновидностей форм выходной информации~---~2):
            $$t^V_{RZ} = 13,\hspace{1cm} t^V_{RP} = 15.$$
            Коэффициент $K_K$ и $K_Z$ были найдены выше:
            $$K_K=1.07, \hspace{1cm} K_Z=0.6.$$
            $K_R$~---~коэффициент учета режима обработки информации. По таблице принимаем~(группа новизны~---~А, режим обработки информации~---~реальный масштаб времени):
            $$K_R = 1.60.$$
            Тогда:
            $$\tau_{V} = (17+19)\cdot 1.07\cdot 1.60\cdot 0.6= 36.9$$
        Общая трудоёмкость разработки ПП:
        $$\tau_{PP} = 79+175+185.24+126.21+36.98= 602.43$$

    \subsection{Расчет количества исполнителей}
    \label{subsec:slaves}
        Средняя численность исполнителей при реализации проекта разработки и внедрения ПО определяется соотношением:
        $$N=\dfrac {Q_p} {F},$$
        где $t$~---~затраты труда на выполнение проекта (разработка и внедрение ПО); $F$~---~фонд рабочего времени.
        Разработка велась 5 месяцев с 1 января 2016 по 31 мая 2016.
        Из таблицы получаем, что фонд рабочего времени $$F=664.$$
        \begin{table}[h!]
            \caption{Количество рабочих дней по месяцам}
            \centering
            \label{tabular:work_days}
            \begin{tabular}{|c|c|c|}
                \hline
                \bf{Номер месяца} & \bf{Интервал дней}& \bf{Количество рабочих дней} \\ \hline
                1 & 01.01.2016~-~31.01.2016 & 15 \\ \hline
                3 & 01.02.2016~-~29.02.2016 & 20 \\ \hline
                4 & 01.03.2016~-~31.03.2016 & 21 \\ \hline
                5 & 01.04.2016~-~30.04.2016 & 21 \\ \hline
                6 & 01.05.2016~-~31.05.2016 & 19 \\ \hline
                \multicolumn{2}{|c|}{Итого} & 96 \\ \hline
            \end{tabular}
        \end{table}
        Получаем число исполнителей проекта:
        $$N=\dfrac{4816}{664}=7$$
        Для реализации проекта потребуются 3 старших инженеров и 4 простых инженеров.
        Исходя из того, что в месяце в среднем 22 рабочих дня, то для выполнения
        всего проекта потребуется около 4,7 месяца. На выполнение всего проекта,
        требуется около 22 недель.
    \subsection{Ленточный график выполнения работ}
        На основе рассчитанных в главах \ref{subsec:trud}, \ref{subsec:slaves}
        трудоёмкости и фонда рабочего времени найдём количество рабочих дней,
        требуемых для выполнения каждого этапа разработка.
        Результаты приведены в таблице~\ref{tabular:work_days}.
        Планирование и контроль хода выполнения разработки проводится по
        ленточному графику выполнения работ (см. таблицу \ref{tabular:lenta}).
        \begin{table}[ht!]
            \begin{adjustwidth}{-1cm}{}
            \caption{Ленточный график выполнения работ}
            \centering
            \label{tabular:lenta}
            \begin{tabular}{|c|c|c|c|c|c|c|c|c|c|c|c|c|c|c|c|c|c|c|c|c|c|c|c|c|}
                \hline
                & & \multicolumn{23}{|c|}{Календарные дни} \\ \cline{3-25}
                \parbox[t]{3mm}{\multirow{4}{*}[2em]{\rotatebox[origin=c]{90}{Номер стадии}}} &
                \parbox[t]{3.6mm}{\multirow{4}{*}[5.8em]{\rotatebox[origin=c]{90}{Продолжительность [раб.-дни]}}} &
                \rotatebox[origin=c]{90}{~01.01.2016~-~03.01.2016~} &
                \rotatebox[origin=c]{90}{~04.01.2016~-~10.01.2016~} &
                \rotatebox[origin=c]{90}{~11.01.2016~-~17.01.2016~} &
                \rotatebox[origin=c]{90}{~18.01.2016~-~24.01.2016~} &
                \rotatebox[origin=c]{90}{~25.01.2016~-~31.01.2016~} &
                \rotatebox[origin=c]{90}{~01.02.2016~-~07.02.2016~} &
                \rotatebox[origin=c]{90}{~08.02.2016~-~14.02.2016~} &
                \rotatebox[origin=c]{90}{~15.02.2016~-~21.02.2016~} &
                \rotatebox[origin=c]{90}{~22.02.2016~-~28.02.2016~} &
                \rotatebox[origin=c]{90}{~29.02.2016~-~06.03.2016~} &
                \rotatebox[origin=c]{90}{~07.03.2016~-~13.03.2016~} &
                \rotatebox[origin=c]{90}{~14.03.2016~-~20.03.2016~} &
                \rotatebox[origin=c]{90}{~21.03.2016~-~27.03.2016~} &
                \rotatebox[origin=c]{90}{~28.03.2016~-~03.04.2016~} &
                \rotatebox[origin=c]{90}{~04.04.2016~-~10.04.2016~} &
                \rotatebox[origin=c]{90}{~11.04.2016~-~17.04.2016~} &
                \rotatebox[origin=c]{90}{~18.04.2016~-~24.04.2016~} &
                \rotatebox[origin=c]{90}{~25.04.2016~-~01.05.2016~} &
                \rotatebox[origin=c]{90}{~02.05.2016~-~08.05.2016~} &
                \rotatebox[origin=c]{90}{~08.05.2016~-~15.05.2016~} &
                \rotatebox[origin=c]{90}{~16.05.2016~-~22.05.2016~} &
                \rotatebox[origin=c]{90}{~23.05.2016~-~29.05.2016~} &
                \rotatebox[origin=c]{90}{~30.05.2016~-~31.05.2016~}
                \\ \cline{3-25}
                & & \multicolumn{23}{|c|}{Количество рабочих дней} \\ \cline{3-25}
                  &    & 0 & 0 & 5 & 5 & 5 & 5 & 5 & 6 & 5 & 5 & 3 & 5 & 5 & 5 & 5 & 5 & 5 & 5 & 4 & 4 & 5 & 5 & 2 \\ \hline
                1 & 26 &   &   & 5 & 5 & 5 & 5 & 5 & 1 &   &   &   &   &   &   &   &   &   &   &   &   &   &   &   \\ \hline
                2 & 25 &   &   &   &   &   &   &   & 4 & 5 & 4 & 5 & 5 & 2 &   &   &   &   &   &   &   &   &   &   \\ \hline
                3 & 27 &   &   &   &   &   &   &   &   &   &   &   &   & 3 & 5 & 5 & 5 & 4 &   &   &   &   &   &   \\ \hline
                4 & 18 &   &   &   &   &   &   &   &   &   &   &   &   &   &   &   &   & 1 & 4 & 4 & 4 & 4 & 1 &   \\ \hline
                5 & 9  &   &   &   &   &   &   &   &   &   &   &   &   &   &   &   &   &   &   &   &   &   & 4 & 5 \\ \hline
            \end{tabular}
        %\end{sidewaystable}
        \end{adjustwidth}
        \end{table}


    \subsection{Определение себестоимости программной продукции}
        В работе над проектом используется специальное оборудование~---~персональные
        электронно-вычислительные машины (ПЭВМ) в количестве 1 шт.
        Стоимость одной ПЭВМ составляет 45000 рублей.
        Месячная норма амортизации K = 2,7\%.
        Тогда за 4 месяцев работы расходы на амортизацию составят
        $$P = 45000 \cdot  1 \cdot  0.027 \cdot  4 = 4860 \textnormal{ рублей}$$
        \begin{table}[ht!]
            %\small
            \centering
            \caption{Итоговая смета затрат}
            \label{tabular:costs}
            \begin{tabular}{|c|c|c|}
                \hline
                \bf{\specialcell{Номер \\ стадии}} & \bf{Название стадии}  &  \bf{\specialcell{Затраты \\ руб.}} \\ \hline
                1 & Затраты на оплату труда             & 2 189 265 \\ \hline
                2 & Дополнительная заработная плата     & 218 926 \\ \hline
                3 & Отчисления в ФСС                    & 842 867 \\ \hline
                4 & Амортизация оборудования            & 5 512 \\ \hline
                5 & Накладные расходы                   & 4 597 456 \\ \hline
                \multicolumn{2}{|c|}{Итого}   & 7 854 027 \\ \hline
            \end{tabular}
        \end{table}
        Вывод (см. таблицу \ref{tabular:costs}): затраты на разработку программы
        составляют: $7 854 027$ рублей.

    \subsection{Определение стоимости программной продукции}
        Для определения стоимости работ необходимо на основании плановых сроков
        выполнения работ и численности исполнителей рассчитать общую сумму затрат на
        разработку программного продукта.
        Если ПП рассматривается и создается как продукция производственно-технического назначения,
        допускающая многократное тиражирование и отчуждение от непосредственных разработчиков, то ее цена~$P$ определяется по формуле:
        $$P = K\cdot C+Pr,$$
        где $C$~---~затраты на разработку ПП (сметная себестоимость);
        $K$~---~коэффициент учёта затрат на изготовление опытного образца ПП как продукции производственно-технического назначения~($K=1.1$);
        $Pr$~---~нормативная прибыль, рассчитываемая по формуле:
        $$Pr= \frac {C \cdot  \rho_N} {100},$$
        где $\rho_N$~---~норматив рентабельности, $\rho_N=30\%$;
        Получаем стоимость программного продукта:
        $$ P=1.1\cdot 7 854 027 + 2 356 208\cdot 0.3=10 \hspace{4pt}995  \hspace{4pt}638\textnormal{ рублей}$$

    \subsection{Расчет экономической эффективности}
        Основными показателями экономической эффективности является чистый дисконтированный доход~(NPV) и срок окупаемости вложенных средств.
        Чистый дисконтированный доход определяется по формуле:
        $$NPV=\sum_{t=0}^T (R_t-Z_t) \cdot  \dfrac{1}{(1+E)^t},$$

        $T$~---~горизонт расчета по месяцам;

        $t$~---~период расчета;

        $R_t$~---~результат, достигнутый на $t$ шаге (стоимость);

        $Z_t$~---~текущие затраты (на шаге $t$);

        $E$~---~приемлемая для инвестора норма прибыли на вложенный капитал.

        На момент начала 2015 года, ставка рефинансирования 8.25\% годовых~(ЦБ РФ),
        что эквивалентно 0.68\% в месяц.
        В виду особенности разрабатываемого продукта он может быть продан лишь
        однократно.
        Отсюда получаем: $E=0.0068.$

        В таблице~\ref{tabular:npv} находится расчёт чистого дисконтированного
        дохода.
        График его изменения приведён на рисунке~\ref{pic:npv}.

        \begin{table}[ht!]
            %\small
            \begin{adjustwidth}{-1cm}{}
            \caption{Расчёт чистого дисконтированного дохода }
            \centering

            \label{tabular:npv}
            \begin{tabular}{|c|c|c|c|c|}
                \hline
                \bf{\specialcell{Месяц}} &
                \bf{\specialcell{Текущие затраты,\\ руб.}} &
                %\bf{\specialcell{Кол-во \\ работников}} &
                \bf{\specialcell{Затраты с начала \\ года, руб.}} &
                \bf{\specialcell{Текущий доход, \\ руб.}} &
                \bf{\specialcell{ЧДД, руб.}} \\ \hline

                Январь  & 939 352   & 939 352   & 0         & -939 352 \\ \hline
                Февраль & 1 783 352 & 2 712 705 & 0         & -2 700 853 \\ \hline
                Март    & 1 783 352 & 4 486 057 & 0         & -4 450 492 \\ \hline
                Апрель  & 1 783 352 & 6 259 410 & 0         & -6 188 439 \\ \hline
                Мая     & 1 353 798 & 7 613 208 & 10 995 638&  3 197 585 \\ \hline

            \end{tabular}
            \end{adjustwidth}
        \end{table}

        \begin{figure}[h!]
            \centering
            \begin{tikzpicture}[scale=1]
                \begin{axis}[ylabel=ЧДД (руб.), xlabel=Количество месяцев с начала проекта, ] %\tiny
                    \addplot coordinates {
                        (1, -939352)
                        (2, -2700853)
                        (3, -4450492)
                        (4, -6188439)
                        (5,  3197585)
                    };
                \end{axis}
            \end{tikzpicture}
            \caption{График изменения чистого дисконтированного дохода}
            \label{pic:npv}
        \end{figure}

        Согласно проведенным расчетам, проект является рентабельным.
        Разрабатываемый проект позволит превысить показатели качества существующих
        систем и сможет их заменить. Итоговый ЧДД составил: 3 197 585 рублей.

    \subsection{Результаты}
        В рамках организационно-экономической части был спланирован календарный
        график проведения работ по созданию подсистемы поддержки проведения
        диагностики промышленных, а также были проведены расчеты по трудозатратам.
        Были исследованы и рассчитаны следующие статьи затрат: материальные затраты;
        заработная плата исполнителей; отчисления на социальное страхование;
        накладные расходы.

        В результате расчетов было получено общее время выполнения проекта,
        которое составило 105 рабочих дней, получены данные по суммарным затратам
        на создание системы для тональной классификации сообщений сети
        {\it Twitter}, которые составили 7 613 208 рублей.

        Согласно проведенным расчетам, проект является рентабельным.
        Цена данного программного проекта составила 10 995 638 рублей, итоговый
        ЧДД составил 3 197 585 рублей.
