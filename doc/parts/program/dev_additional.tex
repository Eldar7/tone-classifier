\subsection{Разработка вспомогательных инструментов} % Выбор языка для реализации вспомогательных инструментов приложения
%   - Python
В роли вспомогательных инструментов выступают компоненты, которые в совокупности
решают две задачи:
\begin{itemize}
    \item Составление сбалансированной коллекции для обучения классификатора;
    \item Построения тонального лексикона на основе сообщений сети \twitter.
\end{itemize}


    \subsubsection{Прием текстовых сообщений сети Twitter}
    Прием тестовых сообщений реализован на основе библиотеки {\tt tweepy}, которая
    предоставляет интерфейс взаимодействия с {\it Streaming Twitter API}.
    Сначала требуется пройти авторизацию в социальной сети Twitter, основанную
    на открытом протоколе {\it OAuth}.
    От пользователя требуются предварительно зарегистрировать новое приложение,
    после чего получить необходимые ключи авторизации.

    Для приема и сохранения сообщений, реализован класс {\tt StdOutListener}
    (см. {\it приложение A}) который  наследует класс {\tt tweepy.StreamListener},
    c переопределением обработчиков следующих событий:
    \begin{itemize}
        \item {\tt on\_data} --- срабатывает в случае поступления новых данных
        с сервиса {\it Twitter Streaming API};
        \item {\tt on\_error} --- обработчик события возникновения исключения
        в процессе трансляции сообщений.
    \end{itemize}

    Все принятые сообщения сохраняются в текстовый файл формата {\tt .csv}.
    После того как объект-слушатель создан, необходимо подключиться к трансляции
    с параметрами фильтрации (строка 6, листинг \ref{lst:tweepyStream}):
    \begin{itemize}
        \item <<{\tt track=\lbrack"twitter"\rbrack}>> --- указывает на
        извлечение всех сообщений сети (все сообщения отмечены тегом
        {\tt twitter});
        \item <<{\tt languages=\lbrack"ru"\rbrack}>> --- определяет регион,
        сообщения которого попадут в трансляцию.
    \end{itemize}

    \lstset{style=python}
    \lstinputlisting[caption="Прием сообщений сети Twitter",
        label={lst:tweepyStream}]{parts/code/dev/tweepyStream.py}

    % КОД ВЫЛОЖИТЬ В ПРИЛОЖЕНИЕ

    \subsubsection{Создание лексиконов методом определения тональности словосочетаний}
    % Модуль pmi.py
    Для построения лексиконов используется подход, рассмотренный в п.
    \ref{sec:soEvaluation}.
    Подход основан на определении семантической ориентации словосочетаний, которая
    определяется метрикой {\it точечной взаимной информации} (на основе формулы
    \ref{eq:pmi}):
    \begin{equation}
        \label{eq:pmi2}
        PMI(t_1, t_2) = \log_2 \dfrac{P(t_1 \cap t_2)}{P(t_1) \cdot P(t_2)}
    \end{equation}

    Поскольку для каждого терма $t$, содержащегося в лексиконе необходимо
    сопоставить оценку тональности, то в качестве одного из аргументов метрики
    PMI можно рассмотреть один из двух маркеров:
    \begin{itemize}
        \item {\tt <<Excellent>>} --- положительный оттенок;
        \item {\tt <<Poor>>} --- негативный оттенок.
    \end{itemize}

    Степень принадлежности терма двум маркерам называется его {\it семантической
    ориентацией}, и определяется формулой (на основе формулы \ref{eq:so}):
    \begin{equation}
        SO(t) = PMI(t, {\tt Excellent}) - PMI(t, {\tt Poor})
    \end{equation}

    Пусть $K$ --- это коллекция сообщений. Тогда лексикон составляется следующим
    образом:
    \begin{equation}
        S : \{\langle t, SO(t) \rangle \hspace{2pt}| \hspace{4pt} t \in K_{Excellent} \cup K_{Poor}\}
    \end{equation}

    Где $K_{Excellent}$ и $K_{Poor}$ – разделение исходной коллекции $K$ на
    не пересекающиеся тональные классы сообщений с положительным и негативным
    оттенками соответственно.  Для построения оценочных лексиконов по сообщениям
    Твиттера вместо сочетаемости с конкретными словами используется встречаемость
    слова с положительными (отрицательными) эмотиконами.

    % Про компонент (pmieval.py)
    Компонент, который реализует построение лексикона\footnote{ \url{github.com/nicolay-r/tone-classifier/blob/2016_jan_contest/tools/pmieval/pmieval.py}},
    представляет собой сценарий, который в качестве аргументов принимает следующие
    параметры
    (
        оба параметра представляют собой названия таблиц СУБД
    ):
    \footnote{Сценарий использует дополнительные параметры, связанные
    с подключением к хранилищу, а также обработкой текстовых сообщений.
    Формат конфигурационных файлов рассматривается в {\it Руководстве
    Пользователя.}}

    \begin{enumerate}
        \item Тонально положительную коллекцию сообщений;
        \item Тонально негативную коллекцию сообщений.
    \end{enumerate}

    % Реализация операций.
    Реализация операции вычисления {\it точечной взаимной информации},
    представлена в листинге \ref{lst:pmi}.
    Аргументами функции {\it PMI} листинга \ref{lst:pmi}, являются параметры:
    \begin{enumerate}
        \item {\tt term} --- является параметром $t_1$
            формулы \ref{eq:pmi2};
        \item {\tt dv1} --- словарь, содержащий все документы положительной коллекции;
        \item {\tt dv2} --- словарь, содержащий все документы негативной коллекции сообщений.
    \end{enumerate}

    Вычисление результирующего значения функции производится на основе формулы \ref{eq:soPhrase}.

    \lstset{style=python}
    \lstinputlisting[caption="Вычисление точечной меры взаимной информации ({\it PMI}).",
        label={lst:pmi}]{parts/code/dev/pmi.py}

    Вычисление {\it семантической ориентации} приводится в листинге \ref{lst:so}.
    Помимо ранее рассмотренных аргументов для вычисления функции {\it PMI}
    листинга \ref{lst:pmi}, добавляются следущие параметры:
    \begin{itemize}
        \item {\tt tv1} --- словарь, содержащий все термы положительной коллекции;
        \item {\tt tv2} --- словарь, содержащий все термы негативной коллекции сообщений.
    \end{itemize}

    \lstset{style=python}
    \lstinputlisting[caption="Вычисление семантической ориентации терма",
        label={lst:so}]{parts/code/dev/so.py}

    % Что является результатом? (Формат таблицы)
        Результатом работы сценария является таблица со столбцами {\tt term}
    и {\tt tone}, которые используются для хранения термов и их семантической
    ориентации соответственно.
    В листинге \ref{lst:lexiconSave} представлена реализация сохранения результата

    \lstset{style=python}
    \lstinputlisting[caption="Сохранение результата",
        label={lst:lexiconSave}]{parts/code/dev/lexiconSave.py}

    \subsubsection{Балансировка исходных обучающих коллекций}
    % Описать с уклоном на атоматическую разметку сообщений на тональные классы.
    % (Как это было сделано в статьях)
    Согласно п. \ref{sec:tonalityCompetition}, обучающие коллекции,
    предоставленные организаторами сохраняют реальную классовую частотность
    сообщений.
    Это означает, что число нейтральных сообщений существенно превышает число
    сообщений в положительном и негативном классах.
    Модуль для балансировки обучающих коллекций основан на формате <<{\tt .csv}>>.
    Именно в одном из таких форматов распространяется {\it тональный корпус коротких
    текстов Ю. Рубцовой} \cite{rubtsovaCollection}.

    Корпус представляет собой набор сообщений, разбитый на классы
    {\it положительных} и {\it отрицательных}.
    % ДОБАВИТЬ ОПИСАНИЕ ПРО ПОСТРОЕНИЕ ЛЕКСИКОНА ДЛЯ КОЛЛЕКЦИИ РУБЦОВОЙ!!!
    На основе корпуса коротких текстов был построен лексикон $L$, который
    используется в дальнейшем для определения {\it наиболее эмоциональных слов}
    $w_i$:
    \begin{equation}
        w_i : |L(w_i)| > K
    \end{equation}

    Где $K$ -- пороговое значение.
    Списки наиболее эмоциональных слов лексикона $L$ представлены в
    {\it Приложении Б } (листинги \ref{lst:positiveTerms}--\ref{lst:negativeTerms}).
    Процесс построения сбалансированной коллекции на основе существующей
    коллекции реализуется выполнением следующих действий:
    \begin{enumerate}
        \item Создание обучающей коллекции на основе уже существующей,
            несбалансированной коллекции (см. п. \ref{sec:devImporting});
        \item Чтение сообщений корпуса коротких текстов;
        \item {\bf Фильтрация сообщений:} в обучающую коллекцию добавляются
            только те сообщения, которые содержат наиболее эмоциональные слова
            из составленного лексикона.
    \end{enumerate}

    Что касается последнего пункта, то в зависимости от знака такого значения,
    определяется тональный класс в обучающей коллекции, в который будет
    произведено добавление. Если в сообщении встречается несколько эмоциональных
    слов, и при этом принадлежащих разным классам, то сообщение игнорируется.
