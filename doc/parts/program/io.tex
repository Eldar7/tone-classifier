\subsection{Формат представления коллекций}
    \label{sec:programmingInnerFormat}
    % Здесь рассматриваются:
    %  -Формат данных, которые подаются на вход для проведения семантической
    % классификации. (XML)
    %  -Формат выходных данных (оценка качества работы классификатора
    %   (опционально), XML файл с проставленными оценками)

    Изначально, для описания коллекций используется документ формата \xml,
    который представляет собой экспорт таблицы с сообщениями из СУБД с поддержкой
    {\it SQL} синтаксиса.
    Именно в таком формате, организаторы соревнований {\it SentiRuEval}
    распространяют все типы коллекций по каждой тематике. Структура документа
    поделена на две части:
    \begin{enumerate}
        \item Описание таблицы, используемой для хранения текстовых сообщений с
            тональными оценками (см. листинг \ref{lst:collectionTableExample});
        \item Сообщения с заполненными в соответствии с форматом таблицы
            параметрами (см. листинг \ref{lst:collectionMessageExample});
    \end{enumerate}

    \newpage
    \lstset{style=xml}
    \newpage
    \lstinputlisting[caption="Формат описание таблицы для хранения сообщений",
        label={lst:collectionTableExample}]{parts/code/collectionTableExample.xml}

    \lstset{style=xml}
    \lstinputlisting[caption="Формат представления сообщений",
    label={lst:collectionMessageExample}]{parts/code/collectionMessageExample.xml}

    Все параметры сообщения можно разделить на два класса: {\it обязательные} и
    {\it дополнительные}. Каждое сообщение содержит следующий набор
    обязательных параметров (см. листинг \ref{lst:collectionMessageExample}):
    \begin{enumerate}
        \item {\tt id} --- идентификатор сообщения, уникальный в пределах рассматриваемой коллекции;
        \item {\tt twitid} --- уникальный идентификатор сообщения сети \twitter;
        \item {\tt date} --- дата появления сообщения в сети;
        \item {\tt text} --- текст сообщения.
    \end{enumerate}

    Набор дополнительных параметров зависит от предметной области, в которой
    рассматривается сообщение. В общем случае, это организации, по отношению к
    которым рассматривается сообщение.

    \subsection{Формат выходных данных}
    %
    % Выходные данные
    %
    Одним из результатов работы классификатора является \xml
    документ, формат которого совпадает с представлением листинга \ref{lst:collectionMessageExample},
    и в котором для каждого сообщения во все дополнительные поля проставлена
    тональная оценка.

    Оценка качества работы классификатора представляет собой текстовое сообщение,
    содержащее результаты показаний  $F_{macro}(neg, pos)$ и $F_{micro}(neg, pos)$,
    а также значения вспомогательных параметров, на основе которых они были
    получены. Пример вывода результата качества работы классификатора представлен
    в листинге \ref{lst:classifierScores}.

    \lstset{style=bash, numbers=left}
    \lstinputlisting[caption="Формат описания результатов классификатора.", label={lst:classifierScores}]{parts/code/classifierScores.xml}

    Разберем структуру листинга \ref{lst:classifierScores} подробнее. В строках 1-2
    указываются значения параметров таблицы контингентности (см. таблицу \
    ref{table:contingent}) для классов {\it positive} и {\it negative}. Значения параметров точности и
    полноты для каждого из тональных классов указаны в строках 3-4. Результаты
    макро- и микрооценок $F-$меры представлены в строках 5-6.

    %
    % Следующий раздел требует подробного рассмотрения здесь формата
    % представления сценария, предоставленного организаторами соревнования SentiRuEval
    % для тестирования качества работы модели
    %
    Для оценки качества работы классификатора, организаторами соревнований
    {\it SentiRuEval} предоставляется сценарий реализованный на языке {\it Javascript}.
    Аргументами сценария являются:
    \begin{itemize}
        \item Тип задачи --- какой сфере компаний принадлежат сообщения. Используется
            для определения полей, которые будут использованы для подсчета оценок;
        \item Тестовая коллекция соответствующей задачи;
        \item Эталонная коллекция соответствующей задачи.
    \end{itemize}

    Тестовая коллекция должна содержать оценки, проставленные классификатором.
    Обе коллекции должны соответствовать формату представления данных п.
    \ref{sec:programmingInnerFormat}.
