\subsection{Разработка тонального класификатора}
    % Архитектура проекта
    % Описать зависимости, а также что требуется реализовать.
    % Для решения задачи сентиментального анализа требуется:
    %   - классификатор (LibSVM)
    %   - модули обработки сообщений (Mystem + тональные префиксы + списки стоп слов)
    %
    % Выбор языка для реализации приложения
    %   - Python
    % хранения коллекций сообщений сети Twitter, а также вспомогательных данных:
    %   - База данных
    В качестве основы для разработки приложения был выбран язык {\it Python},
    ввиду следующих особенностей:
    \begin{enumerate}
        \item Отсутствие временных ограничений на обработку сообщений,
            классификацию;
        \item Возможность быстро подстраиваться под изменение архитектуры
            проекта и введения новых возможностей;
        \item Наличие библиотеки LibSVM\cite{svmClassifier} реализующей
            классификацию методом <<Опорных Векторов>>.
    \end{enumerate}

    Что касается вопроса хранения коллекций и вспомогательной информации
    для решения задачи классификации, для этих целей используется СУБД
    {\it PostgreSQL}. Такой выбор обусловлен удобным интерфейсом взаимодействия
    через сценарии языка {\it Python}, а также возможностью удобного консольного
    администрирования хранилища. Как и в случае с выбором языка для реализации
    задачи, на выбор хранилища не накладываются временные ограничения.

    Под разработкой проекта понимается реализация сценариев на языке Python,
    для выполнения следующих подздачач:
    %
    % Описать в формате потока, какие действия требуются чтобы произвести
    % классификацию с оценкой или без нее.
    %
    \begin{enumerate}
        \item Подготовка коллекций -- импорт XML данных в хранилище;
        \item Построение модели на основе обучающей коллекции;
        \item Применение модели к данным тестовой коллекции;
        \item Экспорт размеченных классификатором сообщений из хранилища в XML формат;
        \item Вычисление оценки качества работы модели.
    \end{enumerate}

    \subsubsection{Обработка сообщений сети Twitter}
    % Описать про модуль обработки.

    \subsubsection{Использование LibSVM для классификации методом <<Опорных векторов>>}
    % Описать про использование SVM классификатора

    \subsubsection{Вспомогательные признаки классификации}
    % Модуль добавления признаков в векторизацию сообщений.

\subsection{Разработка вспомогательных инструментов}
% Выбор языка для реализации вспомогательных инструментов приложения
%   - Python
В роли вспомогательных инструментов выступают компоненты, которые в совокупности
решают задачу построения тонального лексикона на основе сообщений сети \twitter.
Реализация сценариев осуществляется посредством языка {\it Python}.

    \subsubsection{Прием текстовых сообщений сети Twitter}

    \subsubsection{Создание лексиконов методом определения тональности словосочетаний}

