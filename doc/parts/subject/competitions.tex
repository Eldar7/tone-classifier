\subsection{Соревнования в области тональной классификации сообщений}
    \label{sec:tonalityCompetition}
    % цель задачи
    За последние несколько лет, в области автоматической обработки текстовых сообщений
    активно набирает популярность платформа открытого тестирования систем
    тонального анализа русскоязычных сообщений {\it SentiRuEval} \cite{tonalityanalisys}.
    тестирование проводится в формате соревнований, к которым организаторы заранее подготавливают
    коллекции данных. Данные представляют собой сообщения сети {\it Twitter},
    каждое из которых выражает положительное либо негативное мнение автора по
    отношению к рассматриваемой в сообщении организации.
    Среди предметных областей доступны следующие коллекции
    данных:
    \begin{itemize}
        \item Сообщения о банковских компаниях;
        \item Сообщения о телекоммуникационных компаниях.
    \end{itemize}

    Тональная классификация заключается в определении оценки по отношению к
    рассматриваемой в сообщении компании. сообщения, для которых пользователям
    необходимо проставить оценку, представляются в {\it тестовой коллекции}. Изначально все
    сообщения тестовой коллекции отмечены как <<нейтральные>>, и имеют оценку <<0>>.
    Участнику предлагается изменить значение оценки на <<1>> если отношение к
    рассматриваемой организации <<положительное>>, или на <<-1>> в случае
    <<негативного>> отношения.

    % про коллекции (почему они не сбалансированы, процентное соотношение твитов)

    Помимо тестовых коллекций, организаторы также подготавливают {\it обучающие} и
    {\it эталонные} коллекции для каждой из областей. последние, в свою очередь,
    доступны после окончания соревнований. данные для коллекций собираются на
    основе {\it Streaming API Twitter}.
    Отмечается, что при сборе данных отсутствует
    искусственное завышение классов с малым числом сообщений \cite{tonalityanalisys}.
    Так, во всех предметных областях преобладают сообщения с нейтральной оценкой (см. таблицы
    \ref{table:trainTableStat} и \ref{table:testTableStat}).

    \begin{table}[h]
        \centering
        \caption{Распределение объемов классов сообщений в обучающих коллекциях}
        \label{table:trainTableStat}
        \begin{tabular}{|c|c|c|c|}
            \hline
            Область     & Нейтральные \%    & Тональные  \%     & Число сообщений \cite{dialog2015} \\ \hline
            TKK        & 47.59              & 52.42             &   \~ 4\hspace{3pt}800\\ \hline
            Banks      & 58.35              & 41.65             &   \~ 4\hspace{3pt}900\\ \hline
        \end{tabular}
     \end{table}

    \begin{table}[h]
        \centering
        \caption{Распределение объемов классов сообщений в тестовых коллекциях}
        \label{table:testTableStat}
        \begin{tabular}{|c|c|c|}
            \hline
            Область     & Нейтральные \%    & Тональные  \%     \\ \hline
             TKK        & 67.70             & 32.30             \\ \hline
             Banks      & 77.90             & 22.10             \\ \hline
        \end{tabular}
     \end{table}
    % система оценки (метрики, используемые для оценки классификационной модели)
    Для {\it оценки качества} классификатора, используются макро- и микроусреднения
    $F_1$-меры для классов положительных и негативных сообщений. Вычисление
    $F_{macro(pos, neg)}$ и $F_{micro(pos, neg)}$ производится на основе
    формул \ref{eq:fmacro12} и \ref{eq:fmicro12} соответственно. Нейтральный
    класс сообщения в оценке не участвует. Однако в случае, если классификатор
    некорректно определяет тональное сообщение, то это выявляется в расхождении
    мнения классификатора и эксперта в таблицах контингентностей, составленных для
    классов $pos$ и $neg$, на основании которых высчитываются параметры полноты
    и точности.

    % Результаты
    Результаты участников сравниваются относительно нижнего порогового значения -- {\it baseline}.
    В качестве такого порога рассматривается классификатор, который для каждого
    сообщения проставляет тональную оценку наиболее частотного класса. В таблицах
    \ref{table:exampleResultsBanks} и \ref{table:exampleResultsTTK} рассматривается
    прогоны, демонстрирующие лучшие результаты в сравнении с нижним пороговым
    значением. Параметры прогонов следующие \cite{tonalityanalisys}:
    \begin{enumerate}
        \item Классификатор {\it SVM}, с учетом признаков:
            нормализованные леммы, установление связи между леммами;
        \item Метод максимальной энтропии с учетом признаков:
            $n$-граммы (слов и символьные), дополнительные результаты моделирования;
        \item Классификатор {\it SVM} с учетом: $n$-граммы (словесные и буквенные),
            знаки пунктуации, наличие ссылок, символы <<ретвита>>, использование
            эмоционально окрашенного словаря.
    \end{enumerate}

    \begin{table}[H]
        \centering
        \caption{Результаты качества работы лучшего прогона в сравнении с {\it baseline}.
        {\it SentiRuEval-2015}, коллекция сообщений о банках}
        \label{table:exampleResultsBanks}
        \begin{tabular}{|c|c|c|}
        \hline
        №                       &       $F_{macro(pos, neg)}$        & $F_{micro(pos, neg)}$  \\ \hline
        baseline                &           0.1267                      &       0.2377              \\ \hline \hline
        1                       &           0.3354                      &       0.3656              \\ \hline
        2                       &           0.3598                      &       0.3430              \\ \hline
        3                       &           0.3520                      &       0.3370              \\ \hline
        \end{tabular}
     \end{table}

     \begin{table}[H]
        \centering
        \caption{Результаты качества работы лучшего прогона в сравнении с {\it baseline}.
        {\it SentiRuEval-2015}, коллекция телекоммуникационных сообщений}
        \label{table:exampleResultsTTK}
        \begin{tabular}{|c|c|c|}
        \hline
        №                       &       $F_{macro(pos, neg)}$        & $F_{micro(pos, neg)}$  \\ \hline
        baseline                &           0.1823                      &       0.3370              \\ \hline \hline
        1                       &           0.4882                      &       0.5355              \\ \hline
        2                       &           0.4670                      &       0.5060              \\ \hline
        3                       &           0.4477                      &       0.5282              \\ \hline
        \end{tabular}
     \end{table}

    % Анализ (почему наблюдается разница в результатах между задачами)
    Как можно заметить, показания таблиц \ref{table:exampleResultsBanks} и \ref{table:exampleResultsTTK}
    демонстрируют значительную разницу в максимумах для каждого из прогонов двух областей.
    Для выявления причины, вызвавшей такое расхождение, в \cite{tonalityanalisys}
    проводится подсчет разницы между обучающей и тестовых коллекцией для каждой из
    областей на основе:
    \begin{itemize}
        \item Несимметричная мера удаленности -- дивергенция Кульбака-Лейблера;
        \item Симметричная мера Йенсена-Шеннона.
    \end{itemize}

    Результаты показали, что наибольшее расхождение между коллекциями наблюдалось в
    области банков.
    Такое расхождение может быть объяснено тем, что коллекции для каждой из областей
    были подготовлены в разные периоды времени. Обучающие коллекции составлены в
    период июнь-август 2014 года, в момент разгара боевых действий на Украине.
    В тоже время, данные для тестовых коллекций были получены {\it на пол года раньше}, в период с
    декабря 2013 по февраль 2014.

    % Ориентация на анализ
    Несмотря на то, что в сообщениях могло упоминаться отношение к нескольким
    компаниям одновременно, большинство участников все равно предпочли
    оценивать сообщение в целом. Участники с такой стратегией показали
    более высокие результаты чем те, кто пытался производить оценку для каждой
    компании в отдельности. Отсюда можно сделать вывод, что на сегодняшний момент
    задача по тональной оценке отдельных объектов сообщения является весьма
    ограниченной \cite{tonalityanalisys}.
