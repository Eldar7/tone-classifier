\subsection{Способы построения тональных лексиконов}
    В любом предложении естественного языка содержатся ключевые слова (и словосочетания),
    на основании которых можно извлечь дополнительную информацию. Если рассматривать
    предложение с точки зрения тонального анализа, то некоторые слова могут
    придавать негативную или положительную окраску сообщению в целом.
    В связи с этим, умение точно извлекать тональные слова или словосочетания
    с целью составления дополнительной метаинформации о сообщении,
    может оказать положительное влияние на качество классификации.

    Можно сказать, что возникает необходимость в словаре, состоящего из пар
    $\langle w, t \rangle$, где $w$ --- слово или словосочетание, а $t$ --- его оценка.
    Словарь, представленный в таком формате называется {\it лексиконом}.
    Задача, которую решает лексикон в области тонального анализа --- сопоставление
    тональной окраски для каждого слова содержащегося в нем.
    Очевидно, что формализация оценочного параметра приводит к представлению
    параметра $t$ в формате числового значения, возможно с учетом знака.
    Обычно в качестве оценки рассматривают действительные значения в области
    $\left[ -1, 1\right]$.

    Рассмотрим основные методы вычисления оценочных параметров, на основе которых
    возможно автоматическое создания тональных лексиконов.

        \subsubsection{Вычисление оценки на основе метрики log-likelihood}
        Метод, предложенный в \cite{lexiconLL} предполагает наличие двух корпусов, на основании
        которых вычисляется частотная оценка принадлежности слова к каждому из корпусов.
        В общем случае, под {\it <<словом>>} понимается терм предложения, однако
        реальное применение в \cite{lexiconLL} находит свое место в составлении
        списка частот для определенных частей речи.

        Тем не менее, в общем случае для подсчета на уровне термов {\it ожидаемой
        степени принадлежности} $E_i(w)$, применяется следующая формула:
        \begin{equation}
            \label{eq:classDifference}
            E_i(w) = \dfrac{N_i\sum\limits_{j=1}^{2}O_i(w)}{\sum\limits_{j=1}^{2}N_i}
        \end{equation}

        Здесь, параметр $i$ является индексом коллекции, $O_j(w)$ -- обозреваемое
        значение, которое соответствует числу вхождений $w$ в корпус $j$, а $N_j$
        -- общее число слов в корпусе $j$.

        Далее, на основании полученных степеней принадлежности терма соответствующему
        классу, применяют метрику log-likelihood для вычисления степени <<важности>>
        терма:
        \begin{equation}
            \label{eq:loglikelihood}
            - \ln \lambda = \sum\limits_i O_i \ln \left( \dfrac{O_i}{E_i} \right)
        \end{equation}

        Если произвести сортировку словаря на основе формулы параметра формулы
        \ref{eq:loglikelihood} в формате убывания значения, то вершину списка
        будут представлять термы $w$, для которых наблюдается большая разница в
        степени принадлежности рассматриваемых корпусов. Соответственно, термы
        с минимальной разницей окажутся в конце отсортированного списка.

        Степень важности терма является положительной величиной. Дополнительно можно
        ввести знаковый параметр на основании разницы между показаниями формулы
        \ref{eq:classDifference} для двух классов.

        \subsubsection{Предсказывание семантической ориентации прилагательных}
        \label{sec:adjectivesPrediction}
        В работе \cite{lexiconAdjectives} рассматривается способ определения
        тональной оценки прилагательных, которые являются аргументами конъюнктов.
        Основой подхода является гипотеза об ограничении ориентации прилагательных
        в зависимости от типа объединяющего из конъюнкта и наличия префиксов отрицаний:
        %В таких лингвистических конструкциях как конъюнкты, связь и ориентация
        %аргументов зависит от способа их соединения:
        \begin{center}
            \it
            Этот бриллиант \underline{красивый} {\bfи} \underline{дорогой}

            Бриллиант \underline{красивый}, {\bfно} \underline{дорогой}

            Это украшение \underline{некрасивое} {\bfи} \underline{дорогое}

            Украшение \underline{некрасивое}, {\bfно} \underline{дорогое}
        \end{center}

        Для построения автоматического извлечения семантически ориентированной
        информации на основе содержимого корпуса большого объема.
        Извлечение и оценка тональности прилагательных, а также связей между ними
        реализуется следующими этапами:
        \begin{enumerate}
            \item Из корпуса выбираются все конъюнкты с релевантными
                морфологическими связями.

                Извлечение конъюнктов выполнялось с помощью формальной
                грамматики, примененной к корпусу из 21 миллиона словоформ.
                В результате было собрано около 15 тысяч пар прилагательных;

            \item Применяется регерессионная модель для объединения информации
                разных конъюнктов с целью определения ориентации каждой из связанных пар.
                Результатом является граф с одинокого либо разно ориентированными
                связями между прилагательными;
            \item Разделение прилагательных на две группы с разными ориентациями
                с помощью алгоритма кластеризации. Каждая группа включает в себя
                максимально возможное число прилагательных.
                % 6 (если нужно более подробно)
                Для этого, к построенному графу применяется функция $\Phi$,
                параметром которой является $\mathcal{P}$ -- разбиение множества
                прилагательных на группы $C_1$ и $C_2$:
                \begin{equation}
                    \Phi(\mathcal{P}) = \sum\limits_{i=1}^2 \Bigg( \dfrac{1}{|C_i|} \sum\limits_{x,y \in C_i, \\ x \neq y} d(x,y) \Bigg)
                \end{equation}
                Где $|C_i|$ -- мощность соответствующего класса; $d(x, y)$ --
                степень различия прилагательных $x$ и $y$.
                Далее, осуществляется поиск параметра $\mathcal{P}_{min}$ для
                построения наилучшего разбиения;

            \item Для групп $C_i, \hspace{4pt} i=\overline{1,2}$ вычисляются
                средние частоты упоминания входящих в них прилагательных.
                Та группа, для которой полученное значение наибольшее --
                маркируется как тонально-положительная.
                % 7
                Ранее, в работе \cite{lexiconAdjectivesPrevious} показывается,
                что противопоставление качественным прилагательным (т.е.
                прилагательные противоположной группы) в 81\% случаев оказывается
                семантически не размеченным. Отмечается, что этот факт демонстрирует
                корреляцию с ориентацией прилагательного, которая наиболее вероятно
                является положительной.
        \end{enumerate}

        Применение описанного выше подхода к задаче классификации конъюнктов
        позволили добиться довольно высоких показаний точности (91\% и выше).
        Отмечается потенциальная возможность применения метода к другим
        частям речи.

        \subsubsection{Вычисление оценки на основе тональности словосочетаний}
        \label{sec:soEvaluation}
        В работе \cite{lexiconSO} предлагается подход который в отличие от
        статьи \cite{lexiconAdjectives}, рассмотренной в п. \ref{sec:adjectivesPrediction},
        применим к любым словосочетаниям. Алгоритм состоит из выполнения следующих шагов:
        \begin{enumerate}
            \item Извлечение {\it синтаксических конструкций}, для которых будет производиться
                дальнейшая оценка. В качестве таких конструкций могут быть фразы, содержащие
                прилагательные или наречия (п. \ref{sec:adjectivesPrediction}). %[Ссылка на предыдущую работу].
                В тоже время, при извлечении одной лишь части речи может возникнуть
                проблема нехватки контекста для определения оценки.
                Так, например прилагательное <<непредсказуемость>> может иметь
                как негативную окраску в словосочетании <<непредсказуемое поведение>>,
                так и положительную: <<непредсказуемый сюжет>>.
            %Далее, под термином <<слово>> будем понимать произвольную синтаксическую конструкцию, извлеченную
            %из текста.
            \item Производится оценка извлеченных синтаксических конструкций на основе
                {\it меры взаимной информации} (от англ. {\it Pointwise Mutation Information, PMI}).
                Вычисление меры производится между двумя словами $word_1$ и $word_2$,
                и определяется следующим образом:
                \begin{equation}
                    \label{eq:pmi}
                    PMI(word_1, word_2) = log_2 \Bigg[ \dfrac{P(word_1 \hspace{4pt} \wedge \hspace{4pt} word_2)}{P(word_1) \cdot P(word_2)} \Bigg]
                \end{equation}

                Где $P(word_1 \hspace{4pt} \wedge \hspace{4pt} word_2)$ -- вероятность
                размещения слов $word_1$ и $word_2$ вместе. Соотношение числителя
                к знаменателю в формуле \ref{eq:pmi} определяет степень зависимость одного
                слова от другого.

                {\it Семантической ориентацией} (от англ. {\it Semantic Orientation, SO}), для рассматриваемой
                конструкцией называется величина, которая вычисляется на основе
                тональных маркеров $ ``positive\text{''}$ и $``poor\text{''}$, являющихся аргументами
                меры точечной взаимоинформации:
                \begin{equation}
                    \label{eq:so}
                    SO(word) = PMI(word, ``excellent\text{''}) - PMI(word, ``poor\text{''})
                \end{equation}

                Наибольшее значение в формуле \ref{eq:so} достигается в случае
                наличия наибольшей связи рассматриваемой конструкции $phrase$
                с маркером $ ``excellent\text{''}$; наименьшее, в случае наибольшей связи
                с $ ``poor\text{''}$ маркером.

                Формула \ref{eq:pmi} рассматривалась применительно к параметрам
                {\it word}. Для вычисления уравнения \ref{eq:so} от параметра
                $phrase$, используется функция на основе числа совпадений
                $hits(phrase)$, а также оператор {\it NEAR} расположения слов в
                непосредственной близости. В сочетании с формулами \ref{eq:pmi}
                и \ref{eq:so}, семантическая ориентация вычисляется следующим
                образом:
                \begin{equation}
                    \label{eq:soPhrase}
                    SO(phrase) = log \Bigg[ \dfrac{hits(phrase \hspace{4pt} NEAR \hspace{4pt}``excellent\text{''})hits(``poor\text{''}) }{hits(phrase \hspace{4pt}NEAR \hspace{4pt}``poor\text{''})hits(``excellent\text{''})} \Bigg]
                \end{equation}

            \item Для подсчета оценки могут быть использованы формулы \ref{eq:so} и
                \ref{eq:soPhrase}.
        \end{enumerate}

        Эксперименты, проведенные для набора из 410 обзоров с ресурса мнений пользователей {\it Epinions},
    показывают среднюю точность классификации порядка 74\%. Как оказалось, самым
    сложным источником для анализа оказались обзоры на фильмы; на такой коллекции
    точность классификации опустилась до уровня 64\%. Наилучшие показатели были
    достигнуты на коллекциях мнений об автомобилях и банках, на которых точность
    оказалась в пределах 80--84\%.
