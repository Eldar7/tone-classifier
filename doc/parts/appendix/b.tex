\lstset{
numbers=none,
keywordstyle=\bfseries,
basicstyle=\footnotesize,
frame=single,
literate={а}{{\selectfont\char224}}1
{б}{{\selectfont\char225}}1
{в}{{\selectfont\char226}}1
{г}{{\selectfont\char227}}1
{д}{{\selectfont\char228}}1
{е}{{\selectfont\char229}}1
{ё}{{\"e}}1
{ж}{{\selectfont\char230}}1
{з}{{\selectfont\char231}}1
{и}{{\selectfont\char232}}1
{й}{{\selectfont\char233}}1
{к}{{\selectfont\char234}}1
{л}{{\selectfont\char235}}1
{м}{{\selectfont\char236}}1
{н}{{\selectfont\char237}}1
{о}{{\selectfont\char238}}1
{п}{{\selectfont\char239}}1
{р}{{\selectfont\char240}}1
{с}{{\selectfont\char241}}1
{т}{{\selectfont\char242}}1
{у}{{\selectfont\char243}}1
{ф}{{\selectfont\char244}}1
{х}{{\selectfont\char245}}1
{ц}{{\selectfont\char246}}1
{ч}{{\selectfont\char247}}1
{ш}{{\selectfont\char248}}1
{щ}{{\selectfont\char249}}1
{ъ}{{\selectfont\char250}}1
{ы}{{\selectfont\char251}}1
{ь}{{\selectfont\char252}}1
{э}{{\selectfont\char253}}1
{ю}{{\selectfont\char254}}1
{я}{{\selectfont\char255}}1
{А}{{\selectfont\char192}}1
{Б}{{\selectfont\char193}}1
{В}{{\selectfont\char194}}1
{Г}{{\selectfont\char195}}1
{Д}{{\selectfont\char196}}1
{Е}{{\selectfont\char197}}1
{Ё}{{\"E}}1
{Ж}{{\selectfont\char198}}1
{З}{{\selectfont\char199}}1
{И}{{\selectfont\char200}}1
{Й}{{\selectfont\char201}}1
{К}{{\selectfont\char202}}1
{Л}{{\selectfont\char203}}1
{М}{{\selectfont\char204}}1
{Н}{{\selectfont\char205}}1
{О}{{\selectfont\char206}}1
{П}{{\selectfont\char207}}1
{Р}{{\selectfont\char208}}1
{С}{{\selectfont\char209}}1
{Т}{{\selectfont\char210}}1
{У}{{\selectfont\char211}}1
{Ф}{{\selectfont\char212}}1
{Х}{{\selectfont\char213}}1
{Ц}{{\selectfont\char214}}1
{Ч}{{\selectfont\char215}}1
{Ш}{{\selectfont\char216}}1
{Щ}{{\selectfont\char217}}1
{Ъ}{{\selectfont\char218}}1
{Ы}{{\selectfont\char219}}1
{Ь}{{\selectfont\char220}}1
{Э}{{\selectfont\char221}}1
{Ю}{{\selectfont\char222}}1
{Я}{{\selectfont\char223}}1
}



В листингах \ref{lst:positiveTerms}-\ref{lst:negativeTerms} представлены
наиболее эмоциональные слова {\it корпуса коротких текстров Ю. Рубцовой} \cite{rubtsovaCollection}.
Степень эмоциональности определяется на основе подхода, описанного в п.
\ref{sec:soEvaluation}.

\begin{lstlisting}[caption="Эмоционально положительные термы", label={lst:positiveTerms}]
"D", "DDD", "DDDD", "XD", "DD", "царевич", "xD", "DDDDD", "DDDDDD", "уругвай",
"#улыбнуло", "хрещатик", "баярлалаа", "#ЛУЧИРАДОСТИОТРАДОСТИ", "конгениальность",
"#рубль", "аге", "XDD", "позаимствовать", "ржач", "листаться", "бесподобный",
"Microsoft", "позитив", "реквестировать", "улыбнуть", "#музыка", "бугага",
"ехууу";
\end{lstlisting}

\begin{lstlisting}[caption="Эмоционально негативные термы", label={lst:negativeTerms}]
"теракт", "погибший", "еб**ат", "цымбаларь", "#сми", "пострадавший", "гренобль",
"цег", "#Вконтакте", "траур", "критический", "однобокость", "михаэль", "поч",
"хнык", "блинн", "таскание", "пичаливать", "грусный", "пичаливать", "печалька",
"почемууу", "покоиться", "пращать", "смертница", "навидеть", "разочарованный",
"мазерать", "о", "скорбеть";
\end{lstlisting}
