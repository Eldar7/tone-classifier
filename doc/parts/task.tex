\newpage
\section{Постановка задачи}
    %Так, рассматривая произвольную коллекцию $K_i$, {\bf требуется} для каждого
    %сообщения, входящего в коллекцию, определить тональную оценку по отношению к
    %компаниям.
    %Оценка по отношению к каждой из компаний может быть одной из
    %трех типов: положительной, нейтральной, либо негативной.

    {\it SentiRuEval} --- название соревнований, которые проводятся в
    области, которой посвящена эта работа.
    %
    % Что требуется сделать
    %
    Имеются {\it тестовые коллекции} сообщений, для которых известно, что
    все сообщения внутри каждой из коллекций рассматриваются относительно
    фиксированного множества компаний.
    {\bf От участника требуется} классифицировать сообщения тестовых коллекций на три
    класса: положительные ({\tt 1}), нейтральные ({\tt 0}), либо негативные ({\tt -1}) относительно компаний,
    которые упоминаются в сообщении.

    %
    % Что дано
    %
    {\bf Заданы} {\it тестовые} и {\it обучающие} коллекции сообщений сети \twitter в двух областях:
    сообщения о банковских компаниях,
    сообщения о телекоммуникационных компаниях.
    Для каждого сообщения коллекции известно подмножество
    множества компаний, которое в нем рассматривается.
    Тестовая коллекция отличается от обучающей тем, что в первой для каждого
    сообщения проставлена нейтральная оценка
    (т.е. для каждой рассматриваемой в сообщении компании проставлена нейтральная оценка ({\tt 0})).

    %
    % Как протестировать
    %
    {\bf Для тестирования} качества работы классификатора используется сценарий
    предоставляемый организаторами, который вычисляет оценки по каждой задачи
    (областей сообщений) в отдельности. Для оценки используются метрики:
    $F_{macro}(neg, pos)$,
    $F_{micro}(neg, pos)$.
    Тестирование производится на основе {\it эталонной} коллекции сообщений.
    Такая коллекция представляет собой размеченную экспертами тестовую коллекцию.
    Эталонные коллекции становятся доступными после окончания проведения
    соревнований.
