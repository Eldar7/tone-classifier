\newpage
\subsection{Участие в соревнованях SentiRuEval-2016}
    \subsubsection{Результаты}
В таблице \ref{table:tkkResult2015} приведены оценки качества работы
классификатора для тестовой коллекции {\it SentiRuEval-2016} \cite{dialog2016}
при использовании настроек предварительного тестирования.
Прогоны с такими настройками показали лучшие результаты среди других
вариаций настроек предложенного подхода (см. таблицы \ref{table:bankResult2016}-\ref{table:tkkResult2016}).

    \begin{table}[!ht]
    \centering
    \caption{Результаты прогонов соревнования (задача BANK, {\it SentiRuEval-2016})}
    \label{table:bankResult2016}
    \begin{tabular}{|c|c|c|c|c|}
    \hline
    \multirow{3}{*}{№} & \multicolumn{4}{c|}{BANK}                                                                                                                                                                                         \\ \cline{2-5}
                       & \multicolumn{2}{c|}{\begin{tabular}[c]{@{}c@{}}Не сбалансированная \\ коллекция (2015 год)\end{tabular}} & \multicolumn{2}{c|}{\begin{tabular}[c]{@{}c@{}}Расширенная сбалансированная \\ коллекция\end{tabular}} \\ \cline{2-5}
                       & $F_{macro}(neg, pos)$                               & $F_{micro}(neg, pos)$                              & $F_{macro}(neg, pos)$                              & $F_{micro}(neg, pos)$                             \\ \hline
    1                  & 0,384                                               & 0,4203                                             & 0,4536 (+18.1\%)                                   & 0,4982 (+18,53\%)                                 \\ \hline
    2                  & 0,3849                                              & 0,415                                              & 0,4672 (+20.9\%)                                   & 0,5029 (+21,1\%)                                  \\ \hline
    3                  & 0,3862                                              & 0,4218                                             & 0,4683 (+21.25\%)                                  & 0,5022(+19.06\%)                                  \\ \hline
    \end{tabular}
    \end{table}

    \begin{table}[!ht]
    \centering
    \caption{Результаты прогонов соревнования (задача TKK, {\it SentiRuEval-2016})}
    \label{table:tkkResult2016}
    \begin{tabular}{|c|c|c|c|c|}
    \hline
    \multirow{3}{*}{№} & \multicolumn{4}{c|}{TKK}                                                                                                                                                                                          \\ \cline{2-5}
                       & \multicolumn{2}{c|}{\begin{tabular}[c]{@{}c@{}}Не сбалансированная \\ коллекция (2015 год)\end{tabular}} & \multicolumn{2}{c|}{\begin{tabular}[c]{@{}c@{}}Расширенная сбалансированная \\ коллекция\end{tabular}} \\ \cline{2-5}
                       & $F_{macro}(neg, pos)$                               & $F_{micro}(neg, pos)$                              & $F_{macro}(neg, pos)$                              & $F_{micro}(neg, pos)$                             \\ \hline
    1                  & 0,4849                                              & 0,641                                              & 0,5103 (+5.2\%)                                    & 0,6509 (+1.5\%)                                   \\ \hline
    2                  & 0,4832                                              & 0,6473                                             & 0,5231 (+8.2\%)                                    & 0,6508 (+0.5\%)                                   \\ \hline
    3                  & 0,5099                                              & 0,677 (+2.0\%)                                     & 0,5286 (+3.6\%)                                    & 0,6632                                            \\ \hline
    \end{tabular}
    \end{table}

    \subsubsection{Улучшение результатов}

После проведения соревнований, в целях повышения качества классификации,
настройки прогонов изменялись в следующих направлениях:
\begin{enumerate}
    \item Настройка параметра C (Cost) штрафной функции SVM классификатора.
        \begin{itemize}
            \item По умолчанию C=1. Среди множества протестированных значений {
                    1, 0.75, 0.5, 0.25}, наибольший прирост достигается при C =
                    0.5 (см. таблицу \ref{table:cParameter}).
        \end{itemize}
    \item Добавление новых признаков: вычисление максимальных и минимальных
        значений (с учетом нормализации) среди всех термов сообщения по каждому
        из лексиконов.
\end{enumerate}

    \begin{table}[ht!]
    \centering
    \caption{Влияние настройки параметра Cost (С=0.5) ({\it SentiRuEval-2016})}
    \label{table:cParameter}
    \begin{tabular}{|c|c|c|c|c|}
    \hline
    \multirow{2}{*}{№} & \multicolumn{2}{c|}{\begin{tabular}[c]{@{}c@{}}BANK\\ (Расширенная сбалансированная\\ коллекция, C=0.5)\end{tabular}} & \multicolumn{2}{c|}{\begin{tabular}[c]{@{}c@{}}TTK\\ (Расширенная сбалансированная\\ коллекция, C=0.5)\end{tabular}} \\ \cline{2-5}
                       & $F_{macro}(neg, pos)$                                     & $F_{micro}(neg, pos)$                                     & $F_{macro}(neg, pos)$                                     & $F_{micro}(neg, pos)$                                    \\ \hline
    1                  & 0,4558 (+0.4\%)                                            & 0,5037 (+1.1\%)                                            & 0,5235 (+2.5\%)                                            & 0,6612 (+1.5\%)                                           \\ \hline
    2                  & 0,4795 (+2.6\%)                                            & 0,5167 (+2.7\%)                                            & 0,5338 (+2.0\%)                                            & 0,6610 (+1.5\%)                                           \\ \hline
    3                  & 0,4768 (+1.8\%)                                            & 0,5135(+2.2\%)                                             & 0,5452 (+3.1\%)                                            & 0,6733 (+1.5\%)                                           \\ \hline
    \end{tabular}
    \end{table}


    Комбинация рассмотренных выше улучшений привела к настройке {\it финальных
прогонов} (результаты представлены в таблице \ref{table:finalResults}).
Во всех прогонах использовались русскоязычные термы и хэштеги, применялись
тональные префиксы, а также учитывались все признаки. Изменения в настройках
касались только числа используемых лексиконов, а также признаков построенных
на их основе (настройки прогонов):
    \begin{enumerate}
        \item Вычисление суммы, минимума, максимума на основе лексикона №1 (см. таблицу \ref{table:createdLexicons}).
        \item Прогон №1 + признаки суммы, минимума, максимума на основе лексикона №2.
        \item Прогон №2 + признаки суммы, минимума, максимума на основе лексикона №4.
        \item Прогон №3 + признаки минимума и максимума на основе лексиконов №3.
    \end{enumerate}

    \begin{table}[ht!]
    \centering
    \caption{Результаты финального тестирования {\it SentiRuEval-2016}}
    \label{table:finalResults}
    \begin{tabular}{|c|c|c|c|c|}
    \hline
    \multirow{2}{*}{№} & \multicolumn{2}{c|}{\begin{tabular}[c]{@{}c@{}}BANK\\ (Расширенная сбалансированная\\ коллекция, C=0.5)\end{tabular}} & \multicolumn{2}{c|}{\begin{tabular}[c]{@{}c@{}}TTK\\ (Расширенная сбалансированная\\ коллекция, C=0.5)\end{tabular}} \\ \cline{2-5}
                       & $F_{macro}(neg, pos)$                                     & $F_{micro}(neg, pos)$                                     & $F_{macro}(neg, pos)$                                     & $F_{micro}(neg, pos)$                                    \\ \hline
    1                  & 0,4955                                                    & 0,5388                                                    & 0,5259                                                    & 0,6662                                                   \\ \hline
    2                  & 0,5012                                                    & 0,5379                                                    & 0,5283                                                    & 0,6720                                                   \\ \hline
    3                  & \textbf{0,5239}                                           & \textbf{0,5514}                                           & \textbf{0,5453}                                           & \textbf{0,6970}                                          \\ \hline
    4                  & 0,4818                                                    & 0,5238                                                    & 0,5356                                                    & 0,6659                                                   \\ \hline
    \end{tabular}
    \end{table}

    \subsubsection{Вывод}
Использование метаинформации на основе лексиконов стабильно повышает качество
классификации. Наибольший прирост качества достигается в случае, если классификатор
был обучен на коллекции несбалансированного типа (см. таблицу \ref{table:conclusion}\footnote{
Тип обучающей коллекции обозначается следующим образом:
$A$ --- не сбалансированная;
$B$ --- сбалансированная;
$C$ --- расширенная.})\footnote{
В таблице рассматривается прирост качества 3-его прогона по отношению к 1-ому (согласно
таблицам \ref{table:bankResult2015}-\ref{table:tkkResult2015}, и
\ref{table:bankResult2016}-\ref{table:tkkResult2016}).
В скобках указывается общий прирост качества с учетом балансировки.
}.

\begin{table}[ht!]
\begin{adjustwidth}{-1.1cm}{}
\centering
\caption{Рост качества при использовании признаков на основе лексиконов в зависимости от типа обучающей коллекции}
\label{table:conclusion}
\begin{tabular}{|c|c|c|c|c|c|}
\hline
\multicolumn{2}{|c|}{\begin{tabular}[c]{@{}c@{}c@{}}Параметры \\ обучающей \\ коллекции\end{tabular}}                                                        & \multicolumn{2}{c|}{BANK}                                                                                            & \multicolumn{2}{c|}{TKK}                                                                                          \\ \hline
Год                      & Тип         & $F_{macro}(neg, pos)$                                    & $F_{micro}(neg, pos)$                                     & $F_{macro}(neg, pos)$                                   & $F_{micro}(neg, pos)$                                   \\ \hline
\multirow{2}{*}{2015} & $A$                                                    & +12.57\%                                                  & +9.8\%                                                     & +6.8\%                                                   & +3.9\%                                                   \\ \cline{2-6}
                                  & $B$                                                       & \begin{tabular}[c]{@{}c@{}}+3.3\%\\ (+19.0\%)\end{tabular} & \begin{tabular}[c]{@{}c@{}}+4.6\%\\ (+19.8\%)\end{tabular}  & \begin{tabular}[c]{@{}c@{}}+4\%\\ (+3.4\%)\end{tabular}   & \begin{tabular}[c]{@{}c@{}}+2.7\%\\ (+1.9\%)\end{tabular} \\ \hline
\multirow{3}{*}{2016} & $A$                                                    & ---                                                      & ---                                                       & +5.1\%                                                   & +4.6\%                                                   \\ \cline{2-6}
                                  & $B$                                                       & +0.5\%                                                    & +0.03\%                                                    & ---                                                     & ---                                                     \\ \cline{2-6}
& $C$ & \begin{tabular}[c]{@{}c@{}}+4.6\%\\ (+21.95)\end{tabular} & \begin{tabular}[c]{@{}c@{}}+1.9\%\\ (+19.48\%)\end{tabular} & \begin{tabular}[c]{@{}c@{}}+4.1\%\\ (+9.0\%)\end{tabular} & \begin{tabular}[c]{@{}c@{}}+1.8\%\\ (+3.4\%)\end{tabular} \\ \hline
\end{tabular}
\end{adjustwidth}
\end{table}
В таблице \ref{table:conclusion},
значения $(+21.95)$, и $(+19.48)$ последней строки указывают на общий прирост
качества с учетом расширенной балансировки по отношению к обычной балансировке
(тестирование в этих случаях на несбалансированной коллекции не проводилось,
ввиду результатов п. \ref{sec:test2015}, таблица \ref{table:bankResult2015}).

Увеличение числа признаков по каждому из лексиконов позволяет повысить показания
таблицы \ref{table:conclusion}.
В совокупности с использованием сбалансированной обучающей коллекции и настройкой
классификатора, в рамках этой работы были получены максимальные результаты
(см. таблицу \ref{table:finalResults}, прогон №3).

В таблице \ref{table:totalImprovement} представлен прирост качества в результате
использования расширенной сбалансированной коллекции в сочетании с признаками
на основе лексиконов. Наибольший прирост достигается для задачи {\it BANK}.

\begin{table}[!ht]
\centering
\caption{Прирост качества для каждой из задач (сравнение лучшего финального результата с результатами прогона №1, {\it SentiRuEval-2016})}
\label{table:totalImprovement}
\begin{tabular}{|c|c|c|}
\hline
Прирост качества & BANK   & TKK    \\ \hline
Общий            & +36,4\% & +12,4\% \\ \hline
\end{tabular}
\end{table}
