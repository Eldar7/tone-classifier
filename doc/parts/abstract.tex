\renewcommand{\abstractname}{\Huge{Аннотация\\[1.5cm]}}
\begin{abstract}
В данной работе рассматривается применение методов машинного обучения к решению
задачи тональной классификации русскоязычных сообщений сети \twitter в сфере
банков и телекоммуникаций. В качестве подхода к классификации сообщений
рассматривается использование метода {\it <<опорных векторов>> (SVM)}. Для повышения качества
классификации было объявлено множество вспомогательных признаков, в особенности
признаки на основе лексиконов оценочных слов. Сравниваются результаты в зависимости
от типа обучающей коллекции (сбалансированная/не сбалансированная), от их объемов,
преимущество применения признаков на основе лексиконов к каждому из типов.
Была предпринята попытка участия в соревнованиях по тональной классификации
сообщений ({\it SentiRuEval-2016}). Результаты соревнований продемонстрировали устойчивое
третье место по обеим задачам. После соревнований были предприняты попытки улучшения
качества путем более тонкой настройки классификатора, а также извлечением большей
информации из лексиконов. Финальные настройки позволили добиться качества,
сравнимого с победителем соревнования.
\end{abstract}
\clearpage
