\section{Руководство пользователя}
% Какие функции выполняет приложение?
% (Позволяет согласно описанному подходу прозводить семантический анализ
% сообщений, а также измерять качество работы классификатора на основе описанных
% оценочных метрик).
Приложение предоставляет пользователю следующие возможности:
\begin{enumerate}
    \item Определение тональности сообщений тестовой коллекции по отношению к
    рассматриваемым компаниями. Для каждого сообщения коллекции, классификатор
    проставляет одну оценку из множества значений $\{1, 0, -1\}$;
    \item Вычисление качества работы классификатора на сообщениях тестовой
    коллекции с помощью сревнения полученных результатов с эталонными значениями.
    В качестве оценок используются метрики качества $F_{macro}(neg, pos)$ и
    $F_{micro}(neg, pos)$;
    \item Извлекать актуальные данные из сети {\tt Twitter} с помощью
    {\tt Streaming API}, а также инструменты для построения лексиконов на их
    основе.
\end{enumerate}
% Каким целям придерживалось приложение при разработке.
Изначально приложение проектировалось с расчетом на применение в целях
исследования описанного подхода. В связи с этим, предполагается следующий
сценарий использования --- оценка работы классификатора с целью выявления новых
признаков улучшающих качество классификации.
    \subsection{Формат входных и выходных данных приложения}
    % Здесь рассматриваются:
    %  -Формат данных, которые подаются на вход для проведения семантической
    % классификации. (XML)
    %  -Формат выходных данных (оценка качества работы классификатора
    %   (опциально), XML файл с просталвленными оценками)
    Для выполнения задачи тональной классификации сообщений, классификатору
    необходимо предоставить коллекцию данных для обучения. Поскольку за последнее
    время в России в этой области активно проводятся соревнования, организаторы
    которых заранее подготавливают коллекции данных на основе сообщений сети
    {\tt Twitter}, то в качестве формата описания коллекций
    % существует формат
    используется {\tt XML} документ, структура которого приведена в листинге ...
    % листинг с документом

    %
    % Формат описания одного сообщения + нули в тестовой коллекции
    %
    Приложение предоставляет коллекции данных банковских и телекоммуникационных
    компаний.

    %
    % Выходные данные
    %
    Одним из результатов работы классификатора является {\tt XML} документ,
    содержащий сообщения тестовой коллекции с проставленными оценками.
    Пример такого документа представлен в листинге ... .

    % [Листинг] с ответами классикатора.

    В общем случае, результирующий документ
    может исключать текст соответствующего сообщения и содержать только
    идентифицирующую информацию. В данном случае текст сообщения присутствует
    ввиду удобства просмотра результата.

    Оценка работы классфикатора представлет собой текстовое сообщение,
    содержащее оценки по метрикам  $F_{macro}(neg, pos)$ и $F_{micro}(neg, pos)$,
    а также вспомогательную информацию, на основе которой эти результаты были
    вычислены. Пример вывода результата качества работы классифкатора представлен
    в листинге ... .
    % [Листинг] с оценками работы классфикатора.

    \subsection{Настройка компонентов приложения}
    % Здесь необходимо описать, какие предварительные действия требуется выполнить,
    % чтобы пользователь смог, пользоваться приложением.
    % Добавление лексиконов в хранилище
    Сборка и эксплуатация приложения проверялась под операционной системой
    {\tt Ubuntu 14.04 x64}. Изначально необходимо выполнить установку зависимостей.
    Список комманд представлен в листинге ...
    % Установка приложения [Листинг]

        \subsubsection{Инициализация обучающих коллекций}
        По-умолчанию, приложение включает в себя данные соревнований {\tt SentiRuEval}
        2015 и 2016 годов. Для инициализации коллекций в СУБД {\tt PostgreSQL},
        пользователю необходимо выполнить последовательность действий, описанных
        в листинге ...
        % Инициализация коллекций [Листинг]

        \subsubsection{Инициализация лексиконов}
        В приложении содержит по-умолчанию лексиконы, представленные в текстовом
        формате. Для применения лексиконов в классификаторе, необходимо преобразовать
        текстовые данные в хранилище {\tt PostgreSQL}. Для этого пользователю
        достаточно выполнить следующие действия:
        % Инициализация лексиконов [Листинг]

    \subsection{Работа с приложением}
    % 1) Тестирование оценки качества работы классификатора на основе
    % подготовленных обучающих/тестовых/эталонных коллекциях.
    % (Как таковой просмотр ответов не представляет собой интересов; для этих
    % целей создана test набор коллекций (по аналогии с 2015 и 2016 годом)).
        \subsubsection{Векторизация сообщений обучающей и тестовых коллекций}
        % Т.е. есть скрипты вида train_..., test_... , которые подготавливают
        % train и test коллекции. В обоих случаях пусть генерируется словарь.
        % (Нет, они не могут быть разделены). Т.е., это неразделяемые процессы

        \subsubsection{Оценка качества работы и просмотр результатов}
        %
        % Оценка качества действительно может производиться отдельно.
        % check_
        %

        \subsubsection{Создание собственных лексиконов на основе сообщений сети
            {\tt Twitter} }
        Помимо существующих в приложении лексиконов, пользователю предоставляется
        возможность создавать их c нуля. Пользователю необходимо выполнить следующую
        последовательность действий:
        \begin{enumerate}
            \item Подключиться к приему потока сообщений сети {\tt Twitter};
            \item На основе принятого потока, отфильтровать тональные сообщения
            и создать два класса сообщений: положительные и негативные;
            \item Сгенерировать лексикон на основе полученных классов.
        \end{enumerate}

        % Листинг выполнения подключения.
        % Листинг фильтрации собщений.
        % Листинг генерации лексиконов.

