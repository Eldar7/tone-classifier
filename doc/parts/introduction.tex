\newpage
\part*{\large \centering ВВЕДЕНИЕ}
\addcontentsline{toc}{part}{ВВЕДЕНИЕ}

В настоящее время одним из наиболее популярных сервисов распространения коротких
новостей является социальная сеть {\it Twitter}.
Большинство пользователей сети часто выражают свое мнение о том, что им
понравилось или не понравилось в определенной сфере услуг.
Доступность данных сети извне дает возможность обработки и анализа
высказанных мнений.

В этой работе рассматривается построение модели на основе {\it SVM} классификатора для
определения тональности сообщений сети \twitter заданной тематики.
Подразумевается построение моделей применительно к следующим тематикам: отзывы в
банковской и телекоммуникационных сферах.
Каждое сообщение может быть отнесено к одному из трех тональных классов:
негативному, нейтральному, и положительному.

В ходе построения и настройки модели исследовались различные признаки для
представления содержания сообщений.
Особое внимание уделялось применению словарей оценочных слов для повышения
качества классификации.

\newpage
