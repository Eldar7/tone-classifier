\newpage
\section{Обзор предметной области}
    % Объяснить, какие подходы к классификации рассматриваются в этой области.

    \subsection{Подходы к тональной классификации на основе методов машинного обучения}
        \subsubsection{Метод <<Наивного Байеса>>}
        % На основе статьи
        \subsubsection{Метод <<Максимальной энтропии>>}
        % Из статьи
        \subsubsection{Метод <<Опорных векторов>>}
        % 9.1.2

        % SVM with multiple classes
    \subsection{Признаки, используемые для классификации сообщений}
        % Лемматизация термов сообщения на основе метрик
        % bag of words
        % tf-idf
        % tf-idf + вспомогательный словарь.
    \subsection{Способы построения тональных лексиконов}

    \subsection{Оценка качества классификационной модели}
    % Посмотреть обзоры из обзора соревнований.
        \subsubsection{Соревнования в области тональной классификации коротких сообщений}
        % сюда добавить SentiRuEval-2015 года.

